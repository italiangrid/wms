%\subsection{Commandline Interfaces}

In this section we describe syntax and behavior of the commands made available by the WMS-UI to allow job/DAG 
submission, monitoring and control. In the commands synopsis the mandatory arguments are showed between 
angle brackets $<$arg$>$) whilst the optional ones between square brackets ([arg]).

Commands for accessing the WMS through the WMProxy service are described in document ~\cite{WMPROXY}.
Note that usage of the WMProxy submission and control client commands is strongly recommended as they 
provide full support for all new functionality and enhancements of the WMS. 

Before going to the single commands let's have a look at how the WMS-UI can be configured.


\medskip
\subsubsection{Commands Configuration}
\label{config}

\textbf{VO-Specific}

Configuration of the WMS User Interface VO-specific parameters is accomplished through the file:

\smallskip
\begin{verbatim}
$GLITE_LOCATION/etc/<vo name>/glite_wmsui.conf 
\end{verbatim}
\smallskip


i.e. there is one directory and file for each supported VO. 

The common WMS-UI configuration rpm (\textit{glite-wms-ui-configuration}) installs the following example file:

\smallskip
\begin{verbatim}
$GLITE_LOCATION/etc/vo_template/glite_wmsui.conf
\end{verbatim}
\smallskip

If the configuration for your VO is not present on the WMS-UI machine you must create in \$GLITE\_LOCATION/etc a 
directory, named as the VO (lower-case), copy in it the above mentioned template file and update it opportunely.

The \emph{glite\_wmsui.conf} file is a classad containing the following fields:

\smallskip

\begin{itemize}
 \item \textbf{VirtualOrganisation} this is a string representing the name of the virtual organisation the file refers to.
   It should match with the name of the directory containing the file (i.e. the VO). This parameter is 
   mandatory. 
 \item \textbf{NSAddresses} this is a list of strings representing the addresses ($<$hostname$>$:$<$port$>$) of the Network 
   Servers available for the given VO. Job submission is performed towards the NS picked-up randomly from 
   the list and in case of failure it is retried on each other listed NS until succes or the end of the 
   list is reached. This parameter is mandatory.
 \item \textbf{LBAddresses} this is a list of strings or a list of lists of strings representing the address or list 
   of addresses ($<$hostname$>$:$<$port$>$) of the LB servers available for the given VO for the corresponding NS. 
   I.e. the first list of LB addresses correspond to the first NS in the NSAddresses list, the second list 
   of LB addresses correspond to the second NS in the NSAddresses list and so on.
   When job submission is performed, the WMS-UI after having chosen the NS, choses randomly one LB server within
   the corresponding list and uses it for generating the job identifier so that all information related with 
   that job will be managed by the chosen LB server. This allows distributing load on several LB servers.  
   This parameter is mandatory.
 \item \textbf{HLRLocation} this is a string representing the address ($<$hostname$>$:$<$port$>$:$<$X509contact string$>$)  of the 
   HLR for the given VO. HLR is the service responsible for managing the economic transactions and the 
   accounts of user and resources. This parameter is not mandatory. It is not present in the file by default.
   If present, it makes the WMS-UI automatically add to the job description the HLRLocation JDL attribute 
   (if not specified by the user) and this enables accounting.
 \item \textbf{MyProxyServer} this is a string representing the MYProxy server address ($<$hostname$>$) for the given VO. 
   This parameter is not mandatory. It is not present in the file by default. If present, it makes the WMS-UI 
   automatically add to the job description the MyProxyServer JDL attribute (if not specified by the user) 
   and this enables proxy renewal. If the myproxy client package is installed on the WMS-UI node, then this 
   parameter should be set equal to the MYPROXY\_SERVER environment variable.
 \item \textbf{LoggingDestination} this is a string defining the address ($<$host$>$:$[<$port$>]$;) of the LB logging service 
   (glite-lb-locallogger logging daemon ) to be targeted when logging events. The WMS-UI first checks the 
   environment for the EDG\_WL\_LOG\_DESTINATION variable and only if this is not set, the value of the 
   LoggingDestination parameter is taken into account. Otherwise the job related events are logged to the 
   LB logging service running on the WMS node.

\end{itemize}
\smallskip

Here below is provided an example of configuration file for the "atlas" Virtual Organisation. 
This implies that the file path has to be \textit{\$GLITE\_LOCATION/etc/atlas/glite\_wmsui.conf}. 

\smallskip
\begin{verbatim}

[
 VirtualOrganisation = "atlas";
 NSAddresses = {
   "ibm139.cnaf.infn.it:7772",
   "gundam.cnaf.infn.it:7772"
   };
 LBAddresses = {
   {"ibm139.cnaf.infn.it:9000"},
   {"gundam.cnaf.infn.it:9000", "neo.datamat.it:9000", "grid003.ct.infn.it:9876"}
  };
 HLRLocation = "lilith.to.infn.it:56568:/C=IT/O=INFN/OU=Personal Certificate/L=Torino/CN=Andrea 
                Guarise/Email=A.Guarise@to.infn.it";
 MyProxyServer = "skurut.cesnet.cz";
 LoggingDestination = "localhost:9002";  // local instance of LB logging service
]

\end{verbatim}
\medskip
\medskip


\textbf{Generic}


Configuration of the WMS User Interface generic parameters is accomplished through the file:

\smallskip
\begin{verbatim}
$GLITE_LOCATION/etc/glite_wmsui_cmd_var.conf
\end{verbatim}
\smallskip


Update the content of the latter file according to your needs.

The \textit{glite\_wmsui\_cmd\_var.conf} file is a classad containing the following fields:


\smallskip

\begin{itemize}
 \item \textbf{requirements} this is an expression representing the default value for the requirements expression 
   in the JDL job description. This parameter is mandatory. The value of this parameter is assigned by 
   the WMS-UI to the requirements attribute in the JDL if not specified by the user. If the user has instead 
   provided an expression for the requirements attribute in the JDL, the one specified in the configuration 
   file is added (in AND) to the existing one. 
   E.g. if in the glite\_wmsui\_cmd\_var.conf configuration file there is: 

   \begin{scriptsize}   
   \textit{requirements = other.GlueCEStateStatus == "Production" ;} 
   \end{scriptsize}

   and in the JDL file the user has specified: 

   \begin{scriptsize}   
   \textit{requirements = other.GlueCEInfoLRMSType == "PBS";} 
   \end{scriptsize}

   then the job description that is passed to the WMS contains 

   \begin{scriptsize}   
   \textit{requirements = (other.GlueCEInfoLRMSType == "PBS") \&\& (other.GlueCEStateStatus == "Production");} 
   \end{scriptsize}
      
   Obviously the setting TRUE for the requirements in the configuration file does not have any impact on the evaluation 
   of job requirements as it would result in: 

   \begin{scriptsize}   
   \textit{requirements = (other.GlueCEInfoLRMSType == "PBS") \&\& TRUE ;}
   \end{scriptsize}

 \item \textbf{rank} this is an expression representing the default value for the rank expression in the JDL job 
   description. The value of this parameter is assigned by the WMS-UI to the rank attribute in the JDL if not 
   specified by the user. This parameter is mandatory.   
 \item \textbf{RetryCount} this is an integer representing the default value for the number of submission retries for 
   a job upon failure due to some grid component (i.e. not to the job itself). The value of this parameter 
   is assigned by the WMS-UI to the RetryCount attribute in the JDL if not specified by the user.   
 \item \textbf{DefaultVo} this is a string representing the name of the virtual organisation to be taken as the user s 
   VO (VirtualOrganisation attribute in he JDL) if not specified by the user neither in the credentials 
   VOMS extension, nor directly in the job description nor through the --vo option. This attribute can be 
   either set to  unspecified  or not included at all in the file to mean that no default is set for the VO. 
 \item \textbf{ErrorStorage} this is a string representing the path of the directory where the WMS-UI creates log files. 
   This directory is not created by the WMS-UI, so It has to be an already existing directory. Default for this 
   parameter is /tmp.   
 \item \textbf{OutputStorage} this is a string defining the path of the directory where the job OutputSandbox files are 
   stored if not specified by the user through commands options. This directory is not created by the WMS-UI, 
   so It has to be an already existing directory. Default for this parameter is /tmp. 
 \item \textbf{ListenerStorage} this is a string defining the path of the directory where are created the pipes where 
   the glite\_wms\_console\_shadow process saves the job standard streams for interactive jobs. Default for 
   this parameter is /tmp.   
 \item \textbf{LoggingTimeout} this is an integer representing the timeout in seconds for asynchronous logging function 
   called by the WMS-UI when logging events to the LB. Recommended value for WMS-UI that are non-local to the 
   logging service (glite-lb-logd logging daemon) is not less than 30 seconds. 
 \item \textbf{LoggingSyncTimeout} this is an integer representing the timeout in seconds for synchronous logging function
   called by the WMS-UI when logging events to the LB. Recommended value is not less than 30 seconds.   
 \item \textbf{DefaulStatusLevel} this is an integer defining the default level of verbosity for the glite-job-status 
   command. Possible values are 0,1,2 and 3. 1 is the default.
 \item \textbf{DefaultLogInfoLevel} this is an integer defining the default level of verbosity for the glite-job-logging-info
   command. Possible values are 0,1,2 and 3. 1 is the default. 
   Default for this parameter is 0.   
 \item \textbf{NSLoggerLevel} this is an integer defining the quantity of information logged by the NS client. Possible 
   values range from 0 to 6. 0 is the defaults and means that no information is logged. Default for this 
   parameter is 0. 

\end{itemize}
\smallskip


Hereafter is provided an example of the \textit{\$GLITE\_LOCATION/etc/glite\_wmsui\_cmd\_var.conf} configuration file: 


\smallskip
\begin{verbatim}

[ 
 requirements = other.GlueCEStateStatus == "Production" ; 
 rank = - other.GlueCEStateEstimatedResponseTime ; 
 RetryCount = 3 ; 
 ErrorStorage= "/var/tmp" ; 
 OutputStorage="/tmp/jobOutput"; 
 ListenerStorage = "/tmp" 
 LoggingTimeout = 30 ; 
 LoggingSyncTimeout = 45 ;  
 DefaultStatusLevel = 1 ; 
 DefaultLogInfoLevel = 0; 
 NSLoggerLevel = 2; 
 DefaultVo = "EGEE"; 
] 

\end{verbatim}
\smallskip

The files: 

\smallskip
\begin{verbatim}
$GLITE_LOCATION/etc/glite_wmsui_cmd_err.conf
\end{verbatim}
\smallskip

and 

\smallskip
\begin{verbatim}
$GLITE_LOCATION/etc/glite_wmsui_cmd_help.conf 
\end{verbatim}
\smallskip

contain respectively the error codes and error messages returned by the WMS-UI and the text describing the 
commands usage.

\newpage
\subsubsection{Common behaviours}
\label{commonbeh}

As mentioned in the previous section~\ref{quickstart}, 
\textit{\$GLITE\_LOCATION/etc} is the WMS-UI configuration area: it includes the file specifying 
the mapping between error codes and error messages (glite\_wmsui\_cmd\_err.conf), 
the file containing the detailed description of each command (glite\_wmsui\_cmd\_help.conf) and the 
actual configuration files: glite\_wmsui\_cmd\_var.conf and $<$VO name$>$/glite\_wmsui.conf). 
The latter files are the only ones that could need to be edited and tailored according to the user/platform 
characteristics and needs. 
The \emph{glite\_wmsui\_cmd\_var.conf } file contains the following information that are read by and have 
influence on commands behaviour: 

\begin{itemize}
\item default location of the local storage areas for the Output sandbox files,
\item default location for the WMS-UI log files,
\item default values for the JDL mandatory attributes,
\item default values for timeouts when logging events to the LB,
\item default logging destination,
\item user's default VO,
\item default level of information displayed by the monitoring commands
\end{itemize}

Inside \textit{\$GLITE\_LOCATION/etc} there is instead a directory for each supported Virtual Organisation and 
named as the VO lower case e.g. for atlas we will have \$GLITE\_LOCATION/etc/atlas/) that contains a vo-specific 
configuration file glite\_wmsui.conf specifying the list of Network Servers and LBs accessible for the given VO.

When started, WMS-UI commands search for the configuration files in the following locations, in order of 
precedence:
 
\begin{itemize}
 \item \$GLITE\_WMS\_LOCATION/etc,
 \item \$GLITE\_LOCATION/etc,  
 \item /opt/glite/etc,
 \item /usr/local/etc,
 \item /etc
\end{itemize}

If none of the locations contains needed files an error is returned to the user.

Since several users on the same machine can use a single installation of the WMS-UI, people concurrently issuing 
WMS-UI commands share the same configuration files. Anyway for users (or groups of users) having particular needs 
it is possible to use "customised" WMS-UI configuration files through the --config and -config-vo options supported 
by each WMS-UI command.

Indeed every command launched specifying \emph{--config file\_path} reads its configuration settings in the file 
pointed by "file\_path" instead of the default configuration file. The same happens for the vo-specific 
configuration file if the command is started using specifying  \emph{-config-vo vo\_file\_path}. 
Hence the user only needs to create such file according to her needs and to use the appropriate options to work 
under "private" settings.

Moreover if the user wants to make this change in some way permanent avoiding the use for each issued command 
of the --config option, she can set the environment variable GLITE\_WMSUI\_CONFIG\_VAR to point to the non-standard 
path of the configuration file. Indeed if that variable is set commands will read settings from file 
"\$GLITE\_WMSUI\_CONFIG\_VAR". Anyway the --config option takes precedence on all other settings.

Exactly the same applies for the GLITE\_WMSUI\_CONFIG\_VO environment variable and the --config-vo option.

It is important to note that since the job identifiers implicitly holds the information about the LB that is 
managing the corresponding job, all the commands taking the job Id as input parameter do not take into account 
the LB addresses listed in the configuration file to perform the requested operation also if the -config-vo option 
has been specified.

Hereafter are listed the options that are common to all WMS-UI commands: 

{\begin{verbatim}
--config file_path
--config-vo file_path
--noint
--debug
--logfile file_path
--version
--help
\end{verbatim} 

The \verb!--noint! option skips all interactive questions to the user and goes ahead in the command execution. 
All warning messages and errors (if any) are written to the file 

\textbf{$<$command\_name$>$\-\_$<$UID$>$\-\_$<$PID$>$\-\_$<$date\_time$>$.log} 

in the location specified in the configuration file instead of the standard output. 
It is important to note that when \verb!--noint! is specified some checks on "dangerous actions" are skipped. 
For example if jobs cancellation is requested with this option, this action will be performed without requiring 
any confirmation to the user. The same applies if the command output will overwrite an existing file, so it is 
recommended to use the \verb!--noint! option in a safe context.

\medskip

The \verb!--debug! option is mainly thought for testing and debugging purposes; indeed it makes the commands 
print additional information while running. Every time an external API function call is encountered during the 
command execution, values of parameters passed to the API are printed to the user. The info messages are displayed 
on the standard output and are also written together with possible errors and warnings, to 

\textit{$<$command\_name$>$\-\_$<$UID$>$\-\_$<$PID$>$\-\_$<$date\-\_time$>$.log}.

\medskip

If \verb!--noint! option is specified together with \verb!--debug! option the debug message will not be printed on 
standard output.

\medskip

The \verb!--logfile! $<$file\_path$>$ option allows re-location of the commands log files in the location pointed 
by file\_path.

\medskip

The \verb!--version! and \verb!--help! options respectively make the commands display the WMS-UI current version and 
the command usage.

\medskip

Two further options that are common to almost all commands are \verb!--input! and \verb!--output!. The latter one 
makes the commands redirect the outcome to the file specified as option argument whilst the former reads a list of 
input items from the file given as option argument. The only exception is the glite\--job\--list\--match command 
that does not have the \verb!--input! option.

\medskip
\textbf{--input option}
\medskip
\smallskip
for all commands, the file given as argument to the \verb!--input! option shall contain a list of job identifiers 
in the following format: one \textit{jobId} for each line, comments beginning with a "\#" or a "*" character.  
If the input file contains only one \textit{jobId} (see the description of glite-job-submit command later in this 
document for details about \textit{jobId} format), then the request is directly submitted taking the 
\textit{jobId} as input, otherwise a menu is displayed to the user listing all the contained items, 
i.e. something like:

\smallskip
\begin{scriptsize}
\begin{verbatim}
---------------------------------------------------------------
1 : https://ibm139.cnaf.infn.it:9000/ZU9yOC7AP7AOEhMAHirG3w
2 : https://ibm139.cnaf.infn.it:9000/ZU9yOC767gJOEhMAHirG3w
3 : https://ibm135.cnaf.infn.it:9000/ZU9yOC7AP7A55TREAHirG3w
4 : https://grid012f.cnaf.infn.it:7846/ZUHY6707AP7AOEhMAHirG3w
5 : https://grid012f.cnaf.infn.it:9000/Cde341P7AOEhMAHirG3w
6 : https://ibm139.cnaf.infn.it:9000/BgT8T6H\_L92FsKq3OeTWOw
7 : https://ibm139.cnaf.infn.it:9000/lYlPBQez7fiXx9qq7BEdyw
8 : https://ibm139.cnaf.infn.it:9000/_f0Bm\_s6UdFPZIEjSglipg
a : all
q : quit
---------------------------------------------------------------
Choose one or more jobId(s) in the list - [1-10]all:
\end{verbatim} 
\end{scriptsize}
\smallskip

The user can choose one or more jobs from the list entering the corresponding numbers. Single jobs can be selected 
specifying the numbers associated to the job identifiers separated by commas. Ranges can also be selected 
specifying ends separated by a dash and it is worth mentioning that it is possible to select at the same time 
ranges and single jobs. E.g.:

\begin{itemize}
\item [2:] 	makes the command take the second listed \textit{jobId} as input
\item [1,4:]	makes the command take the first and the fourth listed \textit{jobId}s as input
\item [2-5:]	makes the command take listed \textit{jobId}s from 2 to 5 ends included) as input
\item [1,3-5,8:] selects the first job id in the list, the ids from the third to the fifth ends included) 
and finally the eighth one.
\item [all:]	makes the command take all listed \textit{jobId}s as input
\item [q:]	makes the command quit
\end{itemize}

Default value for the choice is all. 

If the \verb!--input! option is used together with the  \verb!--noint! then all \textit{jobId}s contained in 
the input file are taken into account by the command.

There are some commands whose \verb!--input! behaviour differs from the one just described. One of them is 
glite-job-submit: the input file contains in this case CEIds instead of \textit{jobId}s. 
As only one CE at a time can be the target of a submission, the user is allowed to choose one and only one CEId.
Default value for the choice is "1", i.e. the first CEId in the list. 
This is also the choice automatically made by the command when the \verb!--input! option is used together with 
the \verb!--noint! one.

The other commands are \textbf{glite-job-attach} and \textbf{glite-job-get-chkpt} whose \verb!--input! option 
allows to select one (just one) of the \textit{jobId}s contained in the input file.

\newpage

% PLEASE DO NOT MODIFY THIS FILE! It was generated by raskman version: 1.1.0
\subsubsection{glite-job-submit}
\label{glite-job-submit}

\medskip
\textbf{glite-job-submit}
\smallskip


\medskip
\textbf{SYNOPSIS}
\smallskip

\textbf{glite-job-submit [options]  $<$jdl\_file$>$}
{\begin{verbatim}

options:
	--version
	--help
	--config, -c <configfile>
	--debug
	--logfile <filepath>
	--noint
	--input, -i <filepath>
	--output, -o <filepath>
	--resource, -r <ceid>
	--nodes-resource <ceid>
	--nolisten
	--nogui
	--nomsg
	--chkpt <filepath>
	--lrms <lrmstype>
	--valid, -v <hh:mm>
	--config-vo <configfile>
	--vo <voname>
\end{verbatim}

\medskip
\textbf{DESCRIPTION}
\smallskip


glite-job-submit is the command for submitting jobs to the DataGrid and hence allows the user to run a job at one or several remote resources. glite-job-submit requires as input a job description file in which job characteristics and requirements are expressed by means of Condor class-ad-like expressions.
While it does not matter the order of the other arguments, the job description file has to be the last argument of
this command.

\medskip
\textbf{OPTIONS}
\smallskip

\textbf{--version}

displays UI version.

\textbf{--help}

displays command usage

\textbf{--config}, \textbf{-c} <configfile>

if the command is launched with this option, the configuration file pointed by configfile is used. This option is meaningless when used together with "--vo" option

\textbf{--debug}

When this option is specified, debugging information is displayed on the standard output and written into the log file, whose location is eventually printed on screen.
The default UI logfile location is:
glite-wms-job-<command\_name>\_<uid>\_<pid>\_<time>.log  located under the /var/tmp directory
please notice that this path can be overriden with the '--logfile' option

\textbf{--logfile} <filepath>

when this option is specified, all information is written into the specified file pointed by filepath.
This option will override the default location of the logfile:
glite-wms-job-<command\_name>\_<uid>\_<pid>\_<time>.log  located under the /var/tmp directory

\textbf{--noint}

if this option is specified, every interactive question to the user is skipped and the operation is continued (when possible)

\textbf{--input}, \textbf{-i} <filepath>

if this option is specified, the user will be asked to choose a CEId from a list of CEs contained in the filepath. Once a CEId has been selected the command behaves as explained for the resource option. If this option is used together with the --int one and the input file contains more than one CEId, then the first CEId in the list is taken into account for submitting the job.

\textbf{--output}, \textbf{-o} <filepath>

writes the generated jobId assigned to the submitted job in the file specified by filepath,which can be either a simple name or an absolute path (on the submitting machine). In the former case the file is created in the current working directory.

\textbf{--resource}, \textbf{-r} <ceid>

This command is available only for jobs.
if this option is specified, the job-ad sent to the NS contains a line of the type "SubmitTo = <ceid>"  and the job is submitted by the WMS to the resource identified by <ceid> without going through the match-making process.

\textbf{--nodes-resource} <ceid>

This command is available only for dags.
if this option is specified, the job-ad sent to the NS contains a line of the type "SubmitTo = <ceid>"  and the dag is submitted by the WMS to the resource identified by <ceid> without going through the match-making process for each of its nodes.

\textbf{--nolisten}

This option can be used only for interactive jobs. It makes the command forward the job standard streams coming from the WN to named pipes on the client machine whose names are returned to the user together with the OS id of the listener process. This allows the user to interact with the job through her/his own tools. It is important to note that when this option is specified, the command has no more control over the launched listener process that has hence to be killed by the user (through the returned process id) once the job is finished.

\textbf{--nogui}

This option can be used only for interactive jobs. As the command for such jobs opens an X window, the user should make sure that an X server is running on the local machine and if she/he is connected to the UI node from a remote machine (e.g. with ssh) enable secure X11 tunneling.
If this is not possible, the user can specify the --nogui option that makes the command provide a simple standard non-graphical interaction with the running job.

\textbf{--nomsg}

this option makes the command print on the standard output only the jobId generated for the job when submission was successful.The location of the log file containing massages and diagnostics is printed otherwise.

\textbf{--chkpt} <filepath>

This option can be used only for checkpointable jobs. The state specified as input is a checkpoint state generated by a previously submitted job.  This option makes the submitted job start running from the checkpoint state given in input and not from the very beginning.
The initial checkpoint states to be used with this option can be retrieved by means of the glite-job-get-chkpt command.

\textbf{--lrms} <lrmstype>

This option is only for MPICH  jobs and must be used together with either --resource or --input option; it specifies the type of the lrms of the resource the user is submitting to. When the batch system type of the specified CE resource given is not known, the lrms must be provided while submitting. For non-MPICH jobs this option will be ignored.

\textbf{--valid}, \textbf{-v} <hh:mm>

A job for which no compatible CEs have been found during the matchmaking phase is hold in the WMS Task Queue for a certain time so that it can be subjected again to matchmaking from time to time until a compatible CE is found. The JDL ExpiryTime attribute is an integer representing the date and time (in seconds since epoch)until the job request has to be considered valid by the WMS. This option allows to specify the validity in hours and minutes from submission time of the submitted JDL. When this option is used the command sets the value for the ExpiryTime attribute converting appropriately the relative timestamp provided as input. It overrides, if present,the current value. If the specified value exceeds one day from job submission then it is not taken into account by the WMS.

\textbf{--config-vo} <configfile>

if the command is launched with this option, the VO-specific configuration file pointed by configfile is used. This option is meaningless when used together with "--vo" option

\textbf{--vo} <voname>

this option allows the user to specify the name of the Virtual Organisation she/he is currently working for.
If the user proxy contains VOMS extensions then the VO specified through this option is overridden by the default VO contained in the proxy (i.e. this option is only useful when working with non-VOMS proxies).
This option is meaningless when used together with "--config-vo" option


\medskip
\textbf{EXAMPLES}
\smallskip


Upon successful submissions, this command returns to the identifier (JobId) assigned to the job

- saves the returned JobId in a file:
glite-job-submit --output jobid.out ./job.jdl

- forces the submission to the resource specified with the -r option:
glite-job-submit -r lxb1111.glite.it:2119/blah-lsf-jra1\_low ./job.jdl

- forces the submission of the DAG (the parent and all child nodes) to the resource specified with the --nodes-resources option:
glite-job-submit --nodes-resources lxb1111.glite.it:2119/blah-lsf-jra1\_low ./dag.jdl

\medskip
\textbf{ENVIRONMENT}
\smallskip


GLITE\_WMSUI\_CONFIG\_VAR:  This variable may be set to specify the path location of the custom default attribute configuration

GLITE\_WMSUI\_CONFIG\_VO: This variable may be set to specify the path location of the VO-specific configuration file

GLITE\_WMS\_LOCATION:  This variable must be set when the Glite WMS installation is not located in the default paths: either /opt/glite or /usr/local

GLITE\_LOCATION: This variable must be set when the Glite installation is not located in the default paths: either /opt/glite or /usr/local


GLOBUS\_LOCATION: This variable must be set when the Globus installation is not located in the default path /opt/globus.
It is taken into account only by submission and get-output commands

GLOBUS\_TCP\_PORT\_RANGE="<val min> <val max>" This variable must be set to define a range of ports to be used for inbound connections in the interactivity context.
It is taken into account only by submission of interactive jobs and attach commands

X509\_CERT\_DIR: This variable may be set to override the default location of the trusted certificates directory, which is normally /etc/grid-security/certificates.

X509\_USER\_PROXY: This variable may be set to override the default location of the user proxy credentials, which is normally /tmp/x509up\_u<uid>.

\medskip
\textbf{FILES}
\smallskip


One of the following paths must exist (seeked with the specified order):
- \$GLITE\_WMS\_LOCATION/etc/
- \$GLITE\_LOCATION/etc/
- /opt/glite/etc/
- /usr/local/etc/
- /etc/

and contain the following UI configuration files:
glite\_wmsui\_cmd\_var.conf, glite\_wmsui\_cmd\_err.conf, glite\_wmsui\_cmd\_help.conf, <voName>/glite\_wmsui.conf

- glite\_wmsui\_cmd\_var.conf will contain custom configuration default values
A different configuration file may be specified either by using the --config option or by setting the GLITE\_WMSUI\_CONFIG\_VAR environment variable
here follows a possible example:
[
RetryCount = 3 ;
ErrorStorage= "/tmp" ;
OutputStorage="/tmp";
ListenerStorage = "/tmp" ;
LoggingTimeout = 30 ;
LoggingSyncTimeout = 30 ;
NSLoggerLevel = 0;
DefaultStatusLevel = 1 ;
DefaultLogInfoLevel = 1;
]

- glite\_wmsui\_cmd\_err.conf will contain UI exception mapping between error codes and error messages (no relocation possible)

- glite\_wmsui\_cmd\_help.conf will contain UI long-help information (no relocation possible)

- <voName>/glite\_wmsui.conf  will contain User VO-specific attributes.
A different configuration file may be specified either by using the --config-vo option or by setting the GLITE\_WMSUI\_CONFIG\_VO environment variable
here follows a possible example:
[
LBAddresses = { "tigerman.cnaf.infn.it:9000" };
VirtualOrganisation = "egee";
NSAddresses = { "tigerman.cnaf.infn.it:7772" }
]

Besides those files, a valid proxy must be found inside the following path:
/tmp/x509up\_u<uid> ( use the X509\_USER\_PROXY environment variable to override the default location JDL file)

\medskip
\textbf{AUTHORS}
\smallskip


Alessandro Maraschini (egee@datamat.it)


\newpage

% PLEASE DO NOT MODIFY THIS FILE! It was generated by raskman version: 1.1.0
\subsubsection{glite-job-list-match}
\label{glite-job-list-match}

\medskip
\textbf{glite-job-list-match}
\smallskip


\medskip
\textbf{SYNOPSIS}
\smallskip

\textbf{glite-job-list-match [options]  $<$jdl\_file$>$}
{\begin{verbatim}

options:
	--version
	--help
	--config, -c <configfile>
	--debug
	--logfile <filepath>
	--noint
	--output, -o <filepath>
	--verbose
	--rank
	--config-vo <configfile>
	--vo <voname>
\end{verbatim}

\medskip
\textbf{DESCRIPTION}
\smallskip


glite-job-list-match displays the list of identifiers of the resources on which the user is authorized and satisfying the job requirements included in the job description file. The CE identifiers are returned either on the standard output or in a file according to the chosen command options, and are strings univocally identifying the CEs published in the IS.
The returned CEIds are listed in decreasing order of rank, i.e. the one with the best (greater) rank is in the first place and so forth.

\medskip
\textbf{OPTIONS}
\smallskip

\textbf{--version}

displays UI version.

\textbf{--help}

displays command usage

\textbf{--config}, \textbf{-c} <configfile>

if the command is launched with this option, the configuration file pointed by configfile is used. This option is meaningless when used together with "--vo" option

\textbf{--debug}

When this option is specified, debugging information is displayed on the standard output and written into the log file, whose location is eventually printed on screen.
The default UI logfile location is:
glite-wms-job-<command\_name>\_<uid>\_<pid>\_<time>.log  located under the /var/tmp directory
please notice that this path can be overriden with the '--logfile' option

\textbf{--logfile} <filepath>

when this option is specified, all information is written into the specified file pointed by filepath.
This option will override the default location of the logfile:
glite-wms-job-<command\_name>\_<uid>\_<pid>\_<time>.log  located under the /var/tmp directory

\textbf{--noint}

if this option is specified, every interactive question to the user is skipped and the operation is continued (when possible)

\textbf{--output}, \textbf{-o} <filepath>

writes the results of the operation in the file specified by filepath instead of the standard output. filepath can be either a simple name or an absolute path (on the submitting machine). In the former case the file filepath is created in the current working directory.

\textbf{--verbose}

displays on the standard output the job class-ad that is sent to the Network Server generated from the job description file.
This differs from the content of the job description file since the UI adds to it some attributes that cannot be directly inserted by the user
(e.g., defaults for Rank and Requirements if not provided, VirtualOrganisation etc).

\textbf{--rank}

displays the "matching" CEIds toghether with their associated ranking values.

\textbf{--config-vo} <configfile>

if the command is launched with this option, the VO-specific configuration file pointed by configfile is used. This option is meaningless when used together with "--vo" option

\textbf{--vo} <voname>

this option allows the user to specify the name of the Virtual Organisation she/he is currently working for.
If the user proxy contains VOMS extensions then the VO specified through this option is overridden by the default VO contained in the proxy (i.e. this option is only useful when working with non-VOMS proxies).
This option is meaningless when used together with "--config-vo" option


\medskip
\textbf{EXAMPLES}
\smallskip


1) simple request:
glite-job-list-match ./match.jdl

If the operation succeeds, the output will be a list of CEs

2) request for displays CE rank numbers:
glite-job-list-match --rank ./match.jdl

If the operation succeeds, a list of CEs with their rank numbers is displayed on the standard output

3) saves the result in a file:
glite-job-list-match --output match.out ./match.jdl

If the operation succeeds,a list of CEs is saved in the file match.out in the current working directory


\medskip
\textbf{ENVIRONMENT}
\smallskip


GLITE\_WMSUI\_CONFIG\_VAR:  This variable may be set to specify the path location of the custom default attribute configuration

GLITE\_WMSUI\_CONFIG\_VO: This variable may be set to specify the path location of the VO-specific configuration file

GLITE\_WMS\_LOCATION:  This variable must be set when the Glite WMS installation is not located in the default paths: either /opt/glite or /usr/local

GLITE\_LOCATION: This variable must be set when the Glite installation is not located in the default paths: either /opt/glite or /usr/local


GLOBUS\_LOCATION: This variable must be set when the Globus installation is not located in the default path /opt/globus.
It is taken into account only by submission and get-output commands

GLOBUS\_TCP\_PORT\_RANGE="<val min> <val max>" This variable must be set to define a range of ports to be used for inbound connections in the interactivity context.
It is taken into account only by submission of interactive jobs and attach commands

X509\_CERT\_DIR: This variable may be set to override the default location of the trusted certificates directory, which is normally /etc/grid-security/certificates.

X509\_USER\_PROXY: This variable may be set to override the default location of the user proxy credentials, which is normally /tmp/x509up\_u<uid>.

\medskip
\textbf{FILES}
\smallskip


One of the following paths must exist (seeked with the specified order):
- \$GLITE\_WMS\_LOCATION/etc/
- \$GLITE\_LOCATION/etc/
- /opt/glite/etc/
- /usr/local/etc/
- /etc/

and contain the following UI configuration files:
glite\_wmsui\_cmd\_var.conf, glite\_wmsui\_cmd\_err.conf, glite\_wmsui\_cmd\_help.conf, <voName>/glite\_wmsui.conf

- glite\_wmsui\_cmd\_var.conf will contain custom configuration default values
A different configuration file may be specified either by using the --config option or by setting the GLITE\_WMSUI\_CONFIG\_VAR environment variable
here follows a possible example:
[
RetryCount = 3 ;
ErrorStorage= "/tmp" ;
OutputStorage="/tmp";
ListenerStorage = "/tmp" ;
LoggingTimeout = 30 ;
LoggingSyncTimeout = 30 ;
NSLoggerLevel = 0;
DefaultStatusLevel = 1 ;
DefaultLogInfoLevel = 1;
]

- glite\_wmsui\_cmd\_err.conf will contain UI exception mapping between error codes and error messages (no relocation possible)

- glite\_wmsui\_cmd\_help.conf will contain UI long-help information (no relocation possible)

- <voName>/glite\_wmsui.conf  will contain User VO-specific attributes.
A different configuration file may be specified either by using the --config-vo option or by setting the GLITE\_WMSUI\_CONFIG\_VO environment variable
here follows a possible example:
[
LBAddresses = { "tigerman.cnaf.infn.it:9000" };
VirtualOrganisation = "egee";
NSAddresses = { "tigerman.cnaf.infn.it:7772" }
]

Besides those files, a valid proxy must be found inside the following path:
/tmp/x509up\_u<uid> ( use the X509\_USER\_PROXY environment variable to override the default location JDL file)

\medskip
\textbf{AUTHORS}
\smallskip


Alessandro Maraschini (egee@datamat.it)


\newpage

% PLEASE DO NOT MODIFY THIS FILE! It was generated by raskman version: 1.1.0
\subsubsection{glite-job-cancel}
\label{glite-job-cancel}

\medskip
\textbf{glite-job-cancel}
\smallskip


\medskip
\textbf{SYNOPSIS}
\smallskip

\textbf{glite-job-cancel [options]  $<$job Id(s)$>$}
{\begin{verbatim}

options:
	--version
	--help
	--config, -c <configfile>
	--debug
	--logfile <filepath>
	--noint
	--input, -i <filepath>
	--output, -o <filepath>
	--all
	--config-vo <configfile>
	--vo <voname>
\end{verbatim}

\medskip
\textbf{DESCRIPTION}
\smallskip


This command cancels a job previously submitted using glite-job-submit. Before cancellation, it prompts the user for confirmation.
The cancel request is sent to the Network Server that forwards it to the WM that fulfils it.
glite-job-cancel can remove one or more jobs: the jobs to be removed are identified by their job identifiers (jobIds returned by glite-job-submit) provided as arguments to the command and separated by a blank space.
The result of the cancel operation is reported to the user for each specified jobId.

\medskip
\textbf{OPTIONS}
\smallskip

\textbf{--version}

displays UI version.

\textbf{--help}

displays command usage

\textbf{--config}, \textbf{-c} <configfile>

if the command is launched with this option, the configuration file pointed by configfile is used. This option is meaningless when used together with "--vo" option

\textbf{--debug}

When this option is specified, debugging information is displayed on the standard output and written into the log file, whose location is eventually printed on screen.
The default UI logfile location is:
glite-wms-job-<command\_name>\_<uid>\_<pid>\_<time>.log  located under the /var/tmp directory
please notice that this path can be overriden with the '--logfile' option

\textbf{--logfile} <filepath>

when this option is specified, all information is written into the specified file pointed by filepath.
This option will override the default location of the logfile:
glite-wms-job-<command\_name>\_<uid>\_<pid>\_<time>.log  located under the /var/tmp directory

\textbf{--noint}

if this option is specified, every interactive question to the user is skipped and the operation is continued (when possible)

\textbf{--input}, \textbf{-i} <filepath>

Allow the user to select the JobId(s) from an input file located in filepath.
The list of jobIds contained in the file is displayed and the user is prompted for a choice. Single jobs can be selected specifying the numbers associated to the job identifiers separated by commas. E.g. selects the first,the third and the fifth jobId in the list.
Ranges can also be selected specifying ends separated by a dash. E.g. selects jobIds in the list from third position (included) to sixth position (included). It is worth mentioning that it is possible to select at the same time ranges and single jobs. E.g. selects the first job id in the list, the ids from the third to the fifth (ends included) and finally the eighth one.
When specified toghether with '--noint', all available JobId are selected.
This option cannot be used when one or more jobIds have been specified as extra command argument

\textbf{--output}, \textbf{-o} <filepath>

writes the results of the operation in the file specified by filepath instead of the standard output. filepath can be either a simple name or an absolute path (on the submitting machine). In the former case the file filepath is created in the current working directory.

\textbf{--all}

displays status information about all job owned by the user submitting the command. This option can't be used
either if one or more jobIds have been specified or if the --input option has been specified. All LBs
listed in the vo-specific UI configuration file \$GLITE\_WMS\_LOCATION/etc/<vo\_name>/glite\_wmsui.conf are contacted to
fulfil this request.

\textbf{--config-vo} <configfile>

if the command is launched with this option, the VO-specific configuration file pointed by configfile is used. This option is meaningless when used together with "--vo" option

\textbf{--vo} <voname>

this option allows the user to specify the name of the Virtual Organisation she/he is currently working for.
If the user proxy contains VOMS extensions then the VO specified through this option is overridden by the default VO contained in the proxy (i.e. this option is only useful when working with non-VOMS proxies).
This option is meaningless when used together with "--config-vo" option


\medskip
\textbf{EXAMPLES}
\smallskip


1) request for canceling only one job:
glite-job-cancel https://wmproxy.glite.it:9000/7O0j4Fequpg7M6SRJ-NvLg

2)	request for canceling multiple jobs:
glite-job-cancel https://wmproxy.glite.it:9000/7O0j4Fequpg7M6SRJ-NvLg https://wmproxy.glite.it:9000/wqikja\_-de83jdqd https://wmproxy.glite.it:9000/jdh\_wpwkd134ywhq6p

3)	the myids.in input file contains the jobid(s) to be cancelled (the user will be prompted for selection and confirmation)
glite-job-output --input myids.in

A message with the result of the operation is displayed on the standard output

\medskip
\textbf{ENVIRONMENT}
\smallskip


GLITE\_WMSUI\_CONFIG\_VAR:  This variable may be set to specify the path location of the custom default attribute configuration

GLITE\_WMSUI\_CONFIG\_VO: This variable may be set to specify the path location of the VO-specific configuration file

GLITE\_WMS\_LOCATION:  This variable must be set when the Glite WMS installation is not located in the default paths: either /opt/glite or /usr/local

GLITE\_LOCATION: This variable must be set when the Glite installation is not located in the default paths: either /opt/glite or /usr/local


GLOBUS\_LOCATION: This variable must be set when the Globus installation is not located in the default path /opt/globus.
It is taken into account only by submission and get-output commands

GLOBUS\_TCP\_PORT\_RANGE="<val min> <val max>" This variable must be set to define a range of ports to be used for inbound connections in the interactivity context.
It is taken into account only by submission of interactive jobs and attach commands

X509\_CERT\_DIR: This variable may be set to override the default location of the trusted certificates directory, which is normally /etc/grid-security/certificates.

X509\_USER\_PROXY: This variable may be set to override the default location of the user proxy credentials, which is normally /tmp/x509up\_u<uid>.

\medskip
\textbf{FILES}
\smallskip


One of the following paths must exist (seeked with the specified order):
- \$GLITE\_WMS\_LOCATION/etc/
- \$GLITE\_LOCATION/etc/
- /opt/glite/etc/
- /usr/local/etc/
- /etc/

and contain the following UI configuration files:
glite\_wmsui\_cmd\_var.conf, glite\_wmsui\_cmd\_err.conf, glite\_wmsui\_cmd\_help.conf, <voName>/glite\_wmsui.conf

- glite\_wmsui\_cmd\_var.conf will contain custom configuration default values
A different configuration file may be specified either by using the --config option or by setting the GLITE\_WMSUI\_CONFIG\_VAR environment variable
here follows a possible example:
[
RetryCount = 3 ;
ErrorStorage= "/tmp" ;
OutputStorage="/tmp";
ListenerStorage = "/tmp" ;
LoggingTimeout = 30 ;
LoggingSyncTimeout = 30 ;
NSLoggerLevel = 0;
DefaultStatusLevel = 1 ;
DefaultLogInfoLevel = 1;
]

- glite\_wmsui\_cmd\_err.conf will contain UI exception mapping between error codes and error messages (no relocation possible)

- glite\_wmsui\_cmd\_help.conf will contain UI long-help information (no relocation possible)

- <voName>/glite\_wmsui.conf  will contain User VO-specific attributes.
A different configuration file may be specified either by using the --config-vo option or by setting the GLITE\_WMSUI\_CONFIG\_VO environment variable
here follows a possible example:
[
LBAddresses = { "tigerman.cnaf.infn.it:9000" };
VirtualOrganisation = "egee";
NSAddresses = { "tigerman.cnaf.infn.it:7772" }
]

Besides those files, a valid proxy must be found inside the following path:
/tmp/x509up\_u<uid> ( use the X509\_USER\_PROXY environment variable to override the default location JDL file)

\medskip
\textbf{AUTHORS}
\smallskip


Alessandro Maraschini (egee@datamat.it)


\newpage

% PLEASE DO NOT MODIFY THIS FILE! It was generated by raskman version: 1.1.0
\subsubsection{glite-job-output}
\label{glite-job-output}

\medskip
\textbf{glite-job-output}
\smallskip


\medskip
\textbf{SYNOPSIS}
\smallskip

\textbf{glite-job-output [options]  $<$job Id(s)$>$}
{\begin{verbatim}

options:
	--version
	--help
	--config, -c <configfile>
	--debug
	--logfile <filepath>
	--noint
	--input, -i <filepath>
	--dir <directorypath>
\end{verbatim}

\medskip
\textbf{DESCRIPTION}
\smallskip


The glite-job-output command can be used to retrieve the output files of a job that has been submitted through the glite-job-submit command with a job description file including the OutputSandbox attribute.
After the submission, when the job has terminated its execution, the user can download the files generated by the job and temporarily stored on the RB machine as specified by the OutputSandbox attribute, issuing the glite-job-output with as input the jobId returned by the glite-job-submit.

\medskip
\textbf{OPTIONS}
\smallskip

\textbf{--version}

displays UI version.

\textbf{--help}

displays command usage

\textbf{--config}, \textbf{-c} <configfile>

if the command is launched with this option, the configuration file pointed by configfile is used. This option is meaningless when used together with "--vo" option

\textbf{--debug}

When this option is specified, debugging information is displayed on the standard output and written into the log file, whose location is eventually printed on screen.
The default UI logfile location is:
glite-wms-job-<command\_name>\_<uid>\_<pid>\_<time>.log  located under the /var/tmp directory
please notice that this path can be overriden with the '--logfile' option

\textbf{--logfile} <filepath>

when this option is specified, all information is written into the specified file pointed by filepath.
This option will override the default location of the logfile:
glite-wms-job-<command\_name>\_<uid>\_<pid>\_<time>.log  located under the /var/tmp directory

\textbf{--noint}

if this option is specified, every interactive question to the user is skipped and the operation is continued (when possible)

\textbf{--input}, \textbf{-i} <filepath>

Allow the user to select the JobId(s) from an input file located in filepath.
The list of jobIds contained in the file is displayed and the user is prompted for a choice. Single jobs can be selected specifying the numbers associated to the job identifiers separated by commas. E.g. selects the first,the third and the fifth jobId in the list.
Ranges can also be selected specifying ends separated by a dash. E.g. selects jobIds in the list from third position (included) to sixth position (included). It is worth mentioning that it is possible to select at the same time ranges and single jobs. E.g. selects the first job id in the list, the ids from the third to the fifth (ends included) and finally the eighth one.
When specified toghether with '--noint', all available JobId are selected.
This option cannot be used when one or more jobIds have been specified as extra command argument

\textbf{--dir} <directorypath>

if this option is specified, the retrieved files (previously listed by the user through the OutputSandbox attribute of the job description file) are stored in the location indicated by directorypath.


\medskip
\textbf{ENVIRONMENT}
\smallskip


GLITE\_WMSUI\_CONFIG\_VAR:  This variable may be set to specify the path location of the custom default attribute configuration

GLITE\_WMSUI\_CONFIG\_VO: This variable may be set to specify the path location of the VO-specific configuration file

GLITE\_WMS\_LOCATION:  This variable must be set when the Glite WMS installation is not located in the default paths: either /opt/glite or /usr/local

GLITE\_LOCATION: This variable must be set when the Glite installation is not located in the default paths: either /opt/glite or /usr/local


GLOBUS\_LOCATION: This variable must be set when the Globus installation is not located in the default path /opt/globus.
It is taken into account only by submission and get-output commands

GLOBUS\_TCP\_PORT\_RANGE="<val min> <val max>" This variable must be set to define a range of ports to be used for inbound connections in the interactivity context.
It is taken into account only by submission of interactive jobs and attach commands

X509\_CERT\_DIR: This variable may be set to override the default location of the trusted certificates directory, which is normally /etc/grid-security/certificates.

X509\_USER\_PROXY: This variable may be set to override the default location of the user proxy credentials, which is normally /tmp/x509up\_u<uid>.

\medskip
\textbf{FILES}
\smallskip


One of the following paths must exist (seeked with the specified order):
- \$GLITE\_WMS\_LOCATION/etc/
- \$GLITE\_LOCATION/etc/
- /opt/glite/etc/
- /usr/local/etc/
- /etc/

and contain the following UI configuration files:
glite\_wmsui\_cmd\_var.conf, glite\_wmsui\_cmd\_err.conf, glite\_wmsui\_cmd\_help.conf, <voName>/glite\_wmsui.conf

- glite\_wmsui\_cmd\_var.conf will contain custom configuration default values
A different configuration file may be specified either by using the --config option or by setting the GLITE\_WMSUI\_CONFIG\_VAR environment variable
here follows a possible example:
[
RetryCount = 3 ;
ErrorStorage= "/tmp" ;
OutputStorage="/tmp";
ListenerStorage = "/tmp" ;
LoggingTimeout = 30 ;
LoggingSyncTimeout = 30 ;
NSLoggerLevel = 0;
DefaultStatusLevel = 1 ;
DefaultLogInfoLevel = 1;
]

- glite\_wmsui\_cmd\_err.conf will contain UI exception mapping between error codes and error messages (no relocation possible)

- glite\_wmsui\_cmd\_help.conf will contain UI long-help information (no relocation possible)

- <voName>/glite\_wmsui.conf  will contain User VO-specific attributes.
A different configuration file may be specified either by using the --config-vo option or by setting the GLITE\_WMSUI\_CONFIG\_VO environment variable
here follows a possible example:
[
LBAddresses = { "tigerman.cnaf.infn.it:9000" };
VirtualOrganisation = "egee";
NSAddresses = { "tigerman.cnaf.infn.it:7772" }
]

Besides those files, a valid proxy must be found inside the following path:
/tmp/x509up\_u<uid> ( use the X509\_USER\_PROXY environment variable to override the default location JDL file)

\medskip
\textbf{AUTHORS}
\smallskip


Alessandro Maraschini (egee@datamat.it)


\newpage

% PLEASE DO NOT MODIFY THIS FILE! It was generated by raskman version: 1.1.0
\subsubsection{glite-job-attach}
\label{glite-job-attach}

\medskip
\textbf{glite-job-attach}
\smallskip


\medskip
\textbf{SYNOPSIS}
\smallskip

\textbf{glite-job-attach [options] $<$jobId$>$}
{\begin{verbatim}

options:
	--version
	--help
	--config, -c <configfile>
	--debug
	--logfile <filepath>
	--noint
	--input, -i <filepath>
	--nolisten
	--nogui
	--port, -p <<port number>>
\end{verbatim}

\medskip
\textbf{DESCRIPTION}
\smallskip


This commands starts a listener process on the UI machine (grid\_console\_shadow) that allows attaching to the standard streams of a previously submitted interactive job and displays them on a dedicated window. As the command opens a X window, the user should make sure the DISPLAY environment variable is correctly set and if she/he is connected to the UI node from remote machine (e.g. with ssh) enable secure X11 tunneling

\medskip
\textbf{OPTIONS}
\smallskip

\textbf{--version}

displays UI version.

\textbf{--help}

displays command usage

\textbf{--config}, \textbf{-c} <configfile>

if the command is launched with this option, the configuration file pointed by configfile is used. This option is meaningless when used together with "--vo" option

\textbf{--debug}

When this option is specified, debugging information is displayed on the standard output and written into the log file, whose location is eventually printed on screen.
The default UI logfile location is:
glite-wms-job-<command\_name>\_<uid>\_<pid>\_<time>.log  located under the /var/tmp directory
please notice that this path can be overriden with the '--logfile' option

\textbf{--logfile} <filepath>

when this option is specified, all information is written into the specified file pointed by filepath.
This option will override the default location of the logfile:
glite-wms-job-<command\_name>\_<uid>\_<pid>\_<time>.log  located under the /var/tmp directory

\textbf{--noint}

if this option is specified, every interactive question to the user is skipped and the operation is continued (when possible)

\textbf{--input}, \textbf{-i} <filepath>

Allow the user to select the JobId(s) from an input file located in filepath.
The list of jobIds contained in the file is displayed and the user is prompted for a choice. Single jobs can be selected specifying the numbers associated to the job identifiers separated by commas. E.g. selects the first,the third and the fifth jobId in the list.
Ranges can also be selected specifying ends separated by a dash. E.g. selects jobIds in the list from third position (included) to sixth position (included). It is worth mentioning that it is possible to select at the same time ranges and single jobs. E.g. selects the first job id in the list, the ids from the third to the fifth (ends included) and finally the eighth one.
When specified toghether with '--noint', all available JobId are selected.
This option cannot be used when one or more jobIds have been specified as extra command argument

\textbf{--nolisten}

This option can be used only for interactive jobs. It makes the command forward the job standard streams coming from the WN to named pipes on the client machine whose names are returned to the user together with the OS id of the listener process. This allows the user to interact with the job through her/his own tools. It is important to note that when this option is specified, the command has no more control over the launched listener process that has hence to be killed by the user (through the returned process id) once the job is finished.

\textbf{--nogui}

This option can be used only for interactive jobs. As the command for such jobs opens an X window, the user should make sure that an X server is running on the local machine and if she/he is connected to the UI node from a remote machine (e.g. with ssh) enable secure X11 tunneling.
If this is not possible, the user can specify the --nogui option that makes the command provide a simple standard non-graphical interaction with the running job.

\textbf{--port}, \textbf{-p} <<port number>>

make sthe command start a listener on the local machine on the specified port and logs these information to the
LB associated to the job.


\medskip
\textbf{ENVIRONMENT}
\smallskip


GLITE\_WMSUI\_CONFIG\_VAR:  This variable may be set to specify the path location of the custom default attribute configuration

GLITE\_WMSUI\_CONFIG\_VO: This variable may be set to specify the path location of the VO-specific configuration file

GLITE\_WMS\_LOCATION:  This variable must be set when the Glite WMS installation is not located in the default paths: either /opt/glite or /usr/local

GLITE\_LOCATION: This variable must be set when the Glite installation is not located in the default paths: either /opt/glite or /usr/local


GLOBUS\_LOCATION: This variable must be set when the Globus installation is not located in the default path /opt/globus.
It is taken into account only by submission and get-output commands

GLOBUS\_TCP\_PORT\_RANGE="<val min> <val max>" This variable must be set to define a range of ports to be used for inbound connections in the interactivity context.
It is taken into account only by submission of interactive jobs and attach commands

X509\_CERT\_DIR: This variable may be set to override the default location of the trusted certificates directory, which is normally /etc/grid-security/certificates.

X509\_USER\_PROXY: This variable may be set to override the default location of the user proxy credentials, which is normally /tmp/x509up\_u<uid>.

\medskip
\textbf{FILES}
\smallskip


One of the following paths must exist (seeked with the specified order):
- \$GLITE\_WMS\_LOCATION/etc/
- \$GLITE\_LOCATION/etc/
- /opt/glite/etc/
- /usr/local/etc/
- /etc/

and contain the following UI configuration files:
glite\_wmsui\_cmd\_var.conf, glite\_wmsui\_cmd\_err.conf, glite\_wmsui\_cmd\_help.conf, <voName>/glite\_wmsui.conf

- glite\_wmsui\_cmd\_var.conf will contain custom configuration default values
A different configuration file may be specified either by using the --config option or by setting the GLITE\_WMSUI\_CONFIG\_VAR environment variable
here follows a possible example:
[
RetryCount = 3 ;
ErrorStorage= "/tmp" ;
OutputStorage="/tmp";
ListenerStorage = "/tmp" ;
LoggingTimeout = 30 ;
LoggingSyncTimeout = 30 ;
NSLoggerLevel = 0;
DefaultStatusLevel = 1 ;
DefaultLogInfoLevel = 1;
]

- glite\_wmsui\_cmd\_err.conf will contain UI exception mapping between error codes and error messages (no relocation possible)

- glite\_wmsui\_cmd\_help.conf will contain UI long-help information (no relocation possible)

- <voName>/glite\_wmsui.conf  will contain User VO-specific attributes.
A different configuration file may be specified either by using the --config-vo option or by setting the GLITE\_WMSUI\_CONFIG\_VO environment variable
here follows a possible example:
[
LBAddresses = { "tigerman.cnaf.infn.it:9000" };
VirtualOrganisation = "egee";
NSAddresses = { "tigerman.cnaf.infn.it:7772" }
]

Besides those files, a valid proxy must be found inside the following path:
/tmp/x509up\_u<uid> ( use the X509\_USER\_PROXY environment variable to override the default location JDL file)

\medskip
\textbf{AUTHORS}
\smallskip


Alessandro Maraschini (egee@datamat.it)


\newpage

% PLEASE DO NOT MODIFY THIS FILE! It was generated by raskman version: 1.1.0
\subsubsection{glite-job-get-chkpt}
\label{glite-job-get-chkpt}

\medskip
\textbf{glite-job-get-chkpt}
\smallskip


\medskip
\textbf{SYNOPSIS}
\smallskip

\textbf{glite-job-get-chkpt [options] $<$jobId$>$}
{\begin{verbatim}

options:
	--version
	--help
	--config, -c <configfile>
	--debug
	--logfile <filepath>
	--noint
	--input, -i <filepath>
	--output, -o <filepath>
	--cs <chkptStep>
\end{verbatim}

\medskip
\textbf{DESCRIPTION}
\smallskip


This commands allows the user to retrieve one or more checkpoint states saved by a previously submitted job.
Checkpoint states are retrieved from the LB server and are saved locally into a file in JDL format.

\medskip
\textbf{OPTIONS}
\smallskip

\textbf{--version}

displays UI version.

\textbf{--help}

displays command usage

\textbf{--config}, \textbf{-c} <configfile>

if the command is launched with this option, the configuration file pointed by configfile is used. This option is meaningless when used together with "--vo" option

\textbf{--debug}

When this option is specified, debugging information is displayed on the standard output and written into the log file, whose location is eventually printed on screen.
The default UI logfile location is:
glite-wms-job-<command\_name>\_<uid>\_<pid>\_<time>.log  located under the /var/tmp directory
please notice that this path can be overriden with the '--logfile' option

\textbf{--logfile} <filepath>

when this option is specified, all information is written into the specified file pointed by filepath.
This option will override the default location of the logfile:
glite-wms-job-<command\_name>\_<uid>\_<pid>\_<time>.log  located under the /var/tmp directory

\textbf{--noint}

if this option is specified, every interactive question to the user is skipped and the operation is continued (when possible)

\textbf{--input}, \textbf{-i} <filepath>

Allow the user to select the JobId(s) from an input file located in filepath.
The list of jobIds contained in the file is displayed and the user is prompted for a choice. Single jobs can be selected specifying the numbers associated to the job identifiers separated by commas. E.g. selects the first,the third and the fifth jobId in the list.
Ranges can also be selected specifying ends separated by a dash. E.g. selects jobIds in the list from third position (included) to sixth position (included). It is worth mentioning that it is possible to select at the same time ranges and single jobs. E.g. selects the first job id in the list, the ids from the third to the fifth (ends included) and finally the eighth one.
When specified toghether with '--noint', all available JobId are selected.
This option cannot be used when one or more jobIds have been specified as extra command argument

\textbf{--output}, \textbf{-o} <filepath>

writes the results of the operation in the file specified by filepath instead of the standard output. filepath can be either a simple name or an absolute path (on the submitting machine). In the former case the file filepath is created in the current working directory.

\textbf{--cs} <chkptStep>

if the command is launched with this option then it retrieves the "last but state\_num" state saved by the job.
Last saved state is returned if the option is not used (equivalent to state\_num = 0).


\medskip
\textbf{ENVIRONMENT}
\smallskip


GLITE\_WMSUI\_CONFIG\_VAR:  This variable may be set to specify the path location of the custom default attribute configuration

GLITE\_WMSUI\_CONFIG\_VO: This variable may be set to specify the path location of the VO-specific configuration file

GLITE\_WMS\_LOCATION:  This variable must be set when the Glite WMS installation is not located in the default paths: either /opt/glite or /usr/local

GLITE\_LOCATION: This variable must be set when the Glite installation is not located in the default paths: either /opt/glite or /usr/local


GLOBUS\_LOCATION: This variable must be set when the Globus installation is not located in the default path /opt/globus.
It is taken into account only by submission and get-output commands

GLOBUS\_TCP\_PORT\_RANGE="<val min> <val max>" This variable must be set to define a range of ports to be used for inbound connections in the interactivity context.
It is taken into account only by submission of interactive jobs and attach commands

X509\_CERT\_DIR: This variable may be set to override the default location of the trusted certificates directory, which is normally /etc/grid-security/certificates.

X509\_USER\_PROXY: This variable may be set to override the default location of the user proxy credentials, which is normally /tmp/x509up\_u<uid>.

\medskip
\textbf{FILES}
\smallskip


One of the following paths must exist (seeked with the specified order):
- \$GLITE\_WMS\_LOCATION/etc/
- \$GLITE\_LOCATION/etc/
- /opt/glite/etc/
- /usr/local/etc/
- /etc/

and contain the following UI configuration files:
glite\_wmsui\_cmd\_var.conf, glite\_wmsui\_cmd\_err.conf, glite\_wmsui\_cmd\_help.conf, <voName>/glite\_wmsui.conf

- glite\_wmsui\_cmd\_var.conf will contain custom configuration default values
A different configuration file may be specified either by using the --config option or by setting the GLITE\_WMSUI\_CONFIG\_VAR environment variable
here follows a possible example:
[
RetryCount = 3 ;
ErrorStorage= "/tmp" ;
OutputStorage="/tmp";
ListenerStorage = "/tmp" ;
LoggingTimeout = 30 ;
LoggingSyncTimeout = 30 ;
NSLoggerLevel = 0;
DefaultStatusLevel = 1 ;
DefaultLogInfoLevel = 1;
]

- glite\_wmsui\_cmd\_err.conf will contain UI exception mapping between error codes and error messages (no relocation possible)

- glite\_wmsui\_cmd\_help.conf will contain UI long-help information (no relocation possible)

- <voName>/glite\_wmsui.conf  will contain User VO-specific attributes.
A different configuration file may be specified either by using the --config-vo option or by setting the GLITE\_WMSUI\_CONFIG\_VO environment variable
here follows a possible example:
[
LBAddresses = { "tigerman.cnaf.infn.it:9000" };
VirtualOrganisation = "egee";
NSAddresses = { "tigerman.cnaf.infn.it:7772" }
]

Besides those files, a valid proxy must be found inside the following path:
/tmp/x509up\_u<uid> ( use the X509\_USER\_PROXY environment variable to override the default location JDL file)

\medskip
\textbf{AUTHORS}
\smallskip


Alessandro Maraschini (egee@datamat.it)


\newpage

% PLEASE DO NOT MODIFY THIS FILE! It was generated by raskman version: 1.1.0
\subsubsection{glite-job-status}
\label{glite-job-status}

\medskip
\textbf{glite-job-status}
\smallskip


\medskip
\textbf{SYNOPSIS}
\smallskip

\textbf{glite-job-status [options] $<$jobId$>$}
{\begin{verbatim}

options:
	--version
	--help
	--config, -c <configfile>
	--debug
	--logfile <filepath>
	--noint
	--input, -i <filepath>
	--output, -o <filepath>
	--all
	--config-vo <configfile>
	--verbosity, -v <level>
	--from <[MM:DD:]hh:mm[:[CC]YY]>
	--to <[MM:DD:]hh:mm[:[CC]YY]>
	--user-tag <<tag name>=<tag value>>
	--status, -s <<status code>>
	--exclude, -e <<status code>>
	--nonodes
\end{verbatim}

\medskip
\textbf{DESCRIPTION}
\smallskip


This command prints the status of a job previously submitted using glite-job-submit.
The job status request is sent to the LB that provides the requested information.
This can be done during the whole job life.
glite-job-status can monitor one or more jobs: the jobs to be checked are identified by one or more job identifiers (jobIds returned by glite-job-submit) provided as arguments to the command and separated by a blank space.

\medskip
\textbf{OPTIONS}
\smallskip

\textbf{--version}

displays UI version.

\textbf{--help}

displays command usage

\textbf{--config}, \textbf{-c} <configfile>

if the command is launched with this option, the configuration file pointed by configfile is used. This option is meaningless when used together with "--vo" option

\textbf{--debug}

When this option is specified, debugging information is displayed on the standard output and written into the log file, whose location is eventually printed on screen.
The default UI logfile location is:
glite-wms-job-<command\_name>\_<uid>\_<pid>\_<time>.log  located under the /var/tmp directory
please notice that this path can be overriden with the '--logfile' option

\textbf{--logfile} <filepath>

when this option is specified, all information is written into the specified file pointed by filepath.
This option will override the default location of the logfile:
glite-wms-job-<command\_name>\_<uid>\_<pid>\_<time>.log  located under the /var/tmp directory

\textbf{--noint}

if this option is specified, every interactive question to the user is skipped and the operation is continued (when possible)

\textbf{--input}, \textbf{-i} <filepath>

Allow the user to select the JobId(s) from an input file located in filepath.
The list of jobIds contained in the file is displayed and the user is prompted for a choice. Single jobs can be selected specifying the numbers associated to the job identifiers separated by commas. E.g. selects the first,the third and the fifth jobId in the list.
Ranges can also be selected specifying ends separated by a dash. E.g. selects jobIds in the list from third position (included) to sixth position (included). It is worth mentioning that it is possible to select at the same time ranges and single jobs. E.g. selects the first job id in the list, the ids from the third to the fifth (ends included) and finally the eighth one.
When specified toghether with '--noint', all available JobId are selected.
This option cannot be used when one or more jobIds have been specified as extra command argument

\textbf{--output}, \textbf{-o} <filepath>

writes the results of the operation in the file specified by filepath instead of the standard output. filepath can be either a simple name or an absolute path (on the submitting machine). In the former case the file filepath is created in the current working directory.

\textbf{--all}

displays status information about all job owned by the user submitting the command. This option can't be used
either if one or more jobIds have been specified or if the --input option has been specified. All LBs
listed in the vo-specific UI configuration file \$GLITE\_WMS\_LOCATION/etc/<vo\_name>/glite\_wmsui.conf are contacted to
fulfil this request.

\textbf{--config-vo} <configfile>

if the command is launched with this option, the VO-specific configuration file pointed by configfile is used. This option is meaningless when used together with "--vo" option

\textbf{--verbosity}, \textbf{-v} <level>

sets the detail level of information about the job displayed to the user. Possible values for verb\_level are 0 (only JobId and status/event displayed),1 (timestamp and source information added), 2 (all information but jdls displayed), 3 (complete information containing all Jdl strings)

\textbf{--from} <[MM:DD:]hh:mm[:[CC]YY]>

makes the command query LB for jobs that have been submitted (more precisely entered the "Submitted" status) after the specified date/time.
If only hours and minutes are specified then the current day is taken into account. If the year is not specified then the current year is taken into account.

\textbf{--to} <[MM:DD:]hh:mm[:[CC]YY]>

makes the command query LB for jobs that have been submitted (more precisely entered the "Submitted" status) before the specified date/time.
If only hours and minutes are specified then the current day is taken into account.
If the year is not specified then the current year is taken into account.

\textbf{--user-tag} <<tag name>=<tag value>>

makes the command include only jobs that have defined specified usertag name and value

\textbf{--status}, \textbf{-s} <<status code>>

makes the command query LB for jobs that are in the specified status.
The status value can be either an integer or a (case insensitive) string; the following possible values are allowed:
UNDEF (0), SUBMITTED(1), WAITING(2), READY(3), SCHEDULED(4), RUNNING(5), DONE(6), CLEARED(7), ABORTED(8), CANCELLED(9),
UNKNOWN(10), PURGED(11).
This option can be repeated several times, all status conditions will be considered as in a logical OR operation

(i.e.  -s SUBMITTED --status 3  will query all jobs that are either in SUBMITTED or in READY status)

\textbf{--exclude}, \textbf{-e} <<status code>>

makes the command query LB for jobs that are NOT in the specified status.
The status value can be either an integer or a (case insensitive) string; the following possible values are allowed:
UNDEF (0), SUBMITTED(1), WAITING(2), READY(3), SCHEDULED(4), RUNNING(5), DONE(6), CLEARED(7), ABORTED(8), CANCELLED(9),
UNKNOWN(10), PURGED(11).
This option can be repeated several times, all status conditions will be considered as in a logical AND operation

(i.e.  -e SUBMITTED --exclude 3  will query all jobs that are neither in SUBMITTED nor in READY status)

\textbf{--nonodes}

This option will not display any information of (if present) sub jobs of any dag, only requested JobId(s) info will be taken into account


\medskip
\textbf{ENVIRONMENT}
\smallskip


GLITE\_WMSUI\_CONFIG\_VAR:  This variable may be set to specify the path location of the custom default attribute configuration

GLITE\_WMSUI\_CONFIG\_VO: This variable may be set to specify the path location of the VO-specific configuration file

GLITE\_WMS\_LOCATION:  This variable must be set when the Glite WMS installation is not located in the default paths: either /opt/glite or /usr/local

GLITE\_LOCATION: This variable must be set when the Glite installation is not located in the default paths: either /opt/glite or /usr/local


GLOBUS\_LOCATION: This variable must be set when the Globus installation is not located in the default path /opt/globus.
It is taken into account only by submission and get-output commands

GLOBUS\_TCP\_PORT\_RANGE="<val min> <val max>" This variable must be set to define a range of ports to be used for inbound connections in the interactivity context.
It is taken into account only by submission of interactive jobs and attach commands

X509\_CERT\_DIR: This variable may be set to override the default location of the trusted certificates directory, which is normally /etc/grid-security/certificates.

X509\_USER\_PROXY: This variable may be set to override the default location of the user proxy credentials, which is normally /tmp/x509up\_u<uid>.

\medskip
\textbf{FILES}
\smallskip


One of the following paths must exist (seeked with the specified order):
- \$GLITE\_WMS\_LOCATION/etc/
- \$GLITE\_LOCATION/etc/
- /opt/glite/etc/
- /usr/local/etc/
- /etc/

and contain the following UI configuration files:
glite\_wmsui\_cmd\_var.conf, glite\_wmsui\_cmd\_err.conf, glite\_wmsui\_cmd\_help.conf, <voName>/glite\_wmsui.conf

- glite\_wmsui\_cmd\_var.conf will contain custom configuration default values
A different configuration file may be specified either by using the --config option or by setting the GLITE\_WMSUI\_CONFIG\_VAR environment variable
here follows a possible example:
[
RetryCount = 3 ;
ErrorStorage= "/tmp" ;
OutputStorage="/tmp";
ListenerStorage = "/tmp" ;
LoggingTimeout = 30 ;
LoggingSyncTimeout = 30 ;
NSLoggerLevel = 0;
DefaultStatusLevel = 1 ;
DefaultLogInfoLevel = 1;
]

- glite\_wmsui\_cmd\_err.conf will contain UI exception mapping between error codes and error messages (no relocation possible)

- glite\_wmsui\_cmd\_help.conf will contain UI long-help information (no relocation possible)

- <voName>/glite\_wmsui.conf  will contain User VO-specific attributes.
A different configuration file may be specified either by using the --config-vo option or by setting the GLITE\_WMSUI\_CONFIG\_VO environment variable
here follows a possible example:
[
LBAddresses = { "tigerman.cnaf.infn.it:9000" };
VirtualOrganisation = "egee";
NSAddresses = { "tigerman.cnaf.infn.it:7772" }
]

Besides those files, a valid proxy must be found inside the following path:
/tmp/x509up\_u<uid> ( use the X509\_USER\_PROXY environment variable to override the default location JDL file)

\medskip
\textbf{AUTHORS}
\smallskip


Alessandro Maraschini (egee@datamat.it)


\newpage

% PLEASE DO NOT MODIFY THIS FILE! It was generated by raskman version: 1.1.0
\subsubsection{glite-job-logging-info}
\label{glite-job-logging-info}

\medskip
\textbf{glite-job-logging-info}
\smallskip


\medskip
\textbf{SYNOPSIS}
\smallskip

\textbf{glite-job-logging-info [options] $<$jobId$>$}
{\begin{verbatim}

options:
	--version
	--help
	--config, -c <configfile>
	--debug
	--logfile <filepath>
	--noint
	--input, -i <filepath>
	--output, -o <filepath>
	--config-vo <configfile>
	--verbosity, -v <level>
	--from <[MM:DD:]hh:mm[:[CC]YY]>
	--to <[MM:DD:]hh:mm[:[CC]YY]>
	--user-tag <<tag name>=<tag value>>
	--event <<event code>>
	--exclude, -e <<event code>>
\end{verbatim}

\medskip
\textbf{DESCRIPTION}
\smallskip


This command queries the LB persistent DB for logging information about jobs previously submitted using glite-job-submit.
The job logging information are stored permanently by the LB service and can be retrieved also after the job has terminated its life-cycle, differently from the bookkeeping information that are in some way "consumed" by the user during the job existence.

\medskip
\textbf{OPTIONS}
\smallskip

\textbf{--version}

displays UI version.

\textbf{--help}

displays command usage

\textbf{--config}, \textbf{-c} <configfile>

if the command is launched with this option, the configuration file pointed by configfile is used. This option is meaningless when used together with "--vo" option

\textbf{--debug}

When this option is specified, debugging information is displayed on the standard output and written into the log file, whose location is eventually printed on screen.
The default UI logfile location is:
glite-wms-job-<command\_name>\_<uid>\_<pid>\_<time>.log  located under the /var/tmp directory
please notice that this path can be overriden with the '--logfile' option

\textbf{--logfile} <filepath>

when this option is specified, all information is written into the specified file pointed by filepath.
This option will override the default location of the logfile:
glite-wms-job-<command\_name>\_<uid>\_<pid>\_<time>.log  located under the /var/tmp directory

\textbf{--noint}

if this option is specified, every interactive question to the user is skipped and the operation is continued (when possible)

\textbf{--input}, \textbf{-i} <filepath>

Allow the user to select the JobId(s) from an input file located in filepath.
The list of jobIds contained in the file is displayed and the user is prompted for a choice. Single jobs can be selected specifying the numbers associated to the job identifiers separated by commas. E.g. selects the first,the third and the fifth jobId in the list.
Ranges can also be selected specifying ends separated by a dash. E.g. selects jobIds in the list from third position (included) to sixth position (included). It is worth mentioning that it is possible to select at the same time ranges and single jobs. E.g. selects the first job id in the list, the ids from the third to the fifth (ends included) and finally the eighth one.
When specified toghether with '--noint', all available JobId are selected.
This option cannot be used when one or more jobIds have been specified as extra command argument

\textbf{--output}, \textbf{-o} <filepath>

writes the results of the operation in the file specified by filepath instead of the standard output. filepath can be either a simple name or an absolute path (on the submitting machine). In the former case the file filepath is created in the current working directory.

\textbf{--config-vo} <configfile>

if the command is launched with this option, the VO-specific configuration file pointed by configfile is used. This option is meaningless when used together with "--vo" option

\textbf{--verbosity}, \textbf{-v} <level>

sets the detail level of information about the job displayed to the user. Possible values for verb\_level are 0 (only JobId and status/event displayed),1 (timestamp and source information added), 2 (all information but jdls displayed), 3 (complete information containing all Jdl strings)

\textbf{--from} <[MM:DD:]hh:mm[:[CC]YY]>

makes the command query LB for jobs that have been submitted (more precisely entered the "Submitted" status) after the specified date/time.
If only hours and minutes are specified then the current day is taken into account. If the year is not specified then the current year is taken into account.

\textbf{--to} <[MM:DD:]hh:mm[:[CC]YY]>

makes the command query LB for jobs that have been submitted (more precisely entered the "Submitted" status) before the specified date/time.
If only hours and minutes are specified then the current day is taken into account.
If the year is not specified then the current year is taken into account.

\textbf{--user-tag} <<tag name>=<tag value>>

makes the command include only jobs that have defined specified usertag name and value

\textbf{--event} <<event code>>

makes the command query specified events for requested jobid(s)
The event code can be either an integer or a (case insensitive) string; the following possible values are allowed:
UNDEF, TRANSFER, ACCEPTED, REFUSED, ENQUEUED, DEQUEUED, HELPERCALL, HELPERRETURN, RUNNING, RESUBMISSION, DONE,
CANCEL, ABORT, CLEAR, PURGE, MATCH, PENDING, REGJOB, CHKPT, LISTENER, CURDESCR, USERTAG, CHANGEACL, NOTIFICATION,
RESOURCEUSAGE, REALLYRUNNING
This option can be repeated several times, all event conditions will be considered as in a logical OR operation

(i.e.  --event  PURGE --event 4  will query, for specified jobid(s), all PURGE and ENQUEUED events)

\textbf{--exclude}, \textbf{-e} <<event code>>

makes the command exclude specified events for requested jobid(s)
The event code can be either an integer or a (case insensitive) string; the following possible values are allowed:
UNDEF, TRANSFER, ACCEPTED, REFUSED, ENQUEUED, DEQUEUED, HELPERCALL, HELPERRETURN, RUNNING, RESUBMISSION, DONE,
CANCEL, ABORT, CLEAR, PURGE, MATCH, PENDING, REGJOB, CHKPT, LISTENER, CURDESCR, USERTAG, CHANGEACL, NOTIFICATION,
RESOURCEUSAGE, REALLYRUNNING
This option can be repeated several times, all event conditions will be considered as in a logical AND operation

(i.e.  -e PURGE --exclude 4  will query, for specified jobid(s), all events BUT PURGE and ENQUEUED)


\medskip
\textbf{ENVIRONMENT}
\smallskip


GLITE\_WMSUI\_CONFIG\_VAR:  This variable may be set to specify the path location of the custom default attribute configuration

GLITE\_WMSUI\_CONFIG\_VO: This variable may be set to specify the path location of the VO-specific configuration file

GLITE\_WMS\_LOCATION:  This variable must be set when the Glite WMS installation is not located in the default paths: either /opt/glite or /usr/local

GLITE\_LOCATION: This variable must be set when the Glite installation is not located in the default paths: either /opt/glite or /usr/local


GLOBUS\_LOCATION: This variable must be set when the Globus installation is not located in the default path /opt/globus.
It is taken into account only by submission and get-output commands

GLOBUS\_TCP\_PORT\_RANGE="<val min> <val max>" This variable must be set to define a range of ports to be used for inbound connections in the interactivity context.
It is taken into account only by submission of interactive jobs and attach commands

X509\_CERT\_DIR: This variable may be set to override the default location of the trusted certificates directory, which is normally /etc/grid-security/certificates.

X509\_USER\_PROXY: This variable may be set to override the default location of the user proxy credentials, which is normally /tmp/x509up\_u<uid>.

\medskip
\textbf{FILES}
\smallskip


One of the following paths must exist (seeked with the specified order):
- \$GLITE\_WMS\_LOCATION/etc/
- \$GLITE\_LOCATION/etc/
- /opt/glite/etc/
- /usr/local/etc/
- /etc/

and contain the following UI configuration files:
glite\_wmsui\_cmd\_var.conf, glite\_wmsui\_cmd\_err.conf, glite\_wmsui\_cmd\_help.conf, <voName>/glite\_wmsui.conf

- glite\_wmsui\_cmd\_var.conf will contain custom configuration default values
A different configuration file may be specified either by using the --config option or by setting the GLITE\_WMSUI\_CONFIG\_VAR environment variable
here follows a possible example:
[
RetryCount = 3 ;
ErrorStorage= "/tmp" ;
OutputStorage="/tmp";
ListenerStorage = "/tmp" ;
LoggingTimeout = 30 ;
LoggingSyncTimeout = 30 ;
NSLoggerLevel = 0;
DefaultStatusLevel = 1 ;
DefaultLogInfoLevel = 1;
]

- glite\_wmsui\_cmd\_err.conf will contain UI exception mapping between error codes and error messages (no relocation possible)

- glite\_wmsui\_cmd\_help.conf will contain UI long-help information (no relocation possible)

- <voName>/glite\_wmsui.conf  will contain User VO-specific attributes.
A different configuration file may be specified either by using the --config-vo option or by setting the GLITE\_WMSUI\_CONFIG\_VO environment variable
here follows a possible example:
[
LBAddresses = { "tigerman.cnaf.infn.it:9000" };
VirtualOrganisation = "egee";
NSAddresses = { "tigerman.cnaf.infn.it:7772" }
]

Besides those files, a valid proxy must be found inside the following path:
/tmp/x509up\_u<uid> ( use the X509\_USER\_PROXY environment variable to override the default location JDL file)

\medskip
\textbf{AUTHORS}
\smallskip


Alessandro Maraschini (egee@datamat.it)


\newpage


