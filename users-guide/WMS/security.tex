%\subsubsection {Security} 
 
For the Grid to be an effective framework for largely distributed computation, users, user processes 
and grid services must work in a secure environment.
 
Due to this, all interactions between WMS components, especially those that are network separated, will be 
mutually authenticated: depending on the specific interaction, an entity authenticates itself to the 
other peer using either its own credential or a delegated user credential or both. For example when the 
User Interface passes a job to the Network Server, the WMS-UI authenticates using a proxy user credential 
(a proxy certificate) whereas the NS uses its own service credential. The same happens when the WMS-UI 
interacts with the Logging and Bookkeeping service. The WMS-UI uses a proxy user credential to limit the risk 
of compromising the original credential in the hands of the user. 
 
The user or service identity and their public key are included in a X.509 certificate signed by a trusted 
Certification Authority (CA), whose purpose is to guarantee the association between that public key and its 
owner.

According to what just premised, to take advantage of WMS-UI commands the user has to possess a valid X.509 
certificate on the submitting machine, consisting of two files: the certificate file and the private 
key file. The location of the two mentioned files is assumed to be either pointed to respectively 
\textit{"\$X509\_USER\_CERT"} and \textit{"\$X509\_USER\_KEY"} or 
by \textit{"\$HOME\-/.globus\-/usercert.pem"} and \textit{"\$HOME\-/.globus/\-userkey.pem"} if the 
X509 environment variables are not set. 

\medskip


\framebox{
\parbox[c]{15cm}{
 
\textbf{How to extract userkey.pem and usercert.pem from your certificate}\medskip

Usually X.509 Certificates are downloaded using a browser and managed by the browser itself. 
Anyway it is possible to export your certificate in a file PKCS12 (which will probably have the 
extension .p12 or .pfx). 
 
The procedure to export the certificate vary from browser to browser, for example Internet Explorer 
starts with "Tools $->$ Internet Options $->$ Content"; Netscape Communicator has a "Security" button 
on the top menu bar; Mozilla starts with "Edit $->$ Preferences $->$ Privacy and Security $->$ Certificates" 
and Firefox has "Edit $->$ Preferences $->$ Advanced $->$ Certificates $->$ manage certificates $->$ backup".
  
Unfortunately PKCS12 format is not accepted by Globus security infrastructure, but you can easily convert 
it into the supported standard (PEM). This operation will split your *.p12 file in two files: the certificate 
(usercert.pm) and the private key (userkey.pm). The conversion can be performed with openssl tool:

\medskip

{\scriptsize{ 
\$ openssl pkcs12 -nocerts -in mycert.p12 -out userkey.pem\\
\$ openssl pkcs12 -clcerts -nokeys -in mycert.p12 -out usercert.pem\\
\$ chmod 0400 userkey.pem\\
\$ chmod 0600 usercert.pem\\
 }}
\medskip

Permission must be set as shown not only for security reasons: \emph{voms-proxy-init} and \emph{grid-proxy-init}
commands will fail if your private key is not protected as listed above.
}} 
%end of framebox and parbox
\medskip

Actually the user certificate and private key files are not mandatory on the WMS-UI machine; indeed they are 
only needed for the creation of the proxy user credentials through the \textit{grid-proxy-init} or 
\textit{voms-proxy-init} commands. Please refer to the VOMS User's Guide ~\cite{voms-core} for detailed 
information about the VOMS client commands.  

Alternatively you can download the proxy credentials from a trusted site and work with it without having 
the cert and key available locally.

All WMS-UI commands, when started, check for the existence and expiration date of a user proxy credentials in 
the location pointed to by \textit{"\$X509\_USER\_PROXY"} or in \textit{"/tmp/\-x509up\_u$<$UID$>$"} 
(where $<$UID$>$ is the user identifier in the submitting machine OS) if the X509 environment variable is 
not set. If the proxy certificate does not exist or has expired the WMS-UI returns an error message to the user 
and exits.

It is important to note that besides authentication, proxy credential issued by VOMS, i.e. containing 
FQANs (Fully Qualified Attribute Names ~\cite{voms-core}), are used by the WMS-UI to get the VO the user is currently working for. 
If a given proxy credential contains attributes for more then one VO, than the default one (i.e. the one first 
position) is considered.   

Once a job has been submitted by the WMS-UI, it passes through several components of the WMS 
(e.g. the NS, the WM, the JC, CondorC etc.) before it completes its execution. At each step operations 
that are related with the job could require authentication by a certificate. For example during the 
scheduling phase, the WM needs to get some information about the user who wants to schedule a job and 
the certificate of the user could be needed to access this information. Similarly, a valid user's certificate 
is needed by JC/CondorC to submit a job to the CE. Moreover JC has to be able to repeat this process e.g. 
in case of crashing of the CE which the job is running on, therefore, a valid user's certificate is needed 
for all the job lifetime.

A job gets associated a valid proxy certificate (the submitting user's one) when it is submitted by the WMS-UI to 
NS. Validity of such a certificate is set by default to 12 hours unless a longer validity is explicitly 
requested by the user when generating the proxy. Problems could occur if the job spends on CE (in a 
queue or running) more time than lifetime of its proxy certificate.

In order to submit long-running jobs, users can either generate proxy credentials using the respectively 
the \verb!--valid! and \verb!--hours! of the \textit{voms-proxy-init} and \textit{grid-proxy-init} commands 
or (more safely) rely on the features of the MyProxy package (see section~ref{myproxy}). The underlying idea 
is that the user registers in a MyProxy server a valid long-term certificate proxy that will be used by the 
WMS to perform a periodic credential renewal for the submitted job; in this way the user is no longer obliged 
to create very long lifetime proxies when submitting jobs lasting for a great amount of time. 
A more detailed description of this mechanism is provided in the following section~\ref{myproxy}.

\subsubsection {MyProxy} 
\label{myproxy}

The MyProxy credential repository system consists of a server and a set of client tools that can be used 
to delegate and retrieve credentials to and from a server. Normally, a user would start by using the 
\textit{myproxy\_init} client program along with the permanent credentials necessary to contact the server 
and delegate a set of proxy credentials to the server along with authentication information and retrieval 
restrictions.

\medskip
\textbf{MyProxy Client}
\medskip

The set of binaries provided for the client is made of the following files:

\begin{itemize}
\item \textit{myproxy-init}
\item \textit{myproxy-info}
\item \textit{myproxy-destroy}
\item \textit{myproxy-get-delegation}
\item \textit{myproxy-change-pass-phrase}
\end{itemize}


The {\bf myproxy-init} command allows you to create and send a delegated proxy to a MyProxy server for 
later retrieval; in order to launch it you have to assure you're able to execute the grid-proxy-init 
or voms-proxy-init command (i.e. the binary is visible from your \$PATH environment and the required 
cert files are either stored in the common path or specified with the X509 variables). You can use the 
command as follows (you will be asked for your PEM passhprase):

\smallskip
{\scriptsize{\verb!myproxy-init -s <host name> -t <hours> -d -n!}}
\smallskip

The myproxy-init command stores a user proxy in the repository specified by $<$host name$>$ (the -s option). 
Default lifetime of proxies retrieved from the repository will be set to $<$hours$>$  (see -t) and no 
password authorization is permitted when fetching the proxy from the repository (the  -n option). 
The proxy is stored under the same user-name as is your subject in your certificate (-d).

\medskip
The {\bf myproxy-info} command returns the remaining lifetime of the proxy in the repository along 
with subject name of the proxy owner (in our case it will be the same as in your proxy certificate). 
So if you want to get information about the stored proxies you can issue:

\smallskip
{\scriptsize{\verb!myproxy-info -s <host name> -d!}}
\smallskip

where -s and -d options have already been explained in the myproy-init command.

\medskip
The {\bf myproxy-destroy} command simply destroys any existing proxy stored in the myproxy server. 
You can use it as follows:

\smallskip
{\scriptsize{\verb!myproxy-destroy  -s <host name> -d!}}
\smallskip

where -s and -d options have already been explained in the myproy-init command

\medskip
The {\bf myproxy-get-delegation} command is indeed used to retrieve information about the proxies 
stored in the myproxy server. You can use it as follows:

\smallskip
{\scriptsize{\verb!myproxy-get-delegation -s <host name> -d -t <hours> -o <output file> -a <user proxy>!}}
\smallskip

You should end up with a retrieved proxy in $<$output file$>$, which is valid for
$<$hours$>$ hours.

It is worth noting that the environment variable MYPROXY\-\_SERVER can be set to tell to all these 
programs the hostname where the myproxy server is running.

\medskip
{\bf myproxy-change-pass-phrase}  

The  myproxy-change-pass-phrase  command  changes  the passphrase under which a credential is protected 
in the MyProxy repository.  The command first prompts for  the  current  passphrase  for the credential, 
then prompts twice for the new passphrase.  Only the  credential  owner  can change  a  credential's 
passphrase. 
The user must have a valid proxy credential as generated by grid-proxy-init or voms-proxy-init or retrieved 
by myproxy-get-delegation when running this command.
   
%}
