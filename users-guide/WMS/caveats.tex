%\section{Known Problems and Caveats}

\subsection{Submission Failures Analysis}

Analysis of failed job's state can be carried out through the check of the consistency and completeness of the 
job related events returned by the \textit{glite-job-logging-info} command. A further verification if needed, should be 
then performed on the retrieved output files produced by the jobs (if any) and through the inspection of the log 
files ad debugging information traced by the various system components.
As explained in section~\ref{cli} to get the logging information about a job you need to issue the following command:

\smallskip
{\scriptsize{\verb!glite-job-logging-info <job Identifier>!}}
\smallskip


Since the output of the command could be copious we advice usage of the -output option too to redirect it 
to a given file:

\smallskip
{\scriptsize{\verb!glite-job-logging-info --output <my_file> <job Identifier>!}}
\smallskip

Augmenting the verbosity allows then to get more detailed information (the job descriptions at the various stages are 
also included):

\smallskip
{\scriptsize{\verb!glite-job-logging-info -v 3 --output <my_file> <job Identifier>!}}
\smallskip

Before using the \textit{glite-job-logging-info} command, it is in some cases useful a check to the \textit{glite-job-status} 
output that can contain information about the cause of a job failure (e.g. in the "Status Reason" field). 
As explained in section~\ref{cli} to get the status information about a job you need to issue the following command:

\smallskip
{\scriptsize{\verb!glite-job-status -v 2 <job Identifier>!}}
\smallskip

As said at the beginning of this section another way for analysing submission failures is to inspect the standard output 
and error of the job generated on the Worker Node and retrieved on the WMS-UI machine through the \textit{glite-job-output} 
command.  A typical example of errors that can be detected in this way is when the users submits a script that in turn 
tries to start another script or an executable. If e.g. the submitted scripts is like:


\smallskip
\begin{verbatim}

#!/bin/sh
# Use the coincidence file to compare the measurements
curdir=`pwd`
${curdir}/lecture_new_gome_V2_sel1_PT_10
idl appli.pro

\end{verbatim}
\smallskip


Upon job abortion, the error message received through the OutputSandbox retrieval is:

\smallskip
{\scriptsize{\verb!./demo_june: /home/eo004/3042/lecture_new_gome_V2_sel1_PT_10: Permission denied!}}
\smallskip

The reason for this error is that globus-url-copy (used for the InputSanbox files staging) in general doesn't preserve 
the \emph{x} flag so the script specified as Executable in the JDL (on which chmod +x is done automatically by the 
JobWrapper), should perform a chmod +x for all the executable files (\textit{lecture\_new\_gome\_V2\_sel1\_PT\_10} in 
this example) transferred within the InputSandbox of the job.


Hereafter is reported a brief description of the job events that can help when investigating a failure as suggested above.


\subsubsection{Job Event Types}

Hereafter is reported the list of job event types that could be returned to the user by the 
\textit{glite-job-logging-info} command. They are organized in several categories:

\smallskip

Events concerning a job transfer between components:
\begin{itemize}
 	\item \textbf{JobTransfer}	A component generates this event when it tries to transfer a job to some 
other component via network interface (protocol). This event contains the identification of the receiver and 
possibly the job description expressed in the language accepted by the receiver. The result of the transfer, i.e. 
success or failure, as seen by the sender is also included.
 	\item \textbf{JobAccepted}	A component generates this event when it receives a job from another WMS 
component. This event contains also the locally assigned job identifier.
 	\item \textbf{JobRefused}	Receiving component could not accept the job, the reason being a part of the event.
 	\item \textbf{JobEnqueue}	The job is inserted into a queue, e.g., the queue holding the job after it is 
received by Network Server and before it is processed by Workload Manager.
 	\item \textbf{JobDequeue}	The job is removed from queue.
 	\item \textbf{HelperCall}	Helper component is called during the job processing. The type-specific data include 
the name of called Helper, whether the logging component is called or calling one, and optionally parameters passed to the H
elper.
 	\item \textbf{HelperReturn}	Call to Helper returned.
 	
\end{itemize}
\smallskip

Events concerning a job state change during processing within a component:

\begin{itemize}
 	\item \textbf{JobAbort}	The job processing is stopped by WMS due to error condition, the event contains the reason 
for abort.
 	\item \textbf{JobRun}		The job is started on a CE.
 	\item \textbf{JobDone}	Job has exited, has been successfully canceled or is considered to be in terminal state 
by Condor-C.
 	\item \textbf{JobResub}	The result of resubmission decision after the job has failed.
 	\item \textbf{JobCleared}	The user has successfully retrieved the job results, e.g. the output files specified 
in the output sandbox, or the job results has been deleted due to time limit.
 	\item \textbf{JobCancel}	Cancel operation has been attempted on the job.
 	\item \textbf{JobPurge}	The job was purged from bookkeeping server's database. This event is stored only in a 
logging server..

\end{itemize}
\smallskip

Events associated with the Workload Manager or Helper modules:

\begin{itemize}
 	\item \textbf{JobMatch}	An appropriate match between a job and a Computing Element has been found. The event 
contains the identifier of the selected CE.
 	\item \textbf{JobPending}	A match between a job and a suitable Computing Element was not found, so the 
job is kept pending by the WM. The event contains the reason why no match was found.

\end{itemize}
\smallskip

Events used to store special information in logging and bookkeeping services:

\begin{itemize}
 	\item \textbf{JobRegister}	Logged by job creator (User Interface) in order to register the job with 
bookkeeping server.
 	\item \textbf{JobChkpt}	An application-specific checkpoint was created (logged by checkpointing API). Checkpoint 
tag and Class-Ad strings should be included.
 	\item \textbf{JobListener}	Used by WMS-UI to store listener network port information for interactive jobs. 
Listener port number, hostname and service name (multiple ports can be advertised) are included.
 	\item \textbf{JobCurJdl}	This optional event can be used to report Class-Ad describing the current state of 
job processing (output from Helper modules). More details on job event types can be found in ~\cite{lb}.
\end{itemize}

\medskip
\subsection{Job Resubmission and RetryCount}

It is important to note that there are particular cases, as for example temporary network outages in the proximity of 
the CE, that can make the WMS "think" the job has failed (no way currently to distinguish such situations by means of 
error reporting from the underlying components) and hence trigger job resubmission whilst the job is running on the WN. 
This can cause having one or even more copies of the same job running on different CEs since until the network is down 
the WMS is not able to kill the original job.
Due to above mentioned possibility (although should occur very rarely) it is advisable for jobs performing sensitive 
operations (e.g. committing data into a DB) to disable the WMS re-submission feature. This can be easily done on a per-job 
basis setting to 0 the value of the RetryCount attribute in the job description and on a "per-session" basis setting to 0 
the value of the RetryCount parameter in the WMS-UI configuration.


