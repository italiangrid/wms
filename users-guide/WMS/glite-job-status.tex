% PLEASE DO NOT MODIFY THIS FILE! It was generated by raskman version: 1.1.0
\subsubsection{glite-job-status}
\label{glite-job-status}

\medskip
\textbf{glite-job-status}
\smallskip


\medskip
\textbf{SYNOPSIS}
\smallskip

\textbf{glite-job-status [options] $<$jobId$>$}
{\begin{verbatim}

options:
	--version
	--help
	--config, -c <configfile>
	--debug
	--logfile <filepath>
	--noint
	--input, -i <filepath>
	--output, -o <filepath>
	--all
	--config-vo <configfile>
	--verbosity, -v <level>
	--from <[MM:DD:]hh:mm[:[CC]YY]>
	--to <[MM:DD:]hh:mm[:[CC]YY]>
	--user-tag <<tag name>=<tag value>>
	--status, -s <<status code>>
	--exclude, -e <<status code>>
	--nonodes
\end{verbatim}

\medskip
\textbf{DESCRIPTION}
\smallskip


This command prints the status of a job previously submitted using glite-job-submit.
The job status request is sent to the LB that provides the requested information.
This can be done during the whole job life.
glite-job-status can monitor one or more jobs: the jobs to be checked are identified by one or more job identifiers (jobIds returned by glite-job-submit) provided as arguments to the command and separated by a blank space.

\medskip
\textbf{OPTIONS}
\smallskip

\textbf{--version}

displays UI version.

\textbf{--help}

displays command usage

\textbf{--config}, \textbf{-c} <configfile>

if the command is launched with this option, the configuration file pointed by configfile is used. This option is meaningless when used together with "--vo" option

\textbf{--debug}

When this option is specified, debugging information is displayed on the standard output and written into the log file, whose location is eventually printed on screen.
The default UI logfile location is:
glite-wms-job-<command\_name>\_<uid>\_<pid>\_<time>.log  located under the /var/tmp directory
please notice that this path can be overriden with the '--logfile' option

\textbf{--logfile} <filepath>

when this option is specified, all information is written into the specified file pointed by filepath.
This option will override the default location of the logfile:
glite-wms-job-<command\_name>\_<uid>\_<pid>\_<time>.log  located under the /var/tmp directory

\textbf{--noint}

if this option is specified, every interactive question to the user is skipped and the operation is continued (when possible)

\textbf{--input}, \textbf{-i} <filepath>

Allow the user to select the JobId(s) from an input file located in filepath.
The list of jobIds contained in the file is displayed and the user is prompted for a choice. Single jobs can be selected specifying the numbers associated to the job identifiers separated by commas. E.g. selects the first,the third and the fifth jobId in the list.
Ranges can also be selected specifying ends separated by a dash. E.g. selects jobIds in the list from third position (included) to sixth position (included). It is worth mentioning that it is possible to select at the same time ranges and single jobs. E.g. selects the first job id in the list, the ids from the third to the fifth (ends included) and finally the eighth one.
When specified toghether with '--noint', all available JobId are selected.
This option cannot be used when one or more jobIds have been specified as extra command argument

\textbf{--output}, \textbf{-o} <filepath>

writes the results of the operation in the file specified by filepath instead of the standard output. filepath can be either a simple name or an absolute path (on the submitting machine). In the former case the file filepath is created in the current working directory.

\textbf{--all}

displays status information about all job owned by the user submitting the command. This option can't be used
either if one or more jobIds have been specified or if the --input option has been specified. All LBs
listed in the vo-specific UI configuration file \$GLITE\_WMS\_LOCATION/etc/<vo\_name>/glite\_wmsui.conf are contacted to
fulfil this request.

\textbf{--config-vo} <configfile>

if the command is launched with this option, the VO-specific configuration file pointed by configfile is used. This option is meaningless when used together with "--vo" option

\textbf{--verbosity}, \textbf{-v} <level>

sets the detail level of information about the job displayed to the user. Possible values for verb\_level are 0 (only JobId and status/event displayed),1 (timestamp and source information added), 2 (all information but jdls displayed), 3 (complete information containing all Jdl strings)

\textbf{--from} <[MM:DD:]hh:mm[:[CC]YY]>

makes the command query LB for jobs that have been submitted (more precisely entered the "Submitted" status) after the specified date/time.
If only hours and minutes are specified then the current day is taken into account. If the year is not specified then the current year is taken into account.

\textbf{--to} <[MM:DD:]hh:mm[:[CC]YY]>

makes the command query LB for jobs that have been submitted (more precisely entered the "Submitted" status) before the specified date/time.
If only hours and minutes are specified then the current day is taken into account.
If the year is not specified then the current year is taken into account.

\textbf{--user-tag} <<tag name>=<tag value>>

makes the command include only jobs that have defined specified usertag name and value

\textbf{--status}, \textbf{-s} <<status code>>

makes the command query LB for jobs that are in the specified status.
The status value can be either an integer or a (case insensitive) string; the following possible values are allowed:
UNDEF (0), SUBMITTED(1), WAITING(2), READY(3), SCHEDULED(4), RUNNING(5), DONE(6), CLEARED(7), ABORTED(8), CANCELLED(9),
UNKNOWN(10), PURGED(11).
This option can be repeated several times, all status conditions will be considered as in a logical OR operation

(i.e.  -s SUBMITTED --status 3  will query all jobs that are either in SUBMITTED or in READY status)

\textbf{--exclude}, \textbf{-e} <<status code>>

makes the command query LB for jobs that are NOT in the specified status.
The status value can be either an integer or a (case insensitive) string; the following possible values are allowed:
UNDEF (0), SUBMITTED(1), WAITING(2), READY(3), SCHEDULED(4), RUNNING(5), DONE(6), CLEARED(7), ABORTED(8), CANCELLED(9),
UNKNOWN(10), PURGED(11).
This option can be repeated several times, all status conditions will be considered as in a logical AND operation

(i.e.  -e SUBMITTED --exclude 3  will query all jobs that are neither in SUBMITTED nor in READY status)

\textbf{--nonodes}

This option will not display any information of (if present) sub jobs of any dag, only requested JobId(s) info will be taken into account


\medskip
\textbf{ENVIRONMENT}
\smallskip


GLITE\_WMSUI\_CONFIG\_VAR:  This variable may be set to specify the path location of the custom default attribute configuration

GLITE\_WMSUI\_CONFIG\_VO: This variable may be set to specify the path location of the VO-specific configuration file

GLITE\_WMS\_LOCATION:  This variable must be set when the Glite WMS installation is not located in the default paths: either /opt/glite or /usr/local

GLITE\_LOCATION: This variable must be set when the Glite installation is not located in the default paths: either /opt/glite or /usr/local


GLOBUS\_LOCATION: This variable must be set when the Globus installation is not located in the default path /opt/globus.
It is taken into account only by submission and get-output commands

GLOBUS\_TCP\_PORT\_RANGE="<val min> <val max>" This variable must be set to define a range of ports to be used for inbound connections in the interactivity context.
It is taken into account only by submission of interactive jobs and attach commands

X509\_CERT\_DIR: This variable may be set to override the default location of the trusted certificates directory, which is normally /etc/grid-security/certificates.

X509\_USER\_PROXY: This variable may be set to override the default location of the user proxy credentials, which is normally /tmp/x509up\_u<uid>.

\medskip
\textbf{FILES}
\smallskip


One of the following paths must exist (seeked with the specified order):
- \$GLITE\_WMS\_LOCATION/etc/
- \$GLITE\_LOCATION/etc/
- /opt/glite/etc/
- /usr/local/etc/
- /etc/

and contain the following UI configuration files:
glite\_wmsui\_cmd\_var.conf, glite\_wmsui\_cmd\_err.conf, glite\_wmsui\_cmd\_help.conf, <voName>/glite\_wmsui.conf

- glite\_wmsui\_cmd\_var.conf will contain custom configuration default values
A different configuration file may be specified either by using the --config option or by setting the GLITE\_WMSUI\_CONFIG\_VAR environment variable
here follows a possible example:
[
RetryCount = 3 ;
ErrorStorage= "/tmp" ;
OutputStorage="/tmp";
ListenerStorage = "/tmp" ;
LoggingTimeout = 30 ;
LoggingSyncTimeout = 30 ;
NSLoggerLevel = 0;
DefaultStatusLevel = 1 ;
DefaultLogInfoLevel = 1;
]

- glite\_wmsui\_cmd\_err.conf will contain UI exception mapping between error codes and error messages (no relocation possible)

- glite\_wmsui\_cmd\_help.conf will contain UI long-help information (no relocation possible)

- <voName>/glite\_wmsui.conf  will contain User VO-specific attributes.
A different configuration file may be specified either by using the --config-vo option or by setting the GLITE\_WMSUI\_CONFIG\_VO environment variable
here follows a possible example:
[
LBAddresses = { "tigerman.cnaf.infn.it:9000" };
VirtualOrganisation = "egee";
NSAddresses = { "tigerman.cnaf.infn.it:7772" }
]

Besides those files, a valid proxy must be found inside the following path:
/tmp/x509up\_u<uid> ( use the X509\_USER\_PROXY environment variable to override the default location JDL file)

\medskip
\textbf{AUTHORS}
\smallskip


Alessandro Maraschini (egee@datamat.it)

