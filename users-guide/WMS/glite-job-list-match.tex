% PLEASE DO NOT MODIFY THIS FILE! It was generated by raskman version: 1.1.0
\subsubsection{glite-job-list-match}
\label{glite-job-list-match}

\medskip
\textbf{glite-job-list-match}
\smallskip


\medskip
\textbf{SYNOPSIS}
\smallskip

\textbf{glite-job-list-match [options]  $<$jdl\_file$>$}
{\begin{verbatim}

options:
	--version
	--help
	--config, -c <configfile>
	--debug
	--logfile <filepath>
	--noint
	--output, -o <filepath>
	--verbose
	--rank
	--config-vo <configfile>
	--vo <voname>
\end{verbatim}

\medskip
\textbf{DESCRIPTION}
\smallskip


glite-job-list-match displays the list of identifiers of the resources on which the user is authorized and satisfying the job requirements included in the job description file. The CE identifiers are returned either on the standard output or in a file according to the chosen command options, and are strings univocally identifying the CEs published in the IS.
The returned CEIds are listed in decreasing order of rank, i.e. the one with the best (greater) rank is in the first place and so forth.

\medskip
\textbf{OPTIONS}
\smallskip

\textbf{--version}

displays UI version.

\textbf{--help}

displays command usage

\textbf{--config}, \textbf{-c} <configfile>

if the command is launched with this option, the configuration file pointed by configfile is used. This option is meaningless when used together with "--vo" option

\textbf{--debug}

When this option is specified, debugging information is displayed on the standard output and written into the log file, whose location is eventually printed on screen.
The default UI logfile location is:
glite-wms-job-<command\_name>\_<uid>\_<pid>\_<time>.log  located under the /var/tmp directory
please notice that this path can be overriden with the '--logfile' option

\textbf{--logfile} <filepath>

when this option is specified, all information is written into the specified file pointed by filepath.
This option will override the default location of the logfile:
glite-wms-job-<command\_name>\_<uid>\_<pid>\_<time>.log  located under the /var/tmp directory

\textbf{--noint}

if this option is specified, every interactive question to the user is skipped and the operation is continued (when possible)

\textbf{--output}, \textbf{-o} <filepath>

writes the results of the operation in the file specified by filepath instead of the standard output. filepath can be either a simple name or an absolute path (on the submitting machine). In the former case the file filepath is created in the current working directory.

\textbf{--verbose}

displays on the standard output the job class-ad that is sent to the Network Server generated from the job description file.
This differs from the content of the job description file since the UI adds to it some attributes that cannot be directly inserted by the user
(e.g., defaults for Rank and Requirements if not provided, VirtualOrganisation etc).

\textbf{--rank}

displays the "matching" CEIds toghether with their associated ranking values.

\textbf{--config-vo} <configfile>

if the command is launched with this option, the VO-specific configuration file pointed by configfile is used. This option is meaningless when used together with "--vo" option

\textbf{--vo} <voname>

this option allows the user to specify the name of the Virtual Organisation she/he is currently working for.
If the user proxy contains VOMS extensions then the VO specified through this option is overridden by the default VO contained in the proxy (i.e. this option is only useful when working with non-VOMS proxies).
This option is meaningless when used together with "--config-vo" option


\medskip
\textbf{EXAMPLES}
\smallskip


1) simple request:
glite-job-list-match ./match.jdl

If the operation succeeds, the output will be a list of CEs

2) request for displays CE rank numbers:
glite-job-list-match --rank ./match.jdl

If the operation succeeds, a list of CEs with their rank numbers is displayed on the standard output

3) saves the result in a file:
glite-job-list-match --output match.out ./match.jdl

If the operation succeeds,a list of CEs is saved in the file match.out in the current working directory


\medskip
\textbf{ENVIRONMENT}
\smallskip


GLITE\_WMSUI\_CONFIG\_VAR:  This variable may be set to specify the path location of the custom default attribute configuration

GLITE\_WMSUI\_CONFIG\_VO: This variable may be set to specify the path location of the VO-specific configuration file

GLITE\_WMS\_LOCATION:  This variable must be set when the Glite WMS installation is not located in the default paths: either /opt/glite or /usr/local

GLITE\_LOCATION: This variable must be set when the Glite installation is not located in the default paths: either /opt/glite or /usr/local


GLOBUS\_LOCATION: This variable must be set when the Globus installation is not located in the default path /opt/globus.
It is taken into account only by submission and get-output commands

GLOBUS\_TCP\_PORT\_RANGE="<val min> <val max>" This variable must be set to define a range of ports to be used for inbound connections in the interactivity context.
It is taken into account only by submission of interactive jobs and attach commands

X509\_CERT\_DIR: This variable may be set to override the default location of the trusted certificates directory, which is normally /etc/grid-security/certificates.

X509\_USER\_PROXY: This variable may be set to override the default location of the user proxy credentials, which is normally /tmp/x509up\_u<uid>.

\medskip
\textbf{FILES}
\smallskip


One of the following paths must exist (seeked with the specified order):
- \$GLITE\_WMS\_LOCATION/etc/
- \$GLITE\_LOCATION/etc/
- /opt/glite/etc/
- /usr/local/etc/
- /etc/

and contain the following UI configuration files:
glite\_wmsui\_cmd\_var.conf, glite\_wmsui\_cmd\_err.conf, glite\_wmsui\_cmd\_help.conf, <voName>/glite\_wmsui.conf

- glite\_wmsui\_cmd\_var.conf will contain custom configuration default values
A different configuration file may be specified either by using the --config option or by setting the GLITE\_WMSUI\_CONFIG\_VAR environment variable
here follows a possible example:
[
RetryCount = 3 ;
ErrorStorage= "/tmp" ;
OutputStorage="/tmp";
ListenerStorage = "/tmp" ;
LoggingTimeout = 30 ;
LoggingSyncTimeout = 30 ;
NSLoggerLevel = 0;
DefaultStatusLevel = 1 ;
DefaultLogInfoLevel = 1;
]

- glite\_wmsui\_cmd\_err.conf will contain UI exception mapping between error codes and error messages (no relocation possible)

- glite\_wmsui\_cmd\_help.conf will contain UI long-help information (no relocation possible)

- <voName>/glite\_wmsui.conf  will contain User VO-specific attributes.
A different configuration file may be specified either by using the --config-vo option or by setting the GLITE\_WMSUI\_CONFIG\_VO environment variable
here follows a possible example:
[
LBAddresses = { "tigerman.cnaf.infn.it:9000" };
VirtualOrganisation = "egee";
NSAddresses = { "tigerman.cnaf.infn.it:7772" }
]

Besides those files, a valid proxy must be found inside the following path:
/tmp/x509up\_u<uid> ( use the X509\_USER\_PROXY environment variable to override the default location JDL file)

\medskip
\textbf{AUTHORS}
\smallskip


Alessandro Maraschini (egee@datamat.it)

