% PLEASE DO NOT MODIFY THIS FILE! It was generated by raskman version: 1.1.0
\subsubsection{glite-job-output}
\label{glite-job-output}

\medskip
\textbf{glite-job-output}
\smallskip


\medskip
\textbf{SYNOPSIS}
\smallskip

\textbf{glite-job-output [options]  $<$job Id(s)$>$}
{\begin{verbatim}

options:
	--version
	--help
	--config, -c <configfile>
	--debug
	--logfile <filepath>
	--noint
	--input, -i <filepath>
	--dir <directorypath>
\end{verbatim}

\medskip
\textbf{DESCRIPTION}
\smallskip


The glite-job-output command can be used to retrieve the output files of a job that has been submitted through the glite-job-submit command with a job description file including the OutputSandbox attribute.
After the submission, when the job has terminated its execution, the user can download the files generated by the job and temporarily stored on the RB machine as specified by the OutputSandbox attribute, issuing the glite-job-output with as input the jobId returned by the glite-job-submit.

\medskip
\textbf{OPTIONS}
\smallskip

\textbf{--version}

displays UI version.

\textbf{--help}

displays command usage

\textbf{--config}, \textbf{-c} <configfile>

if the command is launched with this option, the configuration file pointed by configfile is used. This option is meaningless when used together with "--vo" option

\textbf{--debug}

When this option is specified, debugging information is displayed on the standard output and written into the log file, whose location is eventually printed on screen.
The default UI logfile location is:
glite-wms-job-<command\_name>\_<uid>\_<pid>\_<time>.log  located under the /var/tmp directory
please notice that this path can be overriden with the '--logfile' option

\textbf{--logfile} <filepath>

when this option is specified, all information is written into the specified file pointed by filepath.
This option will override the default location of the logfile:
glite-wms-job-<command\_name>\_<uid>\_<pid>\_<time>.log  located under the /var/tmp directory

\textbf{--noint}

if this option is specified, every interactive question to the user is skipped and the operation is continued (when possible)

\textbf{--input}, \textbf{-i} <filepath>

Allow the user to select the JobId(s) from an input file located in filepath.
The list of jobIds contained in the file is displayed and the user is prompted for a choice. Single jobs can be selected specifying the numbers associated to the job identifiers separated by commas. E.g. selects the first,the third and the fifth jobId in the list.
Ranges can also be selected specifying ends separated by a dash. E.g. selects jobIds in the list from third position (included) to sixth position (included). It is worth mentioning that it is possible to select at the same time ranges and single jobs. E.g. selects the first job id in the list, the ids from the third to the fifth (ends included) and finally the eighth one.
When specified toghether with '--noint', all available JobId are selected.
This option cannot be used when one or more jobIds have been specified as extra command argument

\textbf{--dir} <directorypath>

if this option is specified, the retrieved files (previously listed by the user through the OutputSandbox attribute of the job description file) are stored in the location indicated by directorypath.


\medskip
\textbf{ENVIRONMENT}
\smallskip


GLITE\_WMSUI\_CONFIG\_VAR:  This variable may be set to specify the path location of the custom default attribute configuration

GLITE\_WMSUI\_CONFIG\_VO: This variable may be set to specify the path location of the VO-specific configuration file

GLITE\_WMS\_LOCATION:  This variable must be set when the Glite WMS installation is not located in the default paths: either /opt/glite or /usr/local

GLITE\_LOCATION: This variable must be set when the Glite installation is not located in the default paths: either /opt/glite or /usr/local


GLOBUS\_LOCATION: This variable must be set when the Globus installation is not located in the default path /opt/globus.
It is taken into account only by submission and get-output commands

GLOBUS\_TCP\_PORT\_RANGE="<val min> <val max>" This variable must be set to define a range of ports to be used for inbound connections in the interactivity context.
It is taken into account only by submission of interactive jobs and attach commands

X509\_CERT\_DIR: This variable may be set to override the default location of the trusted certificates directory, which is normally /etc/grid-security/certificates.

X509\_USER\_PROXY: This variable may be set to override the default location of the user proxy credentials, which is normally /tmp/x509up\_u<uid>.

\medskip
\textbf{FILES}
\smallskip


One of the following paths must exist (seeked with the specified order):
- \$GLITE\_WMS\_LOCATION/etc/
- \$GLITE\_LOCATION/etc/
- /opt/glite/etc/
- /usr/local/etc/
- /etc/

and contain the following UI configuration files:
glite\_wmsui\_cmd\_var.conf, glite\_wmsui\_cmd\_err.conf, glite\_wmsui\_cmd\_help.conf, <voName>/glite\_wmsui.conf

- glite\_wmsui\_cmd\_var.conf will contain custom configuration default values
A different configuration file may be specified either by using the --config option or by setting the GLITE\_WMSUI\_CONFIG\_VAR environment variable
here follows a possible example:
[
RetryCount = 3 ;
ErrorStorage= "/tmp" ;
OutputStorage="/tmp";
ListenerStorage = "/tmp" ;
LoggingTimeout = 30 ;
LoggingSyncTimeout = 30 ;
NSLoggerLevel = 0;
DefaultStatusLevel = 1 ;
DefaultLogInfoLevel = 1;
]

- glite\_wmsui\_cmd\_err.conf will contain UI exception mapping between error codes and error messages (no relocation possible)

- glite\_wmsui\_cmd\_help.conf will contain UI long-help information (no relocation possible)

- <voName>/glite\_wmsui.conf  will contain User VO-specific attributes.
A different configuration file may be specified either by using the --config-vo option or by setting the GLITE\_WMSUI\_CONFIG\_VO environment variable
here follows a possible example:
[
LBAddresses = { "tigerman.cnaf.infn.it:9000" };
VirtualOrganisation = "egee";
NSAddresses = { "tigerman.cnaf.infn.it:7772" }
]

Besides those files, a valid proxy must be found inside the following path:
/tmp/x509up\_u<uid> ( use the X509\_USER\_PROXY environment variable to override the default location JDL file)

\medskip
\textbf{AUTHORS}
\smallskip


Alessandro Maraschini (egee@datamat.it)

