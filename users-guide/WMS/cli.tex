%\subsection{Commandline Interfaces}

In this section we describe syntax and behavior of the commands made available by the WMS-UI to allow job/DAG 
submission, monitoring and control. In the commands synopsis the mandatory arguments are showed between 
angle brackets $<$arg$>$) whilst the optional ones between square brackets ([arg]).

Commands for accessing the WMS through the WMProxy service are described in document ~\cite{WMPROXY}.
Note that usage of the WMProxy submission and control client commands is strongly recommended as they 
provide full support for all new functionality and enhancements of the WMS. 

Before going to the single commands let's have a look at how the WMS-UI can be configured.


\medskip
\subsubsection{Commands Configuration}
\label{config}

\textbf{VO-Specific}

Configuration of the WMS User Interface VO-specific parameters is accomplished through the file:

\smallskip
\begin{verbatim}
$GLITE_LOCATION/etc/<vo name>/glite_wmsui.conf 
\end{verbatim}
\smallskip


i.e. there is one directory and file for each supported VO. 

The common WMS-UI configuration rpm (\textit{glite-wms-ui-configuration}) installs the following example file:

\smallskip
\begin{verbatim}
$GLITE_LOCATION/etc/vo_template/glite_wmsui.conf
\end{verbatim}
\smallskip

If the configuration for your VO is not present on the WMS-UI machine you must create in \$GLITE\_LOCATION/etc a 
directory, named as the VO (lower-case), copy in it the above mentioned template file and update it opportunely.

The \emph{glite\_wmsui.conf} file is a classad containing the following fields:

\smallskip

\begin{itemize}
 \item \textbf{VirtualOrganisation} this is a string representing the name of the virtual organisation the file refers to.
   It should match with the name of the directory containing the file (i.e. the VO). This parameter is 
   mandatory. 
 \item \textbf{NSAddresses} this is a list of strings representing the addresses ($<$hostname$>$:$<$port$>$) of the Network 
   Servers available for the given VO. Job submission is performed towards the NS picked-up randomly from 
   the list and in case of failure it is retried on each other listed NS until succes or the end of the 
   list is reached. This parameter is mandatory.
 \item \textbf{LBAddresses} this is a list of strings or a list of lists of strings representing the address or list 
   of addresses ($<$hostname$>$:$<$port$>$) of the LB servers available for the given VO for the corresponding NS. 
   I.e. the first list of LB addresses correspond to the first NS in the NSAddresses list, the second list 
   of LB addresses correspond to the second NS in the NSAddresses list and so on.
   When job submission is performed, the WMS-UI after having chosen the NS, choses randomly one LB server within
   the corresponding list and uses it for generating the job identifier so that all information related with 
   that job will be managed by the chosen LB server. This allows distributing load on several LB servers.  
   This parameter is mandatory.
 \item \textbf{HLRLocation} this is a string representing the address ($<$hostname$>$:$<$port$>$:$<$X509contact string$>$)  of the 
   HLR for the given VO. HLR is the service responsible for managing the economic transactions and the 
   accounts of user and resources. This parameter is not mandatory. It is not present in the file by default.
   If present, it makes the WMS-UI automatically add to the job description the HLRLocation JDL attribute 
   (if not specified by the user) and this enables accounting.
 \item \textbf{MyProxyServer} this is a string representing the MYProxy server address ($<$hostname$>$) for the given VO. 
   This parameter is not mandatory. It is not present in the file by default. If present, it makes the WMS-UI 
   automatically add to the job description the MyProxyServer JDL attribute (if not specified by the user) 
   and this enables proxy renewal. If the myproxy client package is installed on the WMS-UI node, then this 
   parameter should be set equal to the MYPROXY\_SERVER environment variable.
 \item \textbf{LoggingDestination} this is a string defining the address ($<$host$>$:$[<$port$>]$;) of the LB logging service 
   (glite-lb-locallogger logging daemon ) to be targeted when logging events. The WMS-UI first checks the 
   environment for the EDG\_WL\_LOG\_DESTINATION variable and only if this is not set, the value of the 
   LoggingDestination parameter is taken into account. Otherwise the job related events are logged to the 
   LB logging service running on the WMS node.

\end{itemize}
\smallskip

Here below is provided an example of configuration file for the "atlas" Virtual Organisation. 
This implies that the file path has to be \textit{\$GLITE\_LOCATION/etc/atlas/glite\_wmsui.conf}. 

\smallskip
\begin{verbatim}

[
 VirtualOrganisation = "atlas";
 NSAddresses = {
   "ibm139.cnaf.infn.it:7772",
   "gundam.cnaf.infn.it:7772"
   };
 LBAddresses = {
   {"ibm139.cnaf.infn.it:9000"},
   {"gundam.cnaf.infn.it:9000", "neo.datamat.it:9000", "grid003.ct.infn.it:9876"}
  };
 HLRLocation = "lilith.to.infn.it:56568:/C=IT/O=INFN/OU=Personal Certificate/L=Torino/CN=Andrea 
                Guarise/Email=A.Guarise@to.infn.it";
 MyProxyServer = "skurut.cesnet.cz";
 LoggingDestination = "localhost:9002";  // local instance of LB logging service
]

\end{verbatim}
\medskip
\medskip


\textbf{Generic}


Configuration of the WMS User Interface generic parameters is accomplished through the file:

\smallskip
\begin{verbatim}
$GLITE_LOCATION/etc/glite_wmsui_cmd_var.conf
\end{verbatim}
\smallskip


Update the content of the latter file according to your needs.

The \textit{glite\_wmsui\_cmd\_var.conf} file is a classad containing the following fields:


\smallskip

\begin{itemize}
 \item \textbf{requirements} this is an expression representing the default value for the requirements expression 
   in the JDL job description. This parameter is mandatory. The value of this parameter is assigned by 
   the WMS-UI to the requirements attribute in the JDL if not specified by the user. If the user has instead 
   provided an expression for the requirements attribute in the JDL, the one specified in the configuration 
   file is added (in AND) to the existing one. 
   E.g. if in the glite\_wmsui\_cmd\_var.conf configuration file there is: 

   \begin{scriptsize}   
   \textit{requirements = other.GlueCEStateStatus == "Production" ;} 
   \end{scriptsize}

   and in the JDL file the user has specified: 

   \begin{scriptsize}   
   \textit{requirements = other.GlueCEInfoLRMSType == "PBS";} 
   \end{scriptsize}

   then the job description that is passed to the WMS contains 

   \begin{scriptsize}   
   \textit{requirements = (other.GlueCEInfoLRMSType == "PBS") \&\& (other.GlueCEStateStatus == "Production");} 
   \end{scriptsize}
      
   Obviously the setting TRUE for the requirements in the configuration file does not have any impact on the evaluation 
   of job requirements as it would result in: 

   \begin{scriptsize}   
   \textit{requirements = (other.GlueCEInfoLRMSType == "PBS") \&\& TRUE ;}
   \end{scriptsize}

 \item \textbf{rank} this is an expression representing the default value for the rank expression in the JDL job 
   description. The value of this parameter is assigned by the WMS-UI to the rank attribute in the JDL if not 
   specified by the user. This parameter is mandatory.   
 \item \textbf{RetryCount} this is an integer representing the default value for the number of submission retries for 
   a job upon failure due to some grid component (i.e. not to the job itself). The value of this parameter 
   is assigned by the WMS-UI to the RetryCount attribute in the JDL if not specified by the user.   
 \item \textbf{DefaultVo} this is a string representing the name of the virtual organisation to be taken as the user s 
   VO (VirtualOrganisation attribute in he JDL) if not specified by the user neither in the credentials 
   VOMS extension, nor directly in the job description nor through the --vo option. This attribute can be 
   either set to  unspecified  or not included at all in the file to mean that no default is set for the VO. 
 \item \textbf{ErrorStorage} this is a string representing the path of the directory where the WMS-UI creates log files. 
   This directory is not created by the WMS-UI, so It has to be an already existing directory. Default for this 
   parameter is /tmp.   
 \item \textbf{OutputStorage} this is a string defining the path of the directory where the job OutputSandbox files are 
   stored if not specified by the user through commands options. This directory is not created by the WMS-UI, 
   so It has to be an already existing directory. Default for this parameter is /tmp. 
 \item \textbf{ListenerStorage} this is a string defining the path of the directory where are created the pipes where 
   the glite\_wms\_console\_shadow process saves the job standard streams for interactive jobs. Default for 
   this parameter is /tmp.   
 \item \textbf{LoggingTimeout} this is an integer representing the timeout in seconds for asynchronous logging function 
   called by the WMS-UI when logging events to the LB. Recommended value for WMS-UI that are non-local to the 
   logging service (glite-lb-logd logging daemon) is not less than 30 seconds. 
 \item \textbf{LoggingSyncTimeout} this is an integer representing the timeout in seconds for synchronous logging function
   called by the WMS-UI when logging events to the LB. Recommended value is not less than 30 seconds.   
 \item \textbf{DefaulStatusLevel} this is an integer defining the default level of verbosity for the glite-job-status 
   command. Possible values are 0,1,2 and 3. 1 is the default.
 \item \textbf{DefaultLogInfoLevel} this is an integer defining the default level of verbosity for the glite-job-logging-info
   command. Possible values are 0,1,2 and 3. 1 is the default. 
   Default for this parameter is 0.   
 \item \textbf{NSLoggerLevel} this is an integer defining the quantity of information logged by the NS client. Possible 
   values range from 0 to 6. 0 is the defaults and means that no information is logged. Default for this 
   parameter is 0. 

\end{itemize}
\smallskip


Hereafter is provided an example of the \textit{\$GLITE\_LOCATION/etc/glite\_wmsui\_cmd\_var.conf} configuration file: 


\smallskip
\begin{verbatim}

[ 
 requirements = other.GlueCEStateStatus == "Production" ; 
 rank = - other.GlueCEStateEstimatedResponseTime ; 
 RetryCount = 3 ; 
 ErrorStorage= "/var/tmp" ; 
 OutputStorage="/tmp/jobOutput"; 
 ListenerStorage = "/tmp" 
 LoggingTimeout = 30 ; 
 LoggingSyncTimeout = 45 ;  
 DefaultStatusLevel = 1 ; 
 DefaultLogInfoLevel = 0; 
 NSLoggerLevel = 2; 
 DefaultVo = "EGEE"; 
] 

\end{verbatim}
\smallskip

The files: 

\smallskip
\begin{verbatim}
$GLITE_LOCATION/etc/glite_wmsui_cmd_err.conf
\end{verbatim}
\smallskip

and 

\smallskip
\begin{verbatim}
$GLITE_LOCATION/etc/glite_wmsui_cmd_help.conf 
\end{verbatim}
\smallskip

contain respectively the error codes and error messages returned by the WMS-UI and the text describing the 
commands usage.

\newpage
\subsubsection{Common behaviours}
\label{commonbeh}

As mentioned in the previous section~\ref{quickstart}, 
\textit{\$GLITE\_LOCATION/etc} is the WMS-UI configuration area: it includes the file specifying 
the mapping between error codes and error messages (glite\_wmsui\_cmd\_err.conf), 
the file containing the detailed description of each command (glite\_wmsui\_cmd\_help.conf) and the 
actual configuration files: glite\_wmsui\_cmd\_var.conf and $<$VO name$>$/glite\_wmsui.conf). 
The latter files are the only ones that could need to be edited and tailored according to the user/platform 
characteristics and needs. 
The \emph{glite\_wmsui\_cmd\_var.conf } file contains the following information that are read by and have 
influence on commands behaviour: 

\begin{itemize}
\item default location of the local storage areas for the Output sandbox files,
\item default location for the WMS-UI log files,
\item default values for the JDL mandatory attributes,
\item default values for timeouts when logging events to the LB,
\item default logging destination,
\item user's default VO,
\item default level of information displayed by the monitoring commands
\end{itemize}

Inside \textit{\$GLITE\_LOCATION/etc} there is instead a directory for each supported Virtual Organisation and 
named as the VO lower case e.g. for atlas we will have \$GLITE\_LOCATION/etc/atlas/) that contains a vo-specific 
configuration file glite\_wmsui.conf specifying the list of Network Servers and LBs accessible for the given VO.

When started, WMS-UI commands search for the configuration files in the following locations, in order of 
precedence:
 
\begin{itemize}
 \item \$GLITE\_WMS\_LOCATION/etc,
 \item \$GLITE\_LOCATION/etc,  
 \item /opt/glite/etc,
 \item /usr/local/etc,
 \item /etc
\end{itemize}

If none of the locations contains needed files an error is returned to the user.

Since several users on the same machine can use a single installation of the WMS-UI, people concurrently issuing 
WMS-UI commands share the same configuration files. Anyway for users (or groups of users) having particular needs 
it is possible to use "customised" WMS-UI configuration files through the --config and -config-vo options supported 
by each WMS-UI command.

Indeed every command launched specifying \emph{--config file\_path} reads its configuration settings in the file 
pointed by "file\_path" instead of the default configuration file. The same happens for the vo-specific 
configuration file if the command is started using specifying  \emph{-config-vo vo\_file\_path}. 
Hence the user only needs to create such file according to her needs and to use the appropriate options to work 
under "private" settings.

Moreover if the user wants to make this change in some way permanent avoiding the use for each issued command 
of the --config option, she can set the environment variable GLITE\_WMSUI\_CONFIG\_VAR to point to the non-standard 
path of the configuration file. Indeed if that variable is set commands will read settings from file 
"\$GLITE\_WMSUI\_CONFIG\_VAR". Anyway the --config option takes precedence on all other settings.

Exactly the same applies for the GLITE\_WMSUI\_CONFIG\_VO environment variable and the --config-vo option.

It is important to note that since the job identifiers implicitly holds the information about the LB that is 
managing the corresponding job, all the commands taking the job Id as input parameter do not take into account 
the LB addresses listed in the configuration file to perform the requested operation also if the -config-vo option 
has been specified.

Hereafter are listed the options that are common to all WMS-UI commands: 

{\begin{verbatim}
--config file_path
--config-vo file_path
--noint
--debug
--logfile file_path
--version
--help
\end{verbatim} 

The \verb!--noint! option skips all interactive questions to the user and goes ahead in the command execution. 
All warning messages and errors (if any) are written to the file 

\textbf{$<$command\_name$>$\-\_$<$UID$>$\-\_$<$PID$>$\-\_$<$date\_time$>$.log} 

in the location specified in the configuration file instead of the standard output. 
It is important to note that when \verb!--noint! is specified some checks on "dangerous actions" are skipped. 
For example if jobs cancellation is requested with this option, this action will be performed without requiring 
any confirmation to the user. The same applies if the command output will overwrite an existing file, so it is 
recommended to use the \verb!--noint! option in a safe context.

\medskip

The \verb!--debug! option is mainly thought for testing and debugging purposes; indeed it makes the commands 
print additional information while running. Every time an external API function call is encountered during the 
command execution, values of parameters passed to the API are printed to the user. The info messages are displayed 
on the standard output and are also written together with possible errors and warnings, to 

\textit{$<$command\_name$>$\-\_$<$UID$>$\-\_$<$PID$>$\-\_$<$date\-\_time$>$.log}.

\medskip

If \verb!--noint! option is specified together with \verb!--debug! option the debug message will not be printed on 
standard output.

\medskip

The \verb!--logfile! $<$file\_path$>$ option allows re-location of the commands log files in the location pointed 
by file\_path.

\medskip

The \verb!--version! and \verb!--help! options respectively make the commands display the WMS-UI current version and 
the command usage.

\medskip

Two further options that are common to almost all commands are \verb!--input! and \verb!--output!. The latter one 
makes the commands redirect the outcome to the file specified as option argument whilst the former reads a list of 
input items from the file given as option argument. The only exception is the glite\--job\--list\--match command 
that does not have the \verb!--input! option.

\medskip
\textbf{--input option}
\medskip
\smallskip
for all commands, the file given as argument to the \verb!--input! option shall contain a list of job identifiers 
in the following format: one \textit{jobId} for each line, comments beginning with a "\#" or a "*" character.  
If the input file contains only one \textit{jobId} (see the description of glite-job-submit command later in this 
document for details about \textit{jobId} format), then the request is directly submitted taking the 
\textit{jobId} as input, otherwise a menu is displayed to the user listing all the contained items, 
i.e. something like:

\smallskip
\begin{scriptsize}
\begin{verbatim}
---------------------------------------------------------------
1 : https://ibm139.cnaf.infn.it:9000/ZU9yOC7AP7AOEhMAHirG3w
2 : https://ibm139.cnaf.infn.it:9000/ZU9yOC767gJOEhMAHirG3w
3 : https://ibm135.cnaf.infn.it:9000/ZU9yOC7AP7A55TREAHirG3w
4 : https://grid012f.cnaf.infn.it:7846/ZUHY6707AP7AOEhMAHirG3w
5 : https://grid012f.cnaf.infn.it:9000/Cde341P7AOEhMAHirG3w
6 : https://ibm139.cnaf.infn.it:9000/BgT8T6H\_L92FsKq3OeTWOw
7 : https://ibm139.cnaf.infn.it:9000/lYlPBQez7fiXx9qq7BEdyw
8 : https://ibm139.cnaf.infn.it:9000/_f0Bm\_s6UdFPZIEjSglipg
a : all
q : quit
---------------------------------------------------------------
Choose one or more jobId(s) in the list - [1-10]all:
\end{verbatim} 
\end{scriptsize}
\smallskip

The user can choose one or more jobs from the list entering the corresponding numbers. Single jobs can be selected 
specifying the numbers associated to the job identifiers separated by commas. Ranges can also be selected 
specifying ends separated by a dash and it is worth mentioning that it is possible to select at the same time 
ranges and single jobs. E.g.:

\begin{itemize}
\item [2:] 	makes the command take the second listed \textit{jobId} as input
\item [1,4:]	makes the command take the first and the fourth listed \textit{jobId}s as input
\item [2-5:]	makes the command take listed \textit{jobId}s from 2 to 5 ends included) as input
\item [1,3-5,8:] selects the first job id in the list, the ids from the third to the fifth ends included) 
and finally the eighth one.
\item [all:]	makes the command take all listed \textit{jobId}s as input
\item [q:]	makes the command quit
\end{itemize}

Default value for the choice is all. 

If the \verb!--input! option is used together with the  \verb!--noint! then all \textit{jobId}s contained in 
the input file are taken into account by the command.

There are some commands whose \verb!--input! behaviour differs from the one just described. One of them is 
glite-job-submit: the input file contains in this case CEIds instead of \textit{jobId}s. 
As only one CE at a time can be the target of a submission, the user is allowed to choose one and only one CEId.
Default value for the choice is "1", i.e. the first CEId in the list. 
This is also the choice automatically made by the command when the \verb!--input! option is used together with 
the \verb!--noint! one.

The other commands are \textbf{glite-job-attach} and \textbf{glite-job-get-chkpt} whose \verb!--input! option 
allows to select one (just one) of the \textit{jobId}s contained in the input file.

\newpage

\subsubsection{glite-job-submit}
\label{submit}

Allows the user to submit a job/DAG for execution on remote resources in a grid.

\medskip
SYNOPSIS
\smallskip

\textbf{glite-job-submit  [options]  $<$jdl\_file$>$}
\medskip

{\begin{verbatim}
Options:
   --help
   --version
   --vo            <vo_name>			
   --input, -i     <file_path>
   --resource, -r  <ce_id>
   --chkpt         <file_path>
   --nolisten
   --nogui
   --nomsg
   --lrms          <lrms value>
   --to            [MM:DD:]hh:mm[:[CC]YY]
   --valid, -v     <hours>:<minutes>
   --config, -c    
   --config-vo     <file_path>
   --output, -o    <file_path>
   --noint
   --debug
   --logfile <file_path>
\end{verbatim} 

\medskip
DESCRIPTION 
\smallskip

\textbf{glite-job-submit} is the command for submitting jobs and DAGs to the grid, i.e. allows the user to run her 
application  job at one remote resource. glite-job-submit requires as input a job description file in which job/DAG 
characteristics and requirements are expressed by means of Condor class-ad-like expressions. While it does not 
matter the order of the other arguments, the request description file has to be the last argument of this command.

The job/DAG description file given in input to this command is syntactically checked and default values are 
assigned to some of the not provided mandatory attributes in order to create a meaningful class-ad. 
The resulting \emph{ad} is sent to the NS, which then forwards it to the WM, which via the RB/Matchmaker finds 
the best matching resource (match-making) and then the JC submits the request to it.

Upon successful completion this command returns to the user the submitted job/DAG identifier \textit{jobId} 
(a string that identifies unambiguously the job/DAG in the whole grid), generated by the User Interface, that 
can be later used as a handle to perform monitor and control operations on the job/DAG (e.g. see glite-job-status 
described later in this document). It is important to note that upon submission of a DAG, the WMS-UI command returns
the identifier of the DAG as a whole. This handle will give access (by means of the \emph{glite-job-status} 
command) to the information and identifiers of all DAG subjobs. 


The format of the \textit{jobId} is as follows:

\smallskip
{\verb!https://LB_server_address[:port]/unique_string!}
\smallskip

The unique\_string is a md5 string computed taking into account the following information:

\begin{itemize}
\item IP of the User Interface machine,
\item timestamp,
\item process ID (more WMS-UI instances may occur on the same machine),
\item sequence or just random number (if the User Interface submits jobs in batches and more than one per second 
can be submitted),
\end{itemize}

The final md5 sum is encoded case modified Base64 encoding (":" is used instead fo "/") ensuring reasobable 
uniqueness and compactness of job IDs.


The structure of the \textit{jobId} that could appear in some way complex and not easily readable, has been 
conceived in order to ensure uniqueness and at the same time contain information that are needed by the components 
of the WMS to fulfil user requests. 

\medskip

The \verb!--vo! option allows the user to specify the Virtual Organisation she is currently working for in 
case she is working with non-VOMS credentials. Indeed, if the user proxy credentials currently available on the 
WMS-UI contains VOMS extensions specifying one or more VOs, then the default VO from the proxy credentials has 
precedence over all other possible choiches and is taken as the current working VO.

If the \verb!--vo! option is not used (and the proxy credentials does not contain extensions), then the 
VirtualOrganisation attribute in the JDL is considered. If this attribute has not been specified in the JDL, 
then the default VO specified in the \$GLITE\_LOCATION/\-etc/\-glite\-\_wl\-\_ui\-\_cmd\-\_var.conf 
(DefaultVo field) configuration file is considered. Otherwise an error is returned to the user. 

\medskip

The \verb!--resource! option can be used to target the job submission to a specific known resource identified 
by the provided Computing Element identifier \textit{ce\_id} (returned by glite-job-list-match described later 
in this document). The CE identifier is a string published by the resource (the Glue\-CE\-Unique\-ID field in 
the Glue schema) that univocally identifies a resource belonging to the Grid. The standard format for CEId is:

\medskip
{\verb!<full-hostname>:<port-number>/jobmanager-<service>-<queue-name>!}
\medskip

where \textit{$<$service$>$} is for example lsf, pbs, bqs, condor but can also be a different string as it is 
freely set by the site administrator when the queue is set-up.
No check is done on the CEId format by the WMS-UI as there could be cases in which different formats for the resources
identifiers are adopted by the site administrators.

When the \verb!--resource! option is specified, the WMS skips completely the match making process and directly 
submits the job to the requested CE.  It is important to note that in this case the \textit{".BrokerInfo"} file 
is not generated even if data requirements have been specified in the JDL, so jobs submitted using this option 
should not rely on the \textit{".BrokerInfo"} file information when running on the CE. 

The \textit{".BrokerInfo"} file is a file generated by the RB/Matchmaker during matchmaking and contains 
information about the location where input  data specified in the JDL are physically stored, the SEs that 
are "close" to the CE chosen for submitting the job etc. It is shipped within the InputSandbox to the CE 
where the job is going to run so that it can be used at run-time to get information (through the appropriate 
API) for accessing data. Details about the \textit{".BrokerInfo"} file and the BrokerInfo API can be found 
in ~\cite{brokerinfo}.
 
A way for performing direct submission to a given CE and at the same time having the \textit{".BrokerInfo"} 
file generated by RB and shipped to the CE is to not use the \verb!--resource! option and specify the 
following requirements in the JDL:

\medskip
{\verb!Requirements = other.GlueCEUniqueID == <Ce_identifier>;!}
\medskip

(e.g.  {\scriptsize{Requirements = other.GlueCEUniqueID == "lxde01.pd.infn.it:\-2119/\-jobmanager\--lsf\--grid01";)}}

\medskip

It is also possible to specify the target CE to which submit the job using the \verb!--input! option. 
With the \verb!--input! option an \textit{input\_file} must be supplied containing a list of target CE ids. 
In this case the glite-job-submit command parses the \textit{input\_file} and displays on the standard output 
the list of CE Ids written in the \textit{input\_file}. The user is then asked to choose one CEId between the 
listed ones. The command will then behave exactly like already explained for the \verb!--resource! option. 
The basic idea of this command is to use as \textit{input\_file} the output file generated by the 
glite-job-list-match command when used with the \verb!--output! option (see glite-job-list-match) that 
contains the list of CE Ids (if any) matching the requirements specified in the \textit{jobad.jdl} file.  
An example of a possible sequence of commands is:

\smallskip
\begin{scriptsize}
\begin{verbatim}
> glite-job-list-match --output CEList.out jobad.jdl
> glite-job-submit --input CEList.out jobad.jdl
\end{verbatim}
\end{scriptsize}
\smallskip

If \textit{CEList.out} contains more than one CEId then the user is prompted for choosing one Id from the list.

It is possible to redirect the returned \textit{jobId} to an output file using the \verb!--output! option. 
If the file already exists, a check is performed: if the file was previously created by the command 
glite-job-submit (i.e. it contains a well defined header), the returned \textit{jobId} is appended to the 
existing file every time the command is launched. If the file wasn't created by the command glite-job-submit 
the user will be prompted to choose if overwrite the file or not. If the answer is no the command will abort.

The glite-job-submit command has a particular behaviour when the job description file contains 
the \textit{InputSandbox} attribute whose value is a list of file paths on the WMS-UI machine local disk. 
The purpose of the introduction of the InputSandbox attribute is to list the files that have to be copied 
from the WMS-UI to the CE worker node because they are needed for the job execution.

To better understand, let's suppose to have a job that needs for the execution a certain set of files having 
a small size and available on the submitting machine. Let's also suppose that for performance reasons it is 
preferable not going through the data transfer services for the staging of these files on 
the executing node. Then the user can use the InputSandbox attribute to specify the files that have to be 
copied from the submitting machine to the execution CE. All of them are indeed transferred at job submission 
time together with the job class-ad to the NS that will store them temporarily on its local disk. 
The JobWrapper will then perform the staging of these files on the executing node. The size of files to be 
transferred to the "WMS node" should be small since overfull of WMS node local storage means that no more 
job of this type can be submitted. 

This mechanism can also be used to copy a job executable available locally on the WMS-UI machine to the 
executing CE. Indeed in this case the user has to include this file in the InputSandbox list 
(specifying its absolute path in the file system of the WMS-UI machine) and as Executable attribute value 
has only to specify the file name. On the contrary, if the executable is already available in the file 
system of the executing machine, the user has to specify as Executable an absolute path name for this file 
(if necessary using environment variables). The same argument can be applied to the standard input file 
that is specified through the StdInput JDL attribute.

Since the InputSandbox expression can consist of a great number of file names, it is admitted the use of 
wildcards and environment variables to specify the value of this attribute. 

It is important to note that since the gridftp protocol (the protocol used for the InputSanbox files staging) 
in general doesn't preserve the x flag, the script specified as Executable in the JDL 
(on which chmod +x is done automatically by the WMS JobWrapper), should perform a chmod +x for all the files 
needing execution permission, that are transferred within the InputSandbox of the job.

For the standard output and error of the job the user shall instead always specify just file names 
(without any directory path) through the StdOutput and StdError JDL attributes. To have them copied back on 
the WMS-UI machine it suffices to list them in the OutputSandbox and use after job completion the glite-job-output 
command described later in this document.

The list of data specification JDL attributes is completed by the InputData attribute.

InputData refers to data used as input by the job that are not subjected to staging and are stored in one 
or more storage elements and published in replica catalogues. When the user specifies the InputData attribute 
then he/she also has to provide the protocol her/his application is able to "speak" for accessing data 
(DataAccessProtocol attribute). The InputData attribute should contain a list of Logical File Names (LFN) 
and/or Grid Unique Identifilers (GUID). 


The Arguments attribute in the JDL allows the user to specify all the command line arguments needed to start the 
job. They have to be specified as a single string, e.g. the job sum that is started with:

\smallskip
{\scriptsize{\verb!$ sum  N1 N2 -out result.out!}}
\smallskip

is described by: 

\smallskip
\begin{verbatim}
Executable = "sum";
Arguments = "N1 N2 -out result.out";
\end{verbatim}
\smallskip

If you want to specify a quoted string inside the Arguments then you have to escape quotes with the  
( character. E.g. when describing a job like:

\smallskip
{\scriptsize{\verb!$ grep -i "my name" *.txt!}}
\smallskip

you will have to specify:

\smallskip
\begin{verbatim}
Executable = "/bin/grep";
Arguments = "-i "my name" *.txt";
\end{verbatim}
\smallskip

Analogously, if the job takes as argument a string containing a special character (e.g. the job is the tail 
command issued on a file whose name contains the quotes character, say file1\&file2), since on the shell line 
you would have to write:
 
\smallskip
{\scriptsize{\verb!$ tail -f file1\&file2 !}}
\smallskip

in the JDL you'll have to write:

\smallskip
\begin{verbatim}
Executable = "/usr/bin/tail";
Arguments = "-f file1\\\&file2";
\end{verbatim}
\smallskip

i.e. a $\backslash$ for each special character.

In general, special characters such as \&, |, $>, <$ are only allowed if specified inside a quoted string or 
preceded by triple $\backslash$.
The character "`" cannot be specified in the Arguments attribute of the JDL.

The RetryCount attribute allows setting the number of submission retries for a job upon failure due to some 
grid component (i.e. not to the job itself). RetryCount has to be a positive number and the actual number of 
submission retries for a job is represented by the minimum value between RetryCount itself and the value of the 
MaxrRetryCount parameter in the WM configuration file).  It suffices setting 
RetryCount to 0 to disable job resubmission.

It is important to recall here that the safest way for submitting long-running jobs is to use the proxy renewal 
feature provided by the WMS. To do this the user should use the myproxy-init command (see section~\ref{security}) 
before the glite-job-submit. The myproxy-init command registers indeed in a MyProxy server a valid long-term 
certificate proxy that will be used by WMS to perform a periodic credential renewal for the submitted job. 

When using the myproxy-init command the user has to specify either through the -s option or the MYPROXY\_SERVER 
environment variable the host name of the MyProxy server where to store the certificate proxy.

To trigger the proxy renewal mechanism, the same MyProxy server address has to be specified in the JDL through 
the MyProxyServer attribute (this can also be made a default behaviour through the configuration -  see section~\ref{config}). 
An example of the JDL setting is:

\smallskip
{\verb!MyProxyServer = "skurut.cesnet.cz";!}
\smallskip

Note that the port number must not be provided.

Interactive jobs are specified setting the JDL JobType attribute to "Interactive". When an interactive job is 
submitted, the glite-job-submit command starts a grid console shadow process in the background that listens on a 
port for the job standard streams. Moreover the glite-job-submit command opens a new window where the incoming 
job streams are forwarded. The port on which the shadow process listens is assigned by the OS, but can be forced 
through the ListenerPort attribute in the JDL.

As the command in this case opens a X window, the user should make sure the DISPLAY environment variable is 
correctly set, a X server is running on the local machine and if she/he is connected to the WMS-UI node from remote 
machine (e.g. with ssh) enable secure X11 tunneling. 

\medskip

If this is not possible, the user can specify the \verb!--nogui! option that makes the command provide a simple 
standard non-graphical interaction with the running job.

\medskip

Another option that is reserved for interactive jobs is \verb!--nolisten!: it makes the command forward the job 
standard streams coming from the WN to named pipes on the WMS-UI machine whose names are returned to the user together 
with the OS id of the listener process.  This allows the user to interact with the job through her/his own tools. 
It is important to note that when this option is specified, the WMS-UI has no more control over the launched listener 
process that has to be killed by the user (using the process id returned by the command) when the job is finished.

For interactive jobs the WMS-UI automatically requires for the job outbound IP connectivity on the WN adding  
(in AND to the user defined expression) the other.Glue\-Host\-Network\-Adapter\-Outbound\-IP  to the JDL 
Requirements expression.

Checkpointable jobs are specified setting the JDL JobType attribute to  "Checkpointable". When a checkpointable 
job is submitted the user can specify the number (or list) of steps in which the job can be logically decomposed 
and the step to be considered as the initial one. This can be done setting respectively the JDL attributes 
JobSteps and CurrentStep.  CurrentStep is a mandatory attribute and if not provided by the user, it is set 
automatically to 0 by the WMS-UI.

\medskip

The \verb!--chkpt! option allows the submission of a checkpointable job specifying as input a checkpoint state 
generated by a previously submitted job.  This option makes the submitted job start running from the checkpoint 
state given in input and not from the very beginning. 
The initial checkpoint states to be used with this option can be retrieved by means of the glite-job-get-chkpt 
command. A checkpoint state is a JDL file as described in [R3].
MPI jobs are specified setting the JDL JobType attribute to "MPICH". When a MPI job is submitted the presence 
of the NodeNumber attribute (it specifies the required number of CPUs) in the JDL is mandatory and the WMS-UI 
automatically requires the MPICH runtime environment installed on the CE and a number of CPUs at least equal 
to the required number of nodes. This is done adding (in AND to the user defined expression) the following 
expression

\smallskip
\begin{verbatim}
(other.GlueCEInfoTotalCPUs $>=$ NodeNumber) && 
 Member(other.GlueHostApplicationSoftwareRunTimeEnvironment,"MPICH")
\end{verbatim}
\smallskip

to the the JDL Requirements expression.
Lastly the --nomsg option makes the command display neither messages nor errors on the standard output. 
Only the jobId assigned to the job is printed to the user if the command was successful. Otherwise the 
location of the generated log file containing error messages is printed on the standard output. 
This option has been provided to make easier use of the glite-job-submit command inside scripts in 
alternative to the --output option.
It is important to note that the glite-job-submit is a sort of fire-and-forget command, i.e. it exits 
successfully once the JDL has been passed to the NS and the InputSandbox files have been transferred. 
It does not matter about what happens afterwards to the job. Understanding the reason of a job abort can 
however be accomplished by using the glite-job-status (especially looking at the "Status Reason" field) 
and glite-job-get-logging-info on the job identifier returned from the submission.

\medskip
\textbf{Job Description File}
\smallskip

A job description file contains a description of job characteristics and constraints in a class-ad style. 
A general description of the class-ad language is provided in document [A5].
The job description file must be edited by the user to insert relevant information about the job that is 
later needed by the RB to perform the match-making. Job description file entries are strings having the format 
attribute = expression and are terminated by the semicolon character. Attribute expressions can span several 
lines provided the semicolon is put only at the end of the whole expression. Comments must be preceded by a 
sharp character \#) or have to follow the C++ syntax, i.e a double slash (//) at the beginning of each line or 
statements begun/ended respectively with "/*" and "*/". 

Being the class-ad an extensible language, it doesn't exist a fixed set of admitted attributes, i.e. the user 
can insert in the job/DAG description file whatever attribute he believes meaningful to describe her/his requests, 
anyway only the attributes that can be in some way connected with the ones published by the resources 
are taken into account by the Matchmaker/RB for the match-making process. Unrelated attributes are simply 
ignored except when they are used to build the Requirements expression. In the latter case they are indeed 
evaluated and could affect the match-making result. The attributes taken into account by the RB together with 
their meaning are described in detail in document ~\cite{jdl} which is the document that has to be 
followed by the user when writing the JDL description of her/his jobs/DAGs.

There is a small subset of JDL attributes that are compulsory, i.e. that have to be present in a request class-ad 
before it is sent to the Network Server in order to make possible the performing of the match making and 
submission. They can be grouped in two categories: some of them must be provided by the user whilst some other, 
if not provided, are filled by the WMS-UI with configurable default values. 

In particular for the Requirements and Rank attributes, if not present in the job's JDL, the WMS-UI applies default 
values sepcified in the configuration.
Requirements is set to \verb!other.GlueCEStateStatus == "Production"!, i.e. the target CE has to be active. 
The Rank expression is instead set to \verb!- other.GlueCEStateEstimatedResponseTime!, indeed since the greater is 
the value of Rank the better is considered the match, if no expression for 
Rank has been provided, then the resources where the jobs waits a shorter time to pass from the SCHEDULED 
to the RUNNING status are preferred.

A set of built-in classa-ad functions can be used for building the Requirements and Rank expressions (e.g. 
Member, RegExp, SubStr etc.). These functions are described in ~\cite{jdl-lang}.

MPICH jobs are an exception as they have as default rank \verb!other.GlueCEStateFreeCPUs! meaning that the 
preferred resources are the ones having the higher number of free CPUs.

The default values for the Requirements and Rank attributes can be set in the 
\smallskip
\verb!$GLITE_LOCATION/etc/glite_wmsui_cmd_var.conf! 
\smallskip
file or 
\smallskip
\verb!$GLITE_LOCATION/etc/<vo name>/glite_wmsui.conf! 
\smallskip
file if they are specific to a given VO. See section~\ref{config} for details on how to set these defaults.

As the JDL language is an extensible language, it allows the user to freely include new attributes within the 
job description. These attributes are ignored by the WMS for the scheduling but are passed-through by the WMS-UI 
(if their syntax is correct) since they could be relevant for the submitter of for some other component processing 
the JDL.

However if the job description file contains attributes that are unknown to the WMS, the WMS-UI will print a 
warning (when used with the --debug option) listing all of them. 

\newpage
OPTIONS
\smallskip

\textbf{--help}

displays command usage.

\smallskip
\textbf{--version}

displays WMS-UI version.

\smallskip
\textbf{--vo} vo\_name

\begin{itemize}
\item the default VO from the user proxy (if it contains VOMS extensions),
\item the VO specified through the --vo or --config-vo options,
\item the VO specified in the configuration file pointed by the GLITE\-\_WMSUI\-\_CONFIG\-\_VO environment variable,
\item the VirtualOrganisation attribute in the JDL (if the user proxy contains VOMS extensions this value is 
overridden  as above),
\item the default VO specified in the \$GLITE\-\_LOCATION\-/etc\-/<vo name>/glite\-\_wmsui\.conf (DefaultVo 
field) configuration file.
\item If none of the listed trials has success an error is returned and the submission is aborted.
\end{itemize}

\smallskip

\textbf{--resource}  ce\_id

\textbf{-r} ce\_id

if the command is launched with this option, the job-ad sent to the NS contains a line of the type 
SubmitTo = ce\_id  and the job is submitted by the WMS to the resource identified by ce\_id without going 
through the match-making process. Standard format for the CEId is:

\smallskip
{\scriptsize{\verb!<full hostname>:<port number>/jobmanager-<service>-<queue name>!}}
\smallskip

where $<$service$>$ could be for example lsf, pbs, bqs, condor but can also be a different string as it is 
freely set by the site administrator when setting the queue. This option cannot be used for DAGs.

Note that when this option is used, the ".BrokerInfo" file is not generated.

\smallskip

\textbf{--input} file\_path

\textbf{-i} input\_file

if this option is specified, the user will be asked to choose a CEId from a list of CEs contained in 
the file\_path. Once a CEId has been selected the command behaves as explained for the --resource option. 
If this option is used together with the -noint one and the input file contains more than one CEId, then the 
first CEId in the list is taken into account for submitting the job. This option cannot be used for DAGs.

\smallskip

\textbf{--config} file\_path

\textbf{-c} file\_path

if the command is launched with this option, the configuration file pointed to by file\_path is used 
instead of the standard configuration file.

\smallskip
\textbf{--config-vo} file\_path

if the command is launched with this option, the vo-specific configuration file pointed to by file\_path 
is used instead of the standard vo-specific configuration file.

\smallskip

\textbf{--output} file\_path

\textbf{-o} file\_path

writes the generated jobId assigned to the submitted job in the file specified by out\_file. out\_file can 
be either a simple name or an absolute path (on the submitting machine). In the former case the file out\_file 
is created in the current working directory.

\smallskip
\textbf{--chkpt}  file\_path

This option can be used only for checkpointable jobs. The state specified as input is a checkpoint state 
generated by a previously submitted job.  This option makes the submitted job start running from the checkpoint 
state given in input and not from the very beginning. This option cannot be used for DAGs.

The initial checkpoint states to be used with this option can be retrieved by means of the glite-job-get-chkpt 
command (see ~\ref{chkpt}).

\smallskip

\textbf{--nogui} 

This option can be used only for interactive jobs. As the command for such jobs opens a X window, the user 
should make sure a X server is running on the local machine and if she/he is connected to the WMS-UI node from 
remote machine (e.g. with ssh) enable secure X11 tunneling. If this is not possible, the user can specify 
the --nogui option that makes the command provide a simple standard non-graphical interaction with the 
running job.

\smallskip

\textbf{--nolisten}

This option can be used only for interactive jobs. It makes the command forward the job standard streams 
coming from the WN to named pipes on the WMS-UI machine whose names are returned to the user together with the 
OS id of the listener process.  This allows the user to interact with the job through her own tools. 
It is important to note that when this option is specified, the WMS-UI has no more control over the launched 
listener process that has to be killed by the user (using the process id returned by the command) once the 
job is finished.

\smallskip

\textbf{--nomsg}

this option makes the command print on the standard output only the jobId generated for the job if 
submission was successful; the location of the log file containing massages and diagnostics is printed 
otherwise.

\smallskip

\textbf{--lrms} lrms\_name

this option is only for MPICH jobs and must be used togheter with either --resource or --input option;
it specifies the type of the lrms of the resource the user is submitting to. When the batch system 
type of the specified CE resource given is not known, the lrms must be provided while submitting.
For non-MPICH jobs this option will be ignored.

\smallskip

\textbf{--to} [MM:DD:]hh:mm[:[CC]YY]

A job for which no compatible CEs have been found during the matchmaking phase is hold in the WMS Task 
Queue for a certain time so that it can be subjected again to matchmaking from time to time until a compatible 
CE is found.
The JDL ExpiryTime attribute is an integer representing the date and time (in seconds since epoch) until the 
job request has to be considered valid by the WMS. 
This option sets the value for the ExpiryTime attribute to the submitted JDL converting appropriately the 
absolute timestamp provided as input. It overrides, if present, the current value.
If the specified value exceeds one day from job submission then it is not taken into account by the WMS.

\smallskip

\textbf{--valid} hours:minutes

\textbf{-v} hours:minutes

A job for which no compatible CEs have been found during the matchmaking phase is hold in the WMS Task
Queue for a certain time so that it can be subjected again to matchmaking from time to time until a compatible
CE is found.
The JDL ExpiryTime attribute is an integer representing the date and time (in seconds since epoch) until the
job request has to be considered valid by the WMS.
This option allows to specify the validity in hours and minutes from submission time of the submitted JDL.
When this option is used the command sets the value for the ExpiryTime attribute converting appropriately the
relative timestamp provided as input. It overrides, if present, the current value.
If the specified value exceeds one day from job submission then it is not taken into account by the WMS.


\smallskip

\textbf{--noint}

if this option is specified every interactive question to the user is skipped and all warning messages and 
errors (if occurred) are written to the file glite-job-submit\_$<$UID$>$\_$<$PID$>$\_$<$timestamp$>$.log 
under the /tmp directory. Log file location is configurable.

\smallskip

\textbf{--debug}

when this option is specified, information about parameters used for the API functions calls inside the 
command are displayed on the standard output and are written to 
glite-job-submit\_$<$UID$>$\_$<$PID$>$\_$<$timestamp$>$.log file under the /tmp directory too. 
Log file location is configurable.

\smallskip

\textbf{--logfile file\_path}

when this option is specified, the command log file is relocated to the location pointed by file\_path

\smallskip

\textbf{jdl\_file}

this is the file containing the JDL describing the job to be submitted. It must be the last argument 
of the command.

\medskip
EXIT STATUS
\smallskip

\textbf{glite-job-submit} exits with a status value of 0 (zero) upon success, and $>$0 (greater than zero) 
upon failure. 

\medskip
EXAMPLES
\medskip

1.{\scriptsize{\verb!> glite-job-submit -vo cms myjob1.jdl! }}
where myjob1.jdl is as follows:
\begin{scriptsize}
\begin{verbatim}

##############################################!
#                                                 
# -------- Job description file ----------
# 	                  
##############################################!
[
  JobType = "Normal"�;
  Executable     = "$(CMS)/fpacini/exe/sum.exe";
  InputData      = "lfn:testbed0-00019";
  
  DataAccessProtocol = "gridftp";
  Rank	         = other.GlueCEPolicyMaxCPUTime;
  Requirements   = other.GlueCEInfoLRMSType  == "Condor"; 
  InputData = {
        "lfn:mydatafile1",
        "lfn:mydatafile2",
        "guid:135b7b23-4a6a-11d7-87e7-9d101f8c8b70"
    };
    StorageIndex = 
         "http://lxb1434.cern.ch:8080/EGEE/glite-data-/FiremanCatalog";
    DataAccessProtocol = {"gridftp", "gliteio"};
 ]

\end{verbatim}
\end{scriptsize}

Will display something like:

\begin{scriptsize}
\begin{verbatim}

Selected Virtual Organisation name (from proxy certificate extension): EGEE
Connecting to host gundam.cnaf.infn.it, port 7772
Logging to host gundam.cnaf.infn.it, port 9002


*********************************************************************************************
                               JOB SUBMIT OUTCOME
 The job has been successfully submitted to the Network Server.
 Use glite-job-status command to check job current status. Your job identifier is:

 - https://gundam.cnaf.infn.it:9000/VBVtU_6a7KZ2D-SLP7UTEA


*********************************************************************************************

\end{verbatim}
\end{scriptsize}

\smallskip
2.{\scriptsize{\verb!> glite-job-submit --chkpt  /home/test/state10.chkpt myjob2.jdl! }} 
Submits the checkpointable job described by myjob2.jdl that will start running from the initial state 
state10.chkpt. 

See also glite-job-list-match~\ref{listmatch}, glite-job-attach~\ref{attach}, glite-job-get-chkpt~\ref{chkpt}.


\newpage
\subsubsection{glite-job-output}
\label{output}

This command retrieves the job output files (specified by the OutputSandbox attribute of the job-ad) 
from the RB node and stores them on the submitting machine local disk. When issued for a DAG it retrieves 
the output sandboxes of all DAG nodes.

\medskip
SYNOPSIS 
\smallskip

\textbf{glite-job-output [options] $<$job Id(s)$>$}

\begin{scriptsize}
\begin{verbatim}

Options:
   --help
   --version
   --input, -i     <file_path>
   --dir           <directory_path>
   --config, -c    <file_path>
   --noint
   --debug
   --logfile 	   <file_path>

\end{verbatim}
\end{scriptsize}

\medskip
 DESCRIPTION \smallskip

The \textbf{glite-job-output} command can be used to retrieve the output files of a job/DAG that has been 
submitted through the glite-job-submit command with a job/DAG description file including the OutputSandbox 
attribute. After the submission, when the job has terminated its execution, the user can download the 
files generated by the job and temporarily stored on the RB machine as specified by the OutputSandbox 
attribute, issuing the glite-job-output with as input the Id returned by the glite-job-submit. 
It is also possible to specify a list of job identifiers when calling this command or an input file containing 
Ids by means of the --input option. When the --input is used, the user is requested to choose all, one or 
a subset of the identifiers contained in the input file. 

It is important to note that the OutputSandbox of a submitted job can only be retrieved when the job has reached 
the Done status indicating that the job has successfully terminated its execution and the OutputSandbox files are 
ready for retrieval on the RB node. glite-job-output  will always fail for jobs that are not yet in the Done status.
The user can decide the local directory path on the WMS-UI machine where these files have to be stored by means of 
the --dir option, otherwise the retrieved files are put in a default location specified in the 
\$GLITE\_WL\_LOCATION\-/etc\-/glite\_wl\_ui\_cmd\_var.conf configuration file (OutputStorage parameter). 
In both cases a sub-directory will be added to the path supplied. The name of this sub-directory is the 
unique string of the jobId identifier (see command glite-job-submit for details on the jobId structure) prefixed 
by the user login name (value of the LOGNAME environment variable). For a DAG, a further subdirectory is created 
,inside the one just mentioned, for each DAG node.

If the user wants to use her/his "private" configuration file, this can be done using option --config path\_name. 
As a consequence the glite-job-output command looks for the file "path\_name" instead of the standard 
configuration file. If this file does not exist the user is notified with an error message and the command is 
aborted. 

\medskip
OPTIONS 
\smallskip

{\bf--help}

 displays command usage.
 
\smallskip
{\bf --version}

displays WMS-UI version.

\smallskip
{\bf --dir directory\_path}

retrieved files (previously listed by the user through the OutputSandbox attribute of the job description file)  
are stored in the location indicated by directory\_path/$<$login name$>$\_$<$jobId unique string$>$. A further 
directory named $<$node Id unique string$>$ is created inside the just mentioned dir for each DAG node in case 
the command is issued for a DAG.  

\smallskip

{\bf --config} file\_path

{\bf -c} file\_path

if the command is launched with this option, the configuration file pointed to by file\_path is used instead 
of the standard configuration file.

\smallskip
{\bf --noint}

if this option is specified every interactive question to the user is skipped. All warning messages and errors 
(if occurred) are written to the file glite-job-output\_$<$UID$>$\_$<$PID$>$\_$<$timestamp$>$.log under the 
/tmp directory. Location of log file is configurable.

\smallskip
{\bf --debug}

when this option is specified, information about parameters used for the API functions calls inside the command 
are displayed on the standard output and are written to 
glite\_job\_output\_$<$UID$>$\_$<$PID$>$\_$<$timestamp$>$.log file under the /tmp directory too. 
Location of log file is configurable.

\smallskip
{\bf --logfile} file\_path

when this option is specified, the command log file is relocated to the location pointed by file\_path

\smallskip
{\bf jobId}

job identifier returned by glite-job-submit. If a list of oe or more job identifiers is specified, job Ids 
have to be separated by a blank. Job identifiers must be last argument of the command.

\smallskip
{\bf --input} file\_path
{\bf -i} file\_path

this option makes the command return the OutputSandbox files for each jobId contained in the file\_path. 
This option can't be used if one (or more) jobIds have been already specified. As explained in section~\ref{commonbeh}
The format of the input file must be as follows: one jobId for each line and comment lines must begin with 
a "\#" or a "\*" character. 

\medskip
EXIT STATUS
\smallskip

{\bf glite-job-output} exits with a status value of 0 (zero) upon success, $>$0 upon failure and $<$0 upon 
partial failure. An example of partial failure is when more than one job identifiers has been specified and the 
OuputSandbox could be retrieved only for some of them. 

\medskip
EXAMPLES 
\smallskip

Let us consider the following command issued by the user logges as user \emph{mrossi}:
\begin{scriptsize}
\begin{verbatim}

> glite-job-output https://ibm139.cnaf.infn.it:9000/CiXMLojKC_iLsvSHfEhqIQ --dir /home/data
It retrieves the files listed in the OutputSandbox attribute 
of job identified by 
https://ibm139.cnaf.infn.it:9000/CiXMLojKC_iLsvSHfEhqIQ  
from the RB node and stores them locally in 
/home/data/mrossi_CiXMLojKC_iLsvSHfEhqIQ.

\end{verbatim} 
\end{scriptsize}

\newpage
\subsubsection{glite-job-list-match} 
\label{listmatch}

Returns the list of resources fulfilling job requirements specified in the JDL job description

\medskip
 SYNOPSIS 
\smallskip
 
\textbf{glite-job-list-match  [options]  $<$jdl file$>$}

\begin{scriptsize}
\begin{verbatim}

Options:
   --help
   --version   
   --verbose       
   --rank          
   --config, -c    <file_path>
   --config-vo     <file_path>
   --vo            <vo_name>
   --output, -o    <file_path>
   --noint         
   --debug         
   --logfile       <file_path>

\end{verbatim} 
\end{scriptsize}

\textbf{glite-job-list-match} displays the list of identifiers of the resources on which the user is 
authorized and satisfying the job requirements included in the job description file. The CE identifiers 
are returned either on the standard output or in a file according to the chosen command options, and are 
strings univocally identifying the CEs published in the IS. This command cannot be used for DAG requests.  

The returned CEIds are listed in decreasing order of rank, i.e. the one with the best (greater) rank is 
in the first place and so on.

The \textbf{--rank} option makes the command also display the rank value for each found CEId.

The \textbf{--vo} option allows the user to specify the Virtual Organisation she/he is currently working 
for in case she/he is working with non-VOMS credentials.

Indeed, if the user proxy credentials currently available on the WMS-UI contains VOMS extensions specifying 
one or more VOs, then the default VO from the proxy credentials has precedence over all other possible 
choiches and is taken as the current working VO.

If the --vo option is not used (and the proxy credentials does not contain extensions), then the 
VirtualOrganisation attribute in the JDL is considered. If this attribute has not been specified in the JDL, 
then the default VO specified in the \$GLITE\_LOCATION/\-etc/\-glite\_wmsui\_cmd\_var.conf  
(DefaultVo field) configuration file is considered. Otherwise an error is returned to the user. 

glite-job-list-match requires a job description file in which job characteristics and requirements are 
expressed by means of a class-ad. The job description file is first syntactically checked and then used 
as the main command-line argument to glite-job-list-match. The Network Server is only contacted to find 
job compatible resources; the job is not submitted. See the glite-job-submit~\ref{submit} command for 
general rules for building the job description file.

If the user wants to use his "private" configuration, file this can be done using option --config path\_name.

 
The option \textbf{--verbose} of the dg-job-list-match command can be used to obtain on the standard output 
the class-ad sent to the RB generated from the job description.

The \textbf{--output} option makes the command save the list of compatible resources into the specified file. 
If the provided file name is not an absolute path, then the output file is created in the current working dir.

\medskip
OPTIONS

\smallskip 
\textbf{ --help} 

displays command usage.

\smallskip 
\textbf{ --version}

displays WMS-UI version.

\smallskip 
\textbf{ --verbose}

\textbf{ -v}

displays on the standard output the job class-ad that is sent to the Network Server generated from the 
job description file. This differs from the content of the job description file since the WMS-UI adds to it 
some attributes that cannot be directly inserted by the user (e.g., defaults for Rank and Requirements if 
not provided, VirtualOrganisation etc).

\smallskip 
\textbf{ --rank}

displays the "matching" CEIds and the associated ranking values.

\smallskip 
\textbf{ --vo vo\_name}

This option allows the user to specify the Virtual Organisation she/he is currently working for. 

If the user proxy contains VOMS extensions then the VO specified through this option is overridden by 
the default VO contained in the proxy (i.e. this option is only useful when working with non-VOMS proxies). 
The following precedence rule is followed for determining the user's VO:

\begin{itemize}
\item the default VO from the user proxy (if it contains VOMS extensions),
\item the VO specified through the --vo or --config-vo options,
\item the VO specified in the configuration file pointed by the GLITE\_WMSUI\_CONFIG\_VO environment variable,
\item the VirtualOrganisation attribute in the JDL (if the user proxy contains VOMS extensions this value 
is overridden  as above),
\item the default VO specified in the \$GLITE\_LOCATION\-/etc\-/glite\-\_wmsui\-\_cmd\-\_var.conf (DefaultVo field) 
configuration file.
\end{itemize}

If none of the listed trials has success an error is returned and the submission is aborted.

\smallskip 
\textbf{ --config file\_path}

{\bf -c} file\_path

if the command is launched with this option, the configuration file pointed to by file\_path is used instead 
of the standard configuration file.

\smallskip 
\textbf{ --config-vo file\_path}

if the command is launched with this option, the vo-specific configuration file pointed to by file\_path is 
used instead of the standard vo-specific configuration file.
	
\smallskip 
\textbf{ --output file\_path}

{\bf -o} file\_path

returns the CEIds list in the file specified by file\_path. file\_path  can be either a simple name or an absolute 
path (on the submitting machine). In the former case the file file\_path is created in the current working directory.

	
\smallskip 
\textbf{ --noint}

if this option is specified every interactive question to the user is skipped. All warning messages and errors (if any) 
are written to the file glite-job-list-match $<$UID$>$\_$<$PID$>$\_$<$timestamp$>$.log under the /tmp directory. 
Location of the log file is configurable.

\smallskip 
\textbf{ --debug}

when this option is specified, information about the API functions called inside the command are displayed on the 
standard output and are written to the file glite-job-list-match\_$<$UID$>$\_$<$PID$>$\_$<$timestamp$>$.log under 
the /tmp directory too. Location of the log file is configurable.

\smallskip 
\textbf{ --logfile file\_path}

when this option is specified, the command log file is relocated to the location pointed by file\_path

jdl\_file

this is the file containing the classad describing the job to be submitted. It must be the last argument of 
the command.
\\
\medskip
EXIT STATUS \smallskip


{\bf glite-job-list-match} exits with a status value of 0  upon success, and a $>$0 value upon failure. 

\medskip
EXAMPLES \smallskip


Let us consider the following command:

\begin{scriptsize}
\begin{verbatim}
> glite-job-list-match myjob.jdl
where the job description file myjob.jdl looks like:


################################   
#                                                 
# Sample Job Description File 
# 	                  
################################         
JobType = "Normal";
Executable   = "sum.exe";
StdInput    = "data.in";
InputSandbox = {"/home/fpacini/exe/sum.exe","/home1/data.in"};
OutputSandbox = {"data.out","sum.err"};
Rank	       = other.GlueCEPolicyMaxCPUTime;
Requirements = other.GlueCEInfoLRMSType == "Condor" &&
               other.GlueHostArchitecturePlatformType=="INTEL" &&
               other.GlueHostOperatingSystemName == "LINUX" && 
               other.GlueCEStateFreeCPUs >= 2;


\end{verbatim}
\end{scriptsize}

In this case the job requires CEs being Condor Pools of INTEL LINUX machines with at least 2 free Cpus.  
Moreover the Rank expression states that queues with higher maximum CPU time allowed for jobs are preferred. 

The response of such a command is something as follows:

\begin{scriptsize}
\begin{verbatim}
******************************************************************
                 Computing Element IDs LIST 
The following CE(s) matching your job requirements have been found:
		   *CEId*
 bbq.mi.infn.it:2119/jobmanager-pbs-dque
 skurut.cesnet.cz:2119/jobmanager-pbs-wp1
*******************************************************************

\end{verbatim}
\end{scriptsize}

See also glite-job-submit~\ref{submit}.

\newpage
\subsubsection{glite-job-cancel} 
\label{cancel}

Cancels one or more submitted jobs.

\medskip
 SYNOPSIS 
\smallskip

\textbf{glite-job-cancel  [options]  $<$job Id(s)$>$}

\begin{scriptsize}
\begin{verbatim}

Options:
   --help
   --version
   --all
   --input, -i     <file_path>
   --config, -c    <file_path>
   --config-vo     <file_path>
   --vo            <vo_name>
   --output, -o    <file_path>
   --noint
   --debug
   --logfile 	   <file_path>

\end{verbatim}
\end{scriptsize}

\medskip
DESCRIPTION 
\smallskip 

This command cancels a job previously submitted using glite-job-submit. Before cancellation, it prompts the 
user for confirmation. The cancel request is sent to the Network Server that forwards it to the WM that 
fulfils it. Note that this command cannot be issued directly on DAG nodes, it has to be issued on the DAG 
identifier.

glite-job-cancel can remove one or more jobs: the jobs to be removed are identified by their job identifiers 
(job Ids returned by glite-job-submit) provided as arguments to the command and separated by a blank space. 
The result of the cancel operation is reported to the user for each specified jobId.

If the --all option is specified, all the jobs owned by the user submitting the command are removed. 
When the command is launched with the --all option, no jobId can be specified. It has to be remarked that 
only the owner of the job can remove the job.  When the --all option is specified the WMS-UI queries each LB 
listed in the vo-specific configuration file \$GLITE\_LOCATION/etc/$<$vo\_name$>$/glite\_wmsui.conf for getting 
the identifiers of all the jobs owned by the user identified by her/his certificate subject. Afterwards the 
WMS-UI sends a cancellation request to the NS for each job being in a status for which the cancellation 
is allowed.

Job states for which cancellation is allowed are:
\begin{itemize}
  \item Submitted
  \item Waiting 
  \item Ready 
  \item Scheduled 
  \item Running
  \item Unknown
\end{itemize}

For all the other job states the cancellation request will result in a failure.
If the user wants to use hes "private" configuration file this could be done using option --config file\_path.

The --input option permits to specify a file (file\_path) that contains the jobIds to be removed. 
The format of the file must be as follows: one jobId for each line and comment lines must begin 
with a "\#" or a "\*" character. When using this option the user is interrogated for choosing among 
all, one or a subset of the listed job identifiers. If the file\_path does not represent an absolute 
path the file will be searched in the current working directory.

\medskip
OPTIONS \smallskip

\textbf{ --help}
displays command usage.

\smallskip 
\textbf{ --version}

displays WMS-UI version.

\smallskip 
\textbf{ --all}

cancels all job owned by the user submitting the command. This option can't be used either if one or 
more Ids have been specified explicitly or with the -input option.

\smallskip 
\textbf{ --input} file\_path

{\bf -i} file\_path

cancels jobs/DAGs identified by the Ids contained in the file\_path. This option can't be used neither if one or 
more Ids have been specified nor with the -all option.

\smallskip 
\textbf{ --config file\_path}

{\bf -c} file\_path

if the command is launched with this option, the configuration file pointed to by file\_path is used instead 
of the standard configuration file.

\smallskip 
\textbf{ --config-vo} file\_path

if the command is launched with this option, the vo-specific configuration file pointed to by file\_path 
is used instead of the standard vo-specific configuration file. This option is allowed only when used 
together with the --all one.

\smallskip 
\textbf{ --vo} vo\_name

This option allows the user to specify the Virtual Organisation she/he is currently working for. 
If the user proxy contains VOMS extensions then the VO specified through this option is overridden by 
the default VO contained in the proxy (i.e. this option is only useful when working with non-VOMS proxies). 
The following precedence rule is followed for determining the user's VO:

\begin{itemize}
\item the default VO from the user proxy (if it contains VOMS extensions),
\item the VO specified through the --vo or --config-vo options,
\item the VO specified in the configuration file pointed by the GLITE\_WMSUI\_CONFIG\_VO environment variable,
\item the VirtualOrganisation attribute in the JDL (if the user proxy contains VOMS extensions this value is 
overridden  as above),
\item the default VO specified in the \$GLITE\_LOCATION/etc/glite\_wmsui\_cmd\_var.conf (DefaultVo field) 
configuration file.
\end{itemize}

If none of the listed trials has success an error is returned and the submission is aborted. 
This option is not allowed when one or more jobIds are specified as command arguments.


\smallskip 
\textbf{ --output} file\_path
{\bf -o} file\_path

writes the cancel results in the file specified by file\_path instead of the standard output. file\_path 
can be either a simple name or an absolute path (on the submitting machine). In the former case the file 
file\_path is created in the current working directory.

\smallskip 
\textbf{ --noint}

if this option is specified every interactive question to the user is skipped. All warning messages and 
errors (if occurred) are written to the file 
glite-job-cancel\_$<$UID$>$\_$<$PID$>$\_$<$timestamp$>$.log under the /tmp directory. Location of the 
log file is configurable.

{\bf --debug}

when this option is specified, information about the API functions called inside the command are displayed 
on the standard output and are written to the file glite-job-cancel\_$<$UID$>$\_$<$PID$>$\_$<$timestamp$>$.log 
under the /tmp directory too. Location of the log file is configurable.

\smallskip 
\textbf{ --logfile file\_path}

when this option is specified, the command log file is relocated to the location pointed by file\_path

\smallskip 
\textbf{ jobId}

job identifier returned by glite-job-submit. The job identifier list must be the last argument of this command.
\\  
\medskip
EXIT STATUS 
\smallskip

{\bf glite-job-cancel} exits with a status value 0 if all the specified jobs were cancelled successfully, $>$0 
if errors occurred for each specified job id  and $<$0 in case of partial failure. An example of partial failure 
is when more then one job has been specified: some jobs could be successfully removed and some others could be 
not removed. 

\medskip
EXAMPLES 
\smallskip

1.{\scriptsize{\verb!> glite-job-cancel --input joblist.txt!}}

where joblist.txt is a file containing 3 jobIds, displays the following confirmation message:

\begin{scriptsize}
\begin{verbatim}

Are you sure you want to remove all jobs specified? [y/n]n: y

================ glite-job-cancel Success  ================
The cancel request for the following job(s) has been 
successfully submitted to NS: 
 - https://ibm139.cnaf.infn.it:9000/nUbiIiMFmY1oIusAaWxPhg
 - https://ibm139.cnaf.infn.it:9000/VtMvhs8z7WGCptt92ZMPIQ
 - https://ibm139.cnaf.infn.it:9000/yKTKyrdSgHKQ1wwwSocJiw
============================================================

\end{verbatim}
\end{scriptsize}

In this case the command exit code is 0.

2.{\scriptsize{\verb!> glite-job-cancel --all --noint!}}

removes all job owned by the user submitting the command.

See also glite-job-submit~\ref{submit}.

\newpage
\subsubsection{glite-job-status}
\label{status}

Displays bookkeeping information about submitted jobs/DAGs.

\medskip
 SYNOPSIS 
\smallskip

\textbf{glite-job-status  [options]  $<$job Id(s)$>$}

\begin{scriptsize}
\begin{verbatim}

Options:
   --help              
   --version           
                       
   --all               
   --input, -i         <input file>
   --verbosity         [0|1|2|3]
   --from              [MM:DD:]hh:mm[:[CC]YY]
   --to                [MM:DD:]hh:mm[:[CC]YY]
   --config, -c        <config file>
   --user-tag          <tag name>=<tag value>
   --status, -s        <status value>
   --exclude, -e       <exclude value>
   --config-vo         <config-vo file>
   --vo                <vo value>
   --output, -o        <output file>
   --noint             
   --debug             
   --logfile           <logfile file>

\end{verbatim} 
\end{scriptsize}


\medskip
DESCRIPTION 
\smallskip
 
This command prints the status of a job previously submitted using glite-job-submit. The job status request is 
sent to the LB that provides the requested information. This can be done during the whole job life.
When issued on a DAG Id the command displays status information for the DAG itself and all its nodes.

glite-job-status can monitor one or more jobs: the jobs to be checked are identified by one or more job 
identifiers (job Ids returned by glite-job-submit) provided as arguments to the command and separated by a 
blank space. 

If the --all option is specified, information about all the jobs owned by the user submitting the command is 
printed on the standard output. When the command is launched with the  --all option, neither can an jobId be 
specified nor can the  --input option be specified. 

The --input option permits to specify a file (file\_path) that contains the jobIds to monitor. The format of 
the file must be as follows: one jobId for each line and comment lines have to begin with a "\#" or a "\*" 
character. When using this option the user is requested for choosing among all, one or a subset of the listed 
job identifiers. If the file\_path does not represent an absolute path, it will be searched in the current 
working directory.

If the user wants to use her "private" configuration file, this can be done using option --config file\_path. 

The same applies for the vo-specific configuration file and the --config-vo option. 

The --verbosity option allows setting the detail level of the returned information.  This option can be 
specified with four values, 0, 1, 2 and 3. The default level of verbosity is 1 unless otherwise specified 
in the WMS-UI configuration file \$GLITE\_LOCATION/etc/glite\_wmsui\_cmd\_var.conf (\textit{DefaultStatusLevel} parameter).

Hereafter are listed the information displayed according to the verbosity level:

Verbosity equal 0:

\begin{itemize}
  \item job Id          (the job unique identifier)
  \item Current Status  (the job current status)
\end{itemize}


Verbosity equal 1:

\begin{itemize}
  \item job Id 		(the job unique identifier)
  \item Current Status 	(the job current status)
  \item exit\_code	(Unix exit code - if applicable)
  \item Status Reason	(reason for being in this state)
  \item Reached on 	(date/time when the job entered actual state)
  \item destination 	(ID of CE where the job has been submitted - if applicable)
\end{itemize}

With verbosity equal 2 some additional information fields are added such as:

\begin{itemize}
\item cancelling	(boolean indicating if a cancellation request for the job is in progress)
\item cancelReason	(Reason of cancel)
\item ce\_node		(Worker node where the job is executed)  
\item children\_hist 	(summary -- histogram -- of children job states)
\item children\_num  	(number of subjobs)          
\item subjob\_failed    (Subjob failed -- the parent job will fail too)
\item condorId          (Id within Condor-G)
\item cpuTime           (Consumed CPU time)
\item expectUpdate      (Boolean indicating that some logged information has not arrived yet)
\item expectFrom 	(Sources of the missing information)
\item jobtype           (Type of the request: 0 = Job, 1 = DAG)
\item lastUpdateTime	(Last known event of the job)
\item location          (location Where the job is being processed)
\item network\_server   (Network server handling the job)   
\item owner             (certificate subject of Job owner)
\item resubmitted       (boolean indicating that the job was resubmitted)
\end{itemize}

Lastly, with verbosity equal 3 there are the following additional fields:

\begin{itemize}
\item jdl		(User submitted job description)
\item matched\_jdl	(Full job description after matchmaking)
\item condor\_jdl    	(ClassAd passed to Condor-G for job submission)
\item rsl		(Job RSL sent to Globus)
\item stateEnterTimes   (When all previous states were entered)
\end{itemize}

Information fields that are not available (i.e. not returned by the LB because not applicable for the given 
status) are not printed at all to the user.

\medskip
OPTIONS 
\smallskip

\smallskip 
\textbf{ --help} 

displays command usage.

\smallskip 
\textbf{ --version}

displays WMS-UI version.

\smallskip 
\textbf{ --all}

displays status information about all job owned by the user submitting the command. This option can't be used 
either if one or more jobIds have been specified or if the  --input option has been specified. All LBs listed 
in the vo-specific WMS-UI configuration file \$GLITE\_LOCATION/etc/$<$vo\_name$>$/glite\_wmsui.conf are contacted to 
fulfil this request.


\smallskip 
\textbf{ --input} input\_file

{\bf -i} input\_file

displays bookkeeping info about jobIds contained in the input\_files. When using this option the user is 
interrogated for choosing among all, one or a subset of the listed job identifiers. This option can't be used 
either if one or more jobIds have been specified or if the  --all option has been specified. 

\smallskip 
\textbf{ --verbosity} verb\_level

{\bf --v} verb\_level 

sets the detail level of information about the job displayed to the user. Possible values for verb\_level are 
0,1,2 and 3.


\smallskip 
\textbf{ --from} [MM:DD:]hh:mm[:[CC]YY]

  makes the command query LB for jobs that have been submitted (more precisely entered the "Submitted" status)
  after the specified date/time. If only hours and minutes are specified then the current day is taken into
  account. If the year is not specified then the current year is taken into account.


\smallskip 
\textbf{ --to} [MM:DD:]hh:mm[:[CC]YY]

  makes the command query LB for jobs that have been submitted (more precisely entered the "Submitted" status)
  before the specified date/time. If only hours and minutes are specified then the current day is taken into
  account. If the year is not specified then the current year is taken into account.



\smallskip 
\textbf{ --config} file\_path

{\bf -c} file\_path

if the command is launched with this option, the configuration file pointed to by file\_path is used instead 
of the standard configuration file.

\smallskip 
\textbf{ --config-vo} file\_path

if the command is launched with this option, the vo-specific configuration file pointed to by file\_path is 
used instead of the standard vo-specific configuration file. This option is allowed only when used together 
with the  --all one.


\smallskip 
\textbf{ --user-tag}  tag\_name=tag\_value

  if the command is launched with this option, the WMS-UI returns the status of the jobs that have been submitted with
  an associated tag named $<$tag name$>$ whose value is $<$tag value$>$


\smallskip 
\textbf{ --status}  status\_value

{\bf -s}  status\_value

  if the command is launched with this option, the WMS-UI returns the status information for all those jobs that are
  currently in the status specified by $<$status\_value$>$ Possible status values are: SUBMITTED, WAITING, READY,
  SCHEDULED, RUNNING, DONE, ABORTED, CANCELLED, CLEARED.


\smallskip 
\textbf{ --exclude} status\_value

{\bf -e}       status\_value

  if the command is launched with this option, the WMS-UI returns the status information for all those jobs that are
  currently in a status different from the one specified by $<$status\_value$>$ Possible status values are: SUBMITTED,
  WAITING, READY, SCHEDULED, RUNNING, DONE, ABORTED, CANCELLED, CLEARED.


\smallskip 
\textbf{ --vo vo\_name}

This option allows the user to specify the Virtual Organisation she/he is currently working for. 

If the user proxy contains VOMS extensions then the VO specified through this option is overridden by the 
default VO contained in the proxy (i.e. this option is only useful when working with non-VOMS proxies). 
The following precedence rule is followed for determining the user's VO:
\begin{itemize}
\item the default VO from the user proxy (if it contains VOMS extensions),
\item the VO specified through the  --vo or  --config-vo options,
\item the VO specified in the configuration file pointed by the GLITE\_WMSUI\_CONFIG\_VO environment variable,
\item the VirtualOrganisation attribute in the JDL (if the user proxy contains VOMS extensions this value is 
overridden  as above),
\item the default VO specified in the \$GLITE\_LOCATION/etc/glite\_wmsui\_cmd\_var.conf (DefaultVo field) configuration 
file.
\end{itemize}

If none of the listed trials has success an error is returned and the submission is aborted. This option is 
allowed only when used together with the --all one.


\smallskip 
\textbf{ --output file\_path}

{\bf -o} file\_path

writes the bookkeping information in the file specified by file\_path instead of the standard output. file\_path can be either a simple name or an absolute path (on the submitting machine). In the former case the file file\_path is created in the current working directory.

\smallskip 
\textbf{ --noint}

if this option is specified every interactive question to the user is skipped. All warning messages and errors (if any) are written to the file glite-job-status\_$<$UID$>$\_$<$PID$>$\_$<$timestamp$>$.log under the /tmp directory. Location of log file is configurable.


\smallskip 
\textbf{ --debug}

when this option is specified, information about the API functions called inside the command are displayed on the standard output and are written to the file glite-job
status\_$<$UID$>$\_$<$PID$>$\_$<$timestamp$>$.log under the /tmp directory too. Location of log file is configurable.

\smallskip 
\textbf{ --logfile} file\_path

when this option is specified, the command log file is relocated to the location pointed by file\_path

\smallskip 
\textbf{ jobId}

job identifier returned by glite-job-submit. Job identifiers must always be provided as last arguments of the 
command.


\medskip
EXIT STATUS
\smallskip

{\bf glite-job-status} exits with a value of 0 if the status of all the specified jobs is retrieved 
correctly, $>$0 if errors occurred for each specified job id and $<$0 in case of partial failure. An example 
of partial failure is when more then one job is specified: status info could be successfully retrieved for 
some jobs and not retrieved for some others.

\medskip
EXAMPLES 
\smallskip


\begin{scriptsize}
\begin{verbatim}
> glite-job-status -v 1 https://ibm139.cnaf.infn.it:9000/_tO6hdgToYKGCuV68q-gqQ

\end{verbatim}
\end{scriptsize}

displays the following lines:

\begin{scriptsize}
\begin{verbatim}

******************************************************
BOOKKEEPING INFORMATION:

Printing status info for the Job : 
https://ibm139.cnaf.infn.it:9000/_tO6hdgToYKGCuV68q-gqQ

Current Status: Scheduled 
Destination:    bbq.mi.infn.it:2119/jobmanager-pbs-dque
Status Reason:  Job successfully submitted to Globus
reached on:     Tue May  6 16:14:59 2003
********************************************************
>

\end{verbatim}
\end{scriptsize}

See also glite-job-submit~\ref{submit}



\newpage
\subsubsection{glite-job-get-logging-info}
\label{logging}

Displays logging information about submitted jobs/DAGs.

\medskip
 SYNOPSIS 
\smallskip

\textbf{glite-job-logging-info  [options]  $<$job Id(s)$>$}

\begin{scriptsize}
\begin{verbatim}

Options:
    --help          
    --version                    
    --input, -i     <file_path>        
    --verbosity, -v [0|1|2|3]
    --config, -c    <file_path>
    --output, -o    <file_path>
    --noint         
    --debug         
    --logfile       <file_path>

\end{verbatim} 
\end{scriptsize}         

\medskip
 DESCRIPTION \smallskip 

This command queries the LB persistent DB for logging information about jobs previously submitted using 
glite-job-submit. The job logging information are stored permanently by the LB service and can be retrieved 
also after the job has terminated its life-cycle, differently from the bookkeeping information that are in 
some way "consumed" by the user during the job existence (i.e. at a given time only the current status information
are returned). 
When issued for a DAG, the glite-job-logging-info command only returns events associated to the DAG. Logging 
information about the DAG nodes have to be quieried using the individual node identifiers. 
The glite-job-logging-info request is sent to the LB service that queries the DB and returns the 
retrieved information. Content of the logging information varies according to the type of the event they 
are related to. The most common information fields are: 

\begin{itemize}
  \item Event	          (event type)
  \item source            (WMS component which generated the event)
  \item result            (result of the attempt)
  \item destination       (destination where the job is being transferred to)
  \item timestamp         (timestamp of event generation)
\end{itemize}

\smallskip

The  --verbosity option allows setting the detail level of the returned information.  This option can be 
specified with four values, 0, 1, 2 and 3. The default level of verbosity is 1 unless otherwise specified in 
the WMS-UI configuration file \$GLITE\_LOCATION/etc/glite\_wmsui\_cmd\_var.conf  (\textit{DefaultLoggingLevel} parameter).
The information listed above is displayed when the chosen verbosity level is 1. If the command is issued 
with verbsoity 0 only the events names are displayed (\textit{Event} field above) whilst with 2 as verbosity flag the 
following additional information is shown:

\begin{itemize}
  \item host 		  (hostname of the machine where the event was generated)
  \item dest\_host        (destination hostname)
  \item dest\_instance    (instance of destination WMS component) 
  \item user              (identity  -- cert. subj.  --  of the generator)
  \item dest\_jobid 	  (destination internal jobid)
  \item node		  (worker node where the executable is run)
  \item ns                (Network server handling the job)   
  \item nsubjobs	  (number of subjobs)
  \item local\_jobid  	  (new jobId assigned by the receiving component)
  \item queue             (destination queue name)
  \item status\_code	  (way of job termination/classification of the cancel)
\end{itemize}

\smallskip

Lastly if the command is issued with verbosity level 3, further information mostly consisting in the 
job description within the WMS component that has logged the event, is printed to the user:

\begin{itemize}
  \item jdl         	(job description)
  \item job		(job description in receiver language)
  \item descr		(description of current job transformation  -- output of helper)
  \item classad		(checkpoint state value)
  \item seqcode		(sequence code assigned to the event)
  \item level		(logging level -- system, debug, ...)
\end{itemize}

\smallskip

Data on several jobs can be queried by specifying a list of job identifiers separated by a blank space 
as arguments of the command. Moreover the --input option permits to specify a file (file\_path) which 
contains the jobIds whose information are requested. The format of the file must be as follows: one jobId 
for each line and comment lines have to begin with a "\#" or a "\*" character. When using this option the 
user is interrogated for choosing among all, one or a subset of the listed job identifiers. 
If the file\_path does not represent an absolute path, it will be searched in the current working directory.

Each event logged in the LB has an associated log level according to "Universal Format for Logger Messages" 
(see draft-abela-ulm-05.txt available at \url{http://www-didc.lbl.gov/NetLogger/draft-abela-ulm-05.txt}). 
Default value for the log level used by WMS components is System, anyway there could be special situations 
in which problems investigation is needed and additional events are logged with the Debug log level. 
The --output option can be used to have the retrieved information written in the file identified by 
file\_path instead of the standard output. file\_path can be either a simple name or an absolute path 
(on the submitting machine). In the former case the file file\_path is created in the current working 
directory.

If the user wants to use his "private" configuration file this could be done using option --config file\_path. 

\medskip
OPTIONS 
\smallskip

{\bf --help} 

displays command usage.

\smallskip 
\textbf{ --version}
displays WMS-UI version.

\smallskip 
\textbf{ --input file\_path}

{\bf -i} file\_path

retrieves logging info for all Ids contained in the file\_path. This option can't be used  if one or more 
jobIds have been specified. 

\smallskip 
\textbf{ --verbosity} verb\_level

{\bf --v} verb\_level 

sets the detail level of information about the job displayed to the user. Possible values for verb\_level are 0,1 
and 2.

\smallskip 
\textbf{ --config} file\_path

{\bf -c} file\_path

if the command is launched with this option, the configuration file pointed to by file\_path is used instead 
of the standard configuration file.

\smallskip 
\textbf{ --output} file\_path

{\bf -o} file\_path

writes the logging information in the file specified by file\_path instead of the standard output. file\_path 
can be either a simple name or an absolute path (on the submitting machine). In the former case the file 
file\_path is created in the current working directory.

\smallskip 
\textbf{ --noint}

if this option is specified every interactive question to the user is skipped. All warning messages and 
errors (if occurred) are written to the file glite-job-logging\_$<$UID$>$\_$<$PID$>$\_$<$timestamp$>$.log 
under the /tmp directory. Location for log file is configurable.

\smallskip 
\textbf{ --debug}

when this option is specified, information about the API functions called inside the command are displayed on 
the standard output and are written to the file glite-job-logging\_$<$UID$>$\_$<$PID$>$\_$<$timestamp$>$.log 
under the /tmp directory too.  Location for log file is configurable.

\smallskip 
\textbf{ --logfile} file\_path

when this option is specified, the command log file is relocated to the location pointed by file\_path

\smallskip 
\textbf{ jobId}

job/DAG identifier returned by glite-job-submit. Job identifiers must always be provided as last arguments for 
this command.


\medskip
EXIT STATUS \smallskip


{\bf glite-job-get-logging-info} exits with a value of 0 if the status of all the specified jobs is retrieved 
correctly, $>$0 if errors occurred for each specified job and $<$0 in case of partial failure. An example of 
partial failure is when more then one job is specified: some job's logging info could be successfully retrieved 
and some others could be not retrieved. 

\medskip
EXAMPLES 
\smallskip


1.{\scriptsize{\verb!> glite-job-logging-info https://ibm139.cnaf.infn.it:9000/GMUJtnNqe6Lq7w7MfOzeQw -output mylog.txt !}}

writes in file mylog.txt in the current working directory logging information about the job identified by https://ibm139.cnaf.infn.it:9000/GMUJtnNqe6Lq7w7MfOzeQw.

2.{\scriptsize{\verb!> glite-job-logging-info -v 1 -input $HOME/myIds.txt!}}

where \$HOME/myjobs.txt contains two job identifiers, displays the following output

\begin{scriptsize}
\begin{verbatim}
1 : https://ibm139.cnaf.infn.it:9000/D4S_i25ffAsPnKB3iCqeaA
2 : https://ibm139.cnaf.infn.it:9000/2qzyCbPWr7pDY3rNh9PuXA
a : all
q : quit

Choose one or more glite_jobId(s) in the list - [1-2]all: 2


************************************************************
LOGGING INFORMATION:

Printing info for the Job : https://ibm139.cnaf.infn.it:9000/2qzyCbPWr7pDY3rNh9PuXA
 
        ---
 Event: RegJob
- source             =    UserInterface
- timestamp          =    Wed May 14 10:55:35 2003
        ---
 Event: Transfer
- destination        =    NetworkServer
- result             =    START
- source             =    UserInterface
- timestamp          =    Wed May 14 10:55:36 2003
        ---
Event: Transfer
- destination         =    NetworkServer
- result              =    OK
- source              =    UserInterface
- timestamp           =    Wed May 14 10:55:44 2003
        ---
 Event: Accepted
- source              =    NetworkServer
- timestamp           =    Wed May 14 10:56:42 2003
        ---
 Event: EnQueued
- result              =    OK
- source              =    NetworkServer
- timestamp           =    Wed May 14 10:56:45 2003
**********************************************************************
...
...

\end{verbatim} 
\end{scriptsize}   
\smallskip

See also glite-job-submit~\ref{submit}

\newpage

\subsubsection{glite-job-attach}
\label{attach}

This commands starts an interactive session for a previously submitted interactive job.
\medskip
SYNOPSIS 
\smallskip
\textbf{glite-job-attach  [options]  $<$job Id$>$}
\begin{scriptsize}
\begin{verbatim}

Options:
   --help          
   --version                     
   --port, -p      <port_num>
   --nogui
   --nolisten
   --config, -c    <file_path>
   --input, -i 	   <file_path>
   --noint
   --debug         
   --logfile       <file_path>

\end{verbatim} 
\end{scriptsize}


\medskip
DESCRIPTION 
\smallskip

This command starts a listener process on the WMS-UI machine (glite-wms-grid-console-shadow) that allows attaching to the 
standard streams of a previously submitted interactive job and displays them on a dedicated window. 

As the command opens a X window, the user should make sure the DISPLAY environment variable is correctly set, 
a X server is running on the local machine and if she/he is connected to the WMS-UI node from remote machine (e.g. 
with ssh) enable secure X11 tunneling.

The listener process and the window are started automatically by the glite-job-submit command for interactive 
jobs, so this command can be used for example in case a problem occurred on the WMS-UI machine that made the 
interactive session be lost or in case the user needs to follow the job from another machine or another port 
on the same machine --port option).

This command can only be invoked for interactive jobs.

\medskip
OPTIONS 
\smallskip

\textbf{ --help} 

displays command usage.

\smallskip 
\textbf{ --version}

displays WMS-UI version.

\smallskip 
\textbf{ --port port\_num}

{\bf -p} port\_num

make sthe command start a listener on the local machine on the specified port and logs these information to 
the LB associated to the job.

\smallskip 
\textbf{ --nogui} 

As the glite-job-attach command opens a X window, the user should make sure a X server is running on the 
local machine and if she/he is connected to the WMS-UI node from remote machine (e.g. with ssh) enable secure X11 
tunneling. If this is not possible, the user can specify the --nogui option that makes the command provide 
a simple standard non-graphical interaction with the running job.

\smallskip 
\textbf{ --nolisten}

This option makes the command forward the job standard streams coming from the WN to named pipes on the WMS-UI 
machine whose names are returned to the user together with the OS id of the listener process.  This allows the 
user to interact with the job through her/his own tools. It is important to note that when this option is 
specified, the WMS-UI has no more control over the launched listener process that has to be killed by the 
user (using the process id returned by the command) once the job is finished.

\smallskip 
\textbf{ --config file\_path}

{\bf -c} file\_path

if the command is launched with this option, the configuration file pointed to by file\_path is used instead 
of the standard configuration file.

\smallskip 
\textbf{ --input file\_path}

{\bf -i} file\_path

allows the user to attach to one (just one) of the jobIds contained in the file\_path. This option can't be 
used if one jobIds has been specified.

\smallskip 
\textbf{ --noint}

if this option is specified every interactive question to the user is skipped. All warning messages and 
errors (if occurred) are written to the file glite-job-attach\_$<$UID$>$\_$<$PID$>$\_$<$timestamp$>$.log 
under the /tmp directory. Location for log file is configurable.

{\bf --debug}

when this option is specified, information about the API functions called inside the command are displayed 
on the standard output and are written to the file glite-job-attach\_$<$UID$>$\_$<$PID$>$\_$<$timestamp$>$.log 
under the /tmp directory too.  Location for log file is configurable.

\smallskip 
\textbf{ --logfile} file\_path

when this option is specified, the command log file is relocated to the location pointed by file\_path

jobId

job identifier returned by glite-job-submit. Job identifiers must always be provided as last arguments for 
this command.

\medskip
EXIT STATUS 
\smallskip


{\bf glite-job-attach} exits with a value of 0 on success and $>$0 on failure.

\medskip
EXAMPLES 
\smallskip

{\scriptsize{\verb!> glite-job-attach https://ibm139.cnaf.infn.it:9000/t3KwW8qhXhkYs-ZfNCFidg!}}

displays the following information message:

\begin{scriptsize}
\begin{verbatim}

********************************************************
 JOB ATTACHED:
 The Interactive Session Listener has been successfully 
 launched with the following parameters:
 
 Host:                      193.111.146.90
 Port:                       40713
 Pid:                        18575
********************************************************

\end{verbatim}
\end{scriptsize}

and opens a window allowing interaction with the job through the standard streams.

\newpage


\subsubsection{glite-job-get-chkpt}
\label{chkpt}

This commands retrieves checkpoint states saved by a previously submitted checkpointable job.

\medskip
 SYNOPSIS 
\smallskip
\textbf{glite-job-get-chkpt  [options]  $<$job Id$>$}
\begin{scriptsize}
\begin{verbatim}

Options:
    --help          
    --version       
    --cs            <state_num>
    --config, -c    <file_path>
    --input, -i     <file_path>
    --output, -o    <file_path>
    --noint         
    --debug         
    --logfile       <file_path>

\end{verbatim}
\end{scriptsize}

\medskip
 DESCRIPTION \smallskip

This commands allows the user to retrieve one or more checkpoint states saved by a previously submitted job. 
Checkpoint states are retrieved from the LB server and are saved locally into a file in JDL format. 
The --cs option allows the user to select the checkpoint state she/he wants to be retrieved. Indeed specifying 
the command with "--cs N"  makes the command retrieve the last but N job checkpoint state. Last saved state is 
retrieved otherwise.
The retrieved state is saved in a file in JDL format. The output file path can be set through the --output 
option of the command.
This command can be used only for checkpointable jobs.

\medskip
OPTIONS 
\smallskip

\textbf{ --help} 

displays command usage.

\smallskip 
\textbf{ --version}

displays WMS-UI version.

\smallskip 
\textbf{ --config} file\_path

{\bf -c} file\_path

if the command is launched with this option, the configuration file pointed to by file\_path is used instead 
of the standard configuration file.

\smallskip 
\textbf{ --cs} state\_num

if the command is launched with this option then it retrieves the "last but state\_num" state saved by the job. 
Last saved state is returned if the option is not used (equivalent to state\_num = 0).

\smallskip 
\textbf{ --input} file\_path

{\bf -i} file\_path

allows the user to select one (just one) of the jobIds contained in the file\_path for retrieval of the 
saved checkpoint state. This option can't be used if one jobIds has been specified. 

\smallskip 
\textbf{ --output} file\_path

{\bf -o} file\_path

saves the retrieved state in the file specified by file\_path. file\_path can be either a simple name or 
an absolute path (on the submitting machine). In the former case the file file\_path is created in the current 
working directory. If this option is not used the retrieved state is displayed on the standard output.

\smallskip 
\textbf{ --noint}

if this option is specified every interactive question to the user is skipped. All warning messages and errors 
(if occurred) are written to the file glite-job-get-chkpt\_$<$UID$>$\_$<$PID$>$\_$<$timestamp$>$.log under the 
/tmp directory. Location for log file is configurable

\smallskip 
\textbf{ --debug}

when this option is specified, information about the API functions called inside the command are displayed on 
the standard output and are written to the file glite-job-attach\_$<$UID$>$\_$<$PID$>$\_$<$timestamp$>$.log under 
the /tmp directory too.  Location for log file is configurable.

\smallskip 
\textbf{ --logfile} file\_path

when this option is specified, the command log file is relocated to the location pointed by file\_path

\smallskip 
\textbf{ jobId}

job identifier returned by glite-job-submit. Job identifiers must always be provided as last arguments for this 
command.

\medskip
EXIT STATUS
\smallskip

{\bf glite-job-get-chkpt} exits with a value of 0 on success and $>$0 on failure.

\medskip
EXAMPLES 
\smallskip

The following commands retrieve the last but 3 saved checkpint state of the job and saves it in the file 
specified by the user�:


\begin{scriptsize}
\begin{verbatim}

> glite-job-get-chkpt -o myjob.chk -cs 3 
https://ibm139.cnaf.infn.it:9000/LNn4rOX17LL30e34hSqGjQ
======================= glite-job-get-chkpt Success ==================
 The checkpointable Job state has been successfully retrieved from LB 
 Server and stored in the file: /home/fpacini/CLI/bin/myjob.chk
======================================================================

> more /home/fpacini/CLI/bin/myjob.chk

#      Job State Retrieved for 
#jobId: https://ibm139.cnaf.infn.it:9000/LNn4rOX17LL30e34hSqGjQ
 [
  UserData = 
   [
    distribution = false; 
    hsum_filename = 
	"gsiftp://lxde01.pd.infn.it/tmp/root_test/hsum_lxde04_1200000.root"; 
    first_event = 1200001
   ];

\end{verbatim}
\end{scriptsize}


