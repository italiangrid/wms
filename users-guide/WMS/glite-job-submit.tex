% PLEASE DO NOT MODIFY THIS FILE! It was generated by raskman version: 1.1.0
\subsubsection{glite-job-submit}
\label{glite-job-submit}

\medskip
\textbf{glite-job-submit}
\smallskip


\medskip
\textbf{SYNOPSIS}
\smallskip

\textbf{glite-job-submit [options]  $<$jdl\_file$>$}
{\begin{verbatim}

options:
	--version
	--help
	--config, -c <configfile>
	--debug
	--logfile <filepath>
	--noint
	--input, -i <filepath>
	--output, -o <filepath>
	--resource, -r <ceid>
	--nodes-resource <ceid>
	--nolisten
	--nogui
	--nomsg
	--chkpt <filepath>
	--lrms <lrmstype>
	--valid, -v <hh:mm>
	--config-vo <configfile>
	--vo <voname>
\end{verbatim}

\medskip
\textbf{DESCRIPTION}
\smallskip


glite-job-submit is the command for submitting jobs to the DataGrid and hence allows the user to run a job at one or several remote resources. glite-job-submit requires as input a job description file in which job characteristics and requirements are expressed by means of Condor class-ad-like expressions.
While it does not matter the order of the other arguments, the job description file has to be the last argument of
this command.

\medskip
\textbf{OPTIONS}
\smallskip

\textbf{--version}

displays UI version.

\textbf{--help}

displays command usage

\textbf{--config}, \textbf{-c} <configfile>

if the command is launched with this option, the configuration file pointed by configfile is used. This option is meaningless when used together with "--vo" option

\textbf{--debug}

When this option is specified, debugging information is displayed on the standard output and written into the log file, whose location is eventually printed on screen.
The default UI logfile location is:
glite-wms-job-<command\_name>\_<uid>\_<pid>\_<time>.log  located under the /var/tmp directory
please notice that this path can be overriden with the '--logfile' option

\textbf{--logfile} <filepath>

when this option is specified, all information is written into the specified file pointed by filepath.
This option will override the default location of the logfile:
glite-wms-job-<command\_name>\_<uid>\_<pid>\_<time>.log  located under the /var/tmp directory

\textbf{--noint}

if this option is specified, every interactive question to the user is skipped and the operation is continued (when possible)

\textbf{--input}, \textbf{-i} <filepath>

if this option is specified, the user will be asked to choose a CEId from a list of CEs contained in the filepath. Once a CEId has been selected the command behaves as explained for the resource option. If this option is used together with the --int one and the input file contains more than one CEId, then the first CEId in the list is taken into account for submitting the job.

\textbf{--output}, \textbf{-o} <filepath>

writes the generated jobId assigned to the submitted job in the file specified by filepath,which can be either a simple name or an absolute path (on the submitting machine). In the former case the file is created in the current working directory.

\textbf{--resource}, \textbf{-r} <ceid>

This command is available only for jobs.
if this option is specified, the job-ad sent to the NS contains a line of the type "SubmitTo = <ceid>"  and the job is submitted by the WMS to the resource identified by <ceid> without going through the match-making process.

\textbf{--nodes-resource} <ceid>

This command is available only for dags.
if this option is specified, the job-ad sent to the NS contains a line of the type "SubmitTo = <ceid>"  and the dag is submitted by the WMS to the resource identified by <ceid> without going through the match-making process for each of its nodes.

\textbf{--nolisten}

This option can be used only for interactive jobs. It makes the command forward the job standard streams coming from the WN to named pipes on the client machine whose names are returned to the user together with the OS id of the listener process. This allows the user to interact with the job through her/his own tools. It is important to note that when this option is specified, the command has no more control over the launched listener process that has hence to be killed by the user (through the returned process id) once the job is finished.

\textbf{--nogui}

This option can be used only for interactive jobs. As the command for such jobs opens an X window, the user should make sure that an X server is running on the local machine and if she/he is connected to the UI node from a remote machine (e.g. with ssh) enable secure X11 tunneling.
If this is not possible, the user can specify the --nogui option that makes the command provide a simple standard non-graphical interaction with the running job.

\textbf{--nomsg}

this option makes the command print on the standard output only the jobId generated for the job when submission was successful.The location of the log file containing massages and diagnostics is printed otherwise.

\textbf{--chkpt} <filepath>

This option can be used only for checkpointable jobs. The state specified as input is a checkpoint state generated by a previously submitted job.  This option makes the submitted job start running from the checkpoint state given in input and not from the very beginning.
The initial checkpoint states to be used with this option can be retrieved by means of the glite-job-get-chkpt command.

\textbf{--lrms} <lrmstype>

This option is only for MPICH  jobs and must be used together with either --resource or --input option; it specifies the type of the lrms of the resource the user is submitting to. When the batch system type of the specified CE resource given is not known, the lrms must be provided while submitting. For non-MPICH jobs this option will be ignored.

\textbf{--valid}, \textbf{-v} <hh:mm>

A job for which no compatible CEs have been found during the matchmaking phase is hold in the WMS Task Queue for a certain time so that it can be subjected again to matchmaking from time to time until a compatible CE is found. The JDL ExpiryTime attribute is an integer representing the date and time (in seconds since epoch)until the job request has to be considered valid by the WMS. This option allows to specify the validity in hours and minutes from submission time of the submitted JDL. When this option is used the command sets the value for the ExpiryTime attribute converting appropriately the relative timestamp provided as input. It overrides, if present,the current value. If the specified value exceeds one day from job submission then it is not taken into account by the WMS.

\textbf{--config-vo} <configfile>

if the command is launched with this option, the VO-specific configuration file pointed by configfile is used. This option is meaningless when used together with "--vo" option

\textbf{--vo} <voname>

this option allows the user to specify the name of the Virtual Organisation she/he is currently working for.
If the user proxy contains VOMS extensions then the VO specified through this option is overridden by the default VO contained in the proxy (i.e. this option is only useful when working with non-VOMS proxies).
This option is meaningless when used together with "--config-vo" option


\medskip
\textbf{EXAMPLES}
\smallskip


Upon successful submissions, this command returns to the identifier (JobId) assigned to the job

- saves the returned JobId in a file:
glite-job-submit --output jobid.out ./job.jdl

- forces the submission to the resource specified with the -r option:
glite-job-submit -r lxb1111.glite.it:2119/blah-lsf-jra1\_low ./job.jdl

- forces the submission of the DAG (the parent and all child nodes) to the resource specified with the --nodes-resources option:
glite-job-submit --nodes-resources lxb1111.glite.it:2119/blah-lsf-jra1\_low ./dag.jdl

\medskip
\textbf{ENVIRONMENT}
\smallskip


GLITE\_WMSUI\_CONFIG\_VAR:  This variable may be set to specify the path location of the custom default attribute configuration

GLITE\_WMSUI\_CONFIG\_VO: This variable may be set to specify the path location of the VO-specific configuration file

GLITE\_WMS\_LOCATION:  This variable must be set when the Glite WMS installation is not located in the default paths: either /opt/glite or /usr/local

GLITE\_LOCATION: This variable must be set when the Glite installation is not located in the default paths: either /opt/glite or /usr/local


GLOBUS\_LOCATION: This variable must be set when the Globus installation is not located in the default path /opt/globus.
It is taken into account only by submission and get-output commands

GLOBUS\_TCP\_PORT\_RANGE="<val min> <val max>" This variable must be set to define a range of ports to be used for inbound connections in the interactivity context.
It is taken into account only by submission of interactive jobs and attach commands

X509\_CERT\_DIR: This variable may be set to override the default location of the trusted certificates directory, which is normally /etc/grid-security/certificates.

X509\_USER\_PROXY: This variable may be set to override the default location of the user proxy credentials, which is normally /tmp/x509up\_u<uid>.

\medskip
\textbf{FILES}
\smallskip


One of the following paths must exist (seeked with the specified order):
- \$GLITE\_WMS\_LOCATION/etc/
- \$GLITE\_LOCATION/etc/
- /opt/glite/etc/
- /usr/local/etc/
- /etc/

and contain the following UI configuration files:
glite\_wmsui\_cmd\_var.conf, glite\_wmsui\_cmd\_err.conf, glite\_wmsui\_cmd\_help.conf, <voName>/glite\_wmsui.conf

- glite\_wmsui\_cmd\_var.conf will contain custom configuration default values
A different configuration file may be specified either by using the --config option or by setting the GLITE\_WMSUI\_CONFIG\_VAR environment variable
here follows a possible example:
[
RetryCount = 3 ;
ErrorStorage= "/tmp" ;
OutputStorage="/tmp";
ListenerStorage = "/tmp" ;
LoggingTimeout = 30 ;
LoggingSyncTimeout = 30 ;
NSLoggerLevel = 0;
DefaultStatusLevel = 1 ;
DefaultLogInfoLevel = 1;
]

- glite\_wmsui\_cmd\_err.conf will contain UI exception mapping between error codes and error messages (no relocation possible)

- glite\_wmsui\_cmd\_help.conf will contain UI long-help information (no relocation possible)

- <voName>/glite\_wmsui.conf  will contain User VO-specific attributes.
A different configuration file may be specified either by using the --config-vo option or by setting the GLITE\_WMSUI\_CONFIG\_VO environment variable
here follows a possible example:
[
LBAddresses = { "tigerman.cnaf.infn.it:9000" };
VirtualOrganisation = "egee";
NSAddresses = { "tigerman.cnaf.infn.it:7772" }
]

Besides those files, a valid proxy must be found inside the following path:
/tmp/x509up\_u<uid> ( use the X509\_USER\_PROXY environment variable to override the default location JDL file)

\medskip
\textbf{AUTHORS}
\smallskip


Alessandro Maraschini (egee@datamat.it)

