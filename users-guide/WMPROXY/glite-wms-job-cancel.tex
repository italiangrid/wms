
\subsection{glite-wms-job-cancel}
\label{glite-wms-job-cancel}

\medskip
\textbf{glite-wms-job-cancel}
\smallskip


\textbf{SYNOPSIS}
\smallskip

\textbf{glite-wms-job-cancel [options]  <jobID1> <jobID2> ... }

\begin{verbatim}
options:
        --version
        --help
        --confab, -c <configfile>
        --vo <voname>
        --debug
        --logfile    <filepath>
        --noint
        --input, -i  <filepath>
        --output, -o <filepath>
        --json
        --pretty-print
\end{verbatim}

\medskip
\textbf{DESCRIPTION}
\smallskip

glite-wms-job-cancel cancels a job previously submitted using glite-wms-job-submit. 
Before cancellation, it prompts the user for confirmation. 
The cancel request is sent to the WMProxy that forwards it to the WM that fulfills it. 
glite-wms-job-cancel can remove one or more jobs: the jobs to be removed are 
identified by their job identifiers (jobIds returned by glite-wms-job-submit) 
provided as arguments to the command and separated by a blank space. 
The result of the cancel operation is reported to the user for each specified jobId.


\medskip\textbf{OPTIONS}\smallskip



\textbf{-{}-version}

displays WMProxy client and servers version numbers.
If the -{}-endpoint option is specified only this service is contacted to retrieve its version. If it is not used, the endpoint contacted is that one specified with GLITE\_WMS\_WMPROXY\_ENDPOINT environment variable. If the option is not specified and the environment variable is not set, all servers listed in the WMProxyEndPoints attribute of the user configuration file are contacted. For each of these service, the returned result will be displayed on the standard output (the version numbers in case of success; otherwise the error message).




\textbf{-{}-help}

displays command usage




\textbf{-{}-config, -c <configfile>}

if the command is launched with this option, the configuration file pointed by <configfile> is used. This option is meaningless when used together with -{}-vo option




\textbf{-{}-vo <voname>}

this option allows the user to specify the Virtual Organization she/he is currently working for.
If the user proxy contains VOMS extensions then the VO specified through this option is overridden by the
default VO contained in the proxy (i.e. this option is only useful when working with non-VOMS proxies).
This option is meaningless when used together with -{}-config-vo option.




\textbf{-{}-debug}

when this option is specified, debugging information is displayed on the standard output and written also either into the default log file:


\begin{verbatim}
glite-wms-job-<command_name>_<uid>_<pid>_<time>.log
\end{verbatim}

located under the /var/tmp  directory or in the log file specified with -{}-logfile option.




\textbf{-{}-logfile <filepath>}

When this option is specified, a command log file (whose pathname is <filepath> ) is created.



\textbf{-{}-noint  }

if this option is specified, every interactive question to the user is skipped and the 
operation is continued (when possible)




\textbf{-{}-input, -i <filepath>}

cancels jobs whose jobIds are contained in the file located in filepath. 
This option cannot be used if one or more jobIds have been specified. 
If the -{}-noint is not specified, the list of jobIds contained in the file is displayed and the 
user is prompted for a choice. Single jobs can be selected specifying the numbers associated to 
the job identifiers separated by commas. E.g. selects the first, the third and the fifth jobId in 
the list.
Ranges can also be selected specifying ends separated by a dash. E.g. selects jobIds in the 
list from third position (included) to sixth position (included).
It is worth mentioning that it is possible to select at the same time ranges and single jobs. 
E.g. selects the first job id in the list, the ids from the third to the fifth (ends included) 
and finally the eighth one.




\textbf{-{}-output, -o <filepath>}

writes the cancellation results in the file specified by <filepath> instead of the standard output. 
<filepath> can be either a simple name or an absolute path (on the submitting machine). 
In the former case the file <filepath> is created in the current working directory.





\textbf{-{}-json}

This option makes the command produce its output in JSON-compliant format, that can be 
parsed by proper json libraries for python/perl and other script languages. 
Please note that -{}-json and -{}-output options are mutually exclusive.





\textbf{-{}-pretty-print}

This option should be used with -{}-json. Without it the JSON format is machine-oriented 
(no carriage returns, no indentations). -{}-pretty-print makes the JSON output easily readable 
by a human. Using this option without -{}-json has no effect.





\medskip
\textbf{EXAMPLES}
\smallskip


\begin{enumerate}


\item  request for canceling only one job:
\begin{verbatim}
glite-wms-job-cancel https://wmproxy.glite.it:9000/7O0j4Fequpg7M6SRJ-NvLg
\end{verbatim}

\item  request for canceling multiple jobs:
\begin{verbatim}
glite-wms-job-cancel https://wmproxy.glite.it:9000/7O0j4Fequpg7M6SRJ-NvLg \
                     https://wmproxy.glite.it:9000/wqikja_-de83jdqd \
                     https://wmproxy.glite.it:9000/wugrfja-de91jxxd 
\end{verbatim}

\item  the myids.in input file contains the jobid(s)
\begin{verbatim}
glite-wms-job-cancel --input myids.in
\end{verbatim}

\end{enumerate}

A message with the result of the operation is displayed on the standard output


\medskip
\textbf{FILES}
\smallskip

\verb voName/glite_wms.conf {}: The user configuration file. The standard 
path location is \verb /etc/glite-wms . 

\verb /tmp/x509up_u<uid> {}: A valid X509 user proxy; 
use the X509\_USER\_PROXY environment variable to override the default location



\medskip
\textbf{ENVIRONMENT}
\smallskip


\begin{itemize}



\item GLITE\_WMS\_WMPROXY\_ENDPOINT: This variable may be set to specify the endpoint URL


\item GLOBUS\_LOCATION: This variable must be set when the Globus installation is not located in the default path \verb /opt/globus {}.


\item GLOBUS\_TCP\_PORT\_RANGE="<val min> <val max>": This variable must be set to define a range of ports to be used for inbound connections in the interactivity context


\item X509\_CERT\_DIR: This variable may be set to override the default location of the trusted certificates directory, which is normally \verb /etc/grid-security/certificates {}.


\item X509\_USER\_PROXY: This variable may be set to override the default location of the user proxy credentials, which is normally \verb /tmp/x509up_u<uid> {}.


\item GLITE\_SD\_PLUGIN: If Service Discovery querying is needed, this variable can be used in order to set a specific (or more) plugin, normally bdii, rgma (or both, separated by comma)


\item LCG\_GFAL\_INFOSYS: If Service Discovery querying is needed, this variable can be used in order to set a specific Server where to perform the queries: for instance LCG\_GFAL\_INFOSYS='gridit-bdii-01.cnaf.infn.it:2170'


\end{itemize}

