%\section{Quickstart Guide}

This section briefly explains the sequence of operations,
to be performed by a user in order to submit, cancel, retrieve output etc.
of any kind of request through the WMProxy C++ command line interface.


\subsection{Before starting: get your user proxy}


Before using any of the WMProxy client commands it is necessary to have
a valid proxy credential available on the client machine. 
You can create it using the voms-proxy-init command or 
alternatively the grid-proxy-init one.
If you already have a valid proxy available on your machine just make the
X509\_USER\_PROXY environment variable point to it.
                                                                                
In order to get a proxy certificates issued by VOMS you
should have in your home directory the file
\$HOME/.glite/vomses containing a line as follows:

\smallskip
{\scriptsize{\verb!"EGEE" "kuiken.nikhef.nl" "15001" "/O=dutchgrid/O=hosts/OU=nikhef.nl/CN=kuiken.nikhef.nl" "EGEE" "22"!}}
\smallskip
                                                                                
or the corresponding line for your VO (ask your VO admin for that).
                                                                                
Make moreover sure you have in the directory \$HOME/.globus your certificate/key pair, 
i.e. the following files:
                                                                                
\smallskip
                                                                                
\begin{scriptsize}
\begin{verbatim}
usercert.pem
userkey.pem
\end{verbatim}
\end{scriptsize}
\smallskip
                                                                                
Note that file permissions are important: the two files must have respectively 
0600 and 0400 permissions.

Then you can issue the VOMS client command (you will be prompted for the pass-phrase):

\smallskip
\begin{scriptsize}
\begin{verbatim}
                                                                                 
> voms-proxy-init -userconf vomses -voms EGEE
Your identity: /C=IT/O=INFN/OU=Personal Certificate/L=Padova/CN=Massimo Sgaravatto/Email=massimo.sgaravatto@pd.infn.it
Enter GRID pass phrase:
Creating temporary proxy .................................... Done
Contacting  kuiken.nikhef.nl:15001 [/O=dutchgrid/O=hosts/OU=nikhef.nl/CN=kuiken.nikhef.nl] ``EGEE'' Done
Creating proxy ............................................. Done
Your proxy is valid until Thu Jul 28 23:11:01 2005
                                                                                 
> voms-proxy-info
subject   : /C=IT/O=INFN/OU=Personal Certificate/L=Padova/CN=Massimo Sgaravatto/Email=massimo.sgaravatto@pd.infn.it/CN=proxy
issuer    : /C=IT/O=INFN/OU=Personal Certificate/L=Padova/CN=Massimo Sgaravatto/Email=massimo.sgaravatto@pd.infn.it
identity  : /C=IT/O=INFN/OU=Personal Certificate/L=Padova/CN=Massimo Sgaravatto/Email=massimo.sgaravatto@pd.infn.it
type      : proxy
strength  : 512 bits
path      : /tmp/x509up_u756
timeleft  : 11:59:25
\end{verbatim}
\end{scriptsize}

\medskip


\subsection{WMProxy client commands}
                                                                                   
The commands to interact with WMProxy Service are:
                                                                                   
\begin{itemize}
   \item \begin{verbatim}glite-wms-job-submit <jdlfile>\end{verbatim}
   \item \begin{verbatim}glite-wms-delegate-proxy <delegationId>\end{verbatim}
   \item \begin{verbatim}glite-wms-job-list-match <jdlfile>\end{verbatim}
   \item \begin{verbatim}glite-wms-job-cancel <jobIds>\end{verbatim}
   \item \begin{verbatim}glite-wms-job-output <jobIds>\end{verbatim}
   \item \begin{verbatim}glite-wms-job-perusal <jobId>\end{verbatim}
   \item \begin{verbatim}glite-wms-job-info <jobId> \end{verbatim}  
\end{itemize}


%Before using any command you should assure one of the following paths points to the WMProxy installation path:
%(i.e. the path containing etc and bin directories)
%
% - GLITE\_WMS\_LOCATION env variable
%
% - GLITE\_LOCATION env variable
%
% - /opt/glite path  (as comes from default installed rpms)
%
% - /usr/local path



You can then access information about the usage of each command by issuing:

\smallskip

\begin{verbatim}
<command> --help
\end{verbatim}

\medskip

\textbf{glite-wms-job-submit}
submits a job to a WMProxy Service. It requires a JDL file
as input and returns a WMS job identifier.

\smallskip

\textbf{glite-wms-delegate-proxy}
allows the user to delegate her proxy credential to the WMProxy service.
This delegated credential can then be used for job submissions.

\smallskip

\textbf{glite-wms-job-list-match}
displays the list of identifiers of the resources on which the user issuing the command is authorized and
satisfying the job requirements

\smallskip

\textbf{glite-wms-job-cancel}
cancels one or more jobs previously submitted to WMProxy Service.

\smallskip

\textbf{glite-wms-job-output}
retrieves output files of a job, when finished.
After this operation the job itself is purged and no more operation can be done on it


\textbf{glite-wms-job-perusal}
manages the perusal functionality for a certain job such as:
\begin{itemize}
\item Enabling perusal for one (or more) file(s)
\item Allowing perusal file retrieval
\item Disabling perusal functionality
\end{itemize}
\textbf{glite-wms-job-info}
retrieves information about a submitted Job, its delegated credential, the JDL submitted to WMProxy...


\textbf{glite-wms-job-logging-info}
queries the LB persistent database for logging information about jobs previously submitted using glite-wms-job-submit.


\textbf{glite-wms-job-status}
prints the status of a job previously submitted using glite-wms-job-submit.

\bigskip

All these commands, whose use and purpose is briefly described in the following
subsections,
are described in more detail in the following section


\subsection{Configuration file}
Each command needs a configuration file to be properly executed. This file depends on the VirtualOrganisation
name the user is currently working for.

The following precedence rule is followed for determining the user's VO:
\begin{enumerate}
\item the default VO from the user proxy (if it contains VOMS extensions)

\item the VO specified through the -{}-vo  option (supported by all commands)

\item the VO specified  -{}-config  option (supported by all commands)

\item the VO specified in the configuration file pointed by the GLITE\_WMS\_CLIENT\_CONFIG environment variable

\item when provided, the VirtualOrganisation attribute in the JDL (if the user proxy contains VOMS extensions this value is overridden as above)
\end{enumerate}

 If none of them is found, an error is returned and the command is aborted.

 In cases 1), 2) and 6) the path of the user configuration file will be:  \$HOME/.glite/$<$VO name (in lower case)$>$/glite\_wms.conf

 Moreover, the default configuration file path is the following:

 $<$WMProxy installation path$>$/etc/$<$VO name (in lower case)$>$/glite\_wms.conf

 Once the VirtualOrganisation has been properly evaluated, both files (user and default one) are parsed, at least one of them must be found (an error is raised otherwise). If both are found, the attributes in the user configuration files will overrides the attributes with the same name in the default configuration file. The resulting merged configuration instance is therefore used by the requested command


\subsection{Submitting jobs to WMProxy Service}

To submit jobs to WMProxy Service, the command glite-wms-job-submit
must be used. The glite-wms-job-submit command requires as input a
job description file, which describes the job characteristics and requirements
through the JDL (Job Description Language).
                                                                                   
A typical example of a JDL job description file is:


\smallskip
\begin{verbatim}
 [
   Type = "Job";
   JobType = "Normal";
   Executable = "myexe";
   StdInput = "myinput.txt";
   StdOutput = "message.txt";
   StdError = "error.txt";
   InputSandbox = {"myinput.txt", "/home/user/example/myexe"};
   OutputSandbox = {"message.txt", "error.txt"};
 ]
\end{verbatim}
\smallskip

Such a JDL would make the myexe executable be transferred on the
remote WMProxy server and be run taking the myinput.txt file (also copied
from the client node) as input. The standard streams of the
job are redirected to file message.txt and
error.txt .
\medskip

\bigskip



Each job submitted to a WMProxy Service 
is given the delegated credentials of the user who submitted it.
These credentials can
then be used when operations requiring security support have to be
performed by the job.

There are two possible options to deal with proxy delegation:

\begin{itemize}

\item
asking the ``automatic'' delegation of the credentials during
the submission operation;

\item
explicitly delegating credentials, and then
asking to rely on these previously delegated credentials on the actual
submission operations.

\end{itemize}
                                                                             
It is highly suggested 
to rely on this latter mechanism, using the same delegated
proxy for multiple job submissions, instead of delegating each time a proxy.
Delegating a proxy, in fact, is an operation that can require a non
negligible time.

The command glite-wms-job-delegate-proxy (described in
Sect.~\ref{glite-wms-job-delegate-proxy}) is the command to use to
explicitly delegate the user credentials to a WMProxy server.

\medskip

In the following we provide a brief description (including
examples) for each WMProxy client command

The endpoint URL of the WMProxy service to contact is read from the configuration by all commands not
taking a jobId as input. The value in the configuration can be overridden using the -{}-endpoint  option.
The commands taking a jobId as input get the WMProxy endpoint URL from the LB server.



\textbf{glite-wms-job-delegate-proxy}

  This commands allows delegating a user proxy to the WMProxy service. The user can either specify
  the delegationId to be associated with the delegated proxy by using the -{}-delegationid  option:
\begin{scriptsize}
\begin{verbatim}
  [trinity] /home/dorigoa > glite-wms-job-delegate-proxy -d myfirstdlegationID
\end{verbatim}
\end{scriptsize}
  or make the command automatically generate an ID by using the -{}-autm-delegation  option:
\begin{scriptsize}
\begin{verbatim}
  [trinity] /home/dorigoa > glite-wms-job-delegate-proxy -a
\end{verbatim}
\end{scriptsize}
  The delegationId is needed for performing job submission later. In particular glite-wms-job-submit and
  glite-wms-job-list-match are the only commands needing proxy delegation.


\textbf{glite-wms-job-submit}

  This commands allows submitting a job, DAG or collection to the WMProxy. The request to be submitted
  has to be described by a JDL file as specified by document https://edms.cern.ch/document/590869/1.
  Submission can be performed either by using a previously delegated proxy (advised option), e.g.:
\begin{scriptsize}
\begin{verbatim}
  [trinity] /home/dorigoa > glite-wms-job-delegate-proxy -d myfirstdlegationID
\end{verbatim}
\end{scriptsize}
  or delegating a new proxy at each new submission, e.g.:
\begin{scriptsize}
\begin{verbatim}
  [trinity] /home/dorigoa > glite-wms-job-submit --autm-delegation -o myids JDL/mydag.jdl
\end{verbatim}
\end{scriptsize}
  To submit complex requests such as DAGs and Collections there is no change in the command syntax; it 
  suffices passing a JDL file describing such requests to the job as shown above.

  For job collections there is a further way to perform submission. The -{}-collection  option allows indeed to
  specify the path to a directory containing all the single JDL files of the jobs you want the collection is 
  made of. The glite-wms-job-submit will generate the corresponding collection JDL and submit it. E.g.:
\begin{scriptsize}
\begin{verbatim}
  [trinity] /home/dorigoa > glite-wms-job-submit -d myfirstdlegationID -o myids --collection /home/dorigoa/KOLL/
\end{verbatim}
\end{scriptsize}
  where /home/dorigoa/KOLL/ is the directory containing all the JDL files.

  It is important to note that files that are shared from the sandboxes of different jobs will be transferred 
  only once to the WMS node.

  In all cases, upon success, the command returns an ID to the user that can be used as the handle to follow 
  up the progress of the request itself.


\textbf{glite-wms-job-list-match}

  This command provides the list of CEs matching the requirements of the job specified in the JDL. If the 
  -{}-rank  option is specified it also prints the rank corresponding to each listed CE. E.g.:
\begin{scriptsize}
\begin{verbatim}
  [trinity] /home/dorigoa > glite-wms-job-list-match -d myfirstdlegationID --rank JDL/tt.jdl 

    Connecting to the service https://tigerman.cnaf.infn.it:7443/glite_wms_wmproxy_server

    ==========================================================================
                         COMPUTING ELEMENT IDs LIST 
    The following CE(s) matching your job requirements have been found:

           *CEId*                                    *Rank*
    - atlfarm006.mi.infn.it:2119/blah-pbs-infinite      0
    - atlfarm006.mi.infn.it:2119/blah-pbs-long          0
    - atlfarm006.mi.infn.it:2119/blah-pbs-short         0
    - lxb2077.cern.ch:2119/blah-pbs-short               0
    - lxde01.pd.infn.it:2119/blah-lsf-grid01            0
    - lxde01.pd.infn.it:2119/blah-lsf-grid03            0
    - lxde01.pd.infn.it:2119/blah-lsf-grid04            0
    - lxb2077.cern.ch:2119/blah-pbs-infinite            -624999
    - lxb2077.cern.ch:2119/blah-pbs-long                -624999
    ==========================================================================
\end{verbatim}
\end{scriptsize}
  This command does not support complex requests such as DAGs and collections.



\textbf{glite-wms-job-cancel}

  This command allows canceling a previously submitted job. It takes as input a request ID (or job ID), e.g.:
\begin{scriptsize}
\begin{verbatim}
  [trinity] /home/dorigoa > glite-wms-job-cancel  https://tigerman.cnaf.infn.it:9000/K4Tsz4jkqiAqZ1fcWU-pBQ
\end{verbatim}
\end{scriptsize}
  or a list of request IDs either on the command line or from an input file, e.g.:
\begin{scriptsize}
\begin{verbatim}
  [trinity] /home/dorigoa > glite-wms-job-cancel -i myids
\end{verbatim}
\end{scriptsize}
  It is important to note that nodes of a DAG cannot be canceled singularly as this could break dependencies.


\textbf{glite-wms-job-output}

  This command performs the retrieval of the output sandbox files of a previously submitted request. Of course 
  file retrieval is only allowed when the job has finished its execution (either succeeding or failing).
  For complex requests the command retrieves recursively all the output sandboxes of the request sub-jobs.

  The user can specify the location on the local file system where the files have to be stored using the -{}-dir  option, and she
  can specify a single job ID or more than one, e.g.:
\begin{scriptsize}
\begin{verbatim}
  [trinity] /home/dorigoa > glite-wms-job-output --dir /home/dorigoa/myout https://tigerman.cnaf.infn.it:9000/6WBIXoTt1kk1SSHJc5hmcw
\end{verbatim}
\end{scriptsize}
%  If only a job ID is specified the retrieved files are stored in the specified dir; if more than one job is 
%  given as input to the command, e.g.:
\begin{scriptsize}
\begin{verbatim}
  [trinity] /home/dorigoa > glite-wms-job-output --dir /home/dorigoa/myout -i myids
\end{verbatim}
\end{scriptsize}
  then a subdir (named <user name>\_<jobId unique string> ) is created for each job specified on the command line and/or in /home/dorigoa/myout for containing
  the files for the given jobs. The same applies if the specified handle is the ID of a complex request.

  The -{}-list-only  option allows listing the files of the output sandbox of a job without retrieving them.



\textbf{Commands Sequence:}


 We report here an example of the sequence of steps that have to be performed to submit a job and to monitor 
 its progress.

 - Create a valid VOMS proxy as follows:
\begin{scriptsize}
\begin{verbatim}

	dorigoa@lxgrid05 10:53:26 ~>voms-proxy-init -voms dteam
	Your identity: /C=IT/O=INFN/OU=Personal Certificate/L=Padova/CN=Alvise Dorigo
	Creating temporary proxy .......................................... Done
	Contacting  voms.hellasgrid.gr:15004 [/C=GR/O=HellasGrid/OU=hellasgrid.gr/CN=voms.hellasgrid.gr] "dteam" Done
	Creating proxy ............................................................................... Done
	Your proxy is valid until Mon Jul 25 22:53:31 2011

\end{verbatim}
\end{scriptsize}

- Delegate your proxy to the WMProxy service:
\begin{scriptsize}
\begin{verbatim}
  [trinity] /home/dorigoa > glite-wms-job-delegate-proxy -d delID1234

  Delegating Credential to the service https://tigerman.cnaf.infn.it:7443/glite_wms_wmproxy_server

  ================== glite-wms-job-delegate-proxy Success ==================

  Your proxy has been successfully delegated to the WMProxy:
  https://tigerman.cnaf.infn.it:7443/glite_wms_wmproxy_server

  with the delegation identifier: delID1234

  ========================================================================== 
\end{verbatim}
\end{scriptsize}

- Submit a job to the WMProxy associating the previously delegated proxy to it:
\begin{scriptsize}
\begin{verbatim}
  [trinity] /home/dorigoa > glite-wms-job-submit -d delID1234 -o ids JDL/tt.jdl 

  Connecting to the service https://tigerman.cnaf.infn.it:7443/glite_wms_wmproxy_server

  ====================== glite-wms-job-submit Success ======================

  The job has been successfully submitted to the WMProxy
  Your job identifier is:

  https://tigerman.cnaf.infn.it:9000/NEs1GSI8uYDE33SaobBIyA

  The job identifier has been saved in the following file:
  /home/dorigoa/ids

  ==========================================================================
\end{verbatim}
\end{scriptsize}

- Submit a collection to the WMProxy associating the same delegated proxy to it:
\begin{scriptsize}
\begin{verbatim}
  [trinity] /home/dorigoa > glite-wms-job-submit -d delID1234 -o ids --collection /home/dorigoa/JDLS

  Connecting to the service https://tigerman.cnaf.infn.it:7443/glite_wms_wmproxy_server

  ====================== glite-wms-job-submit Success ======================

  The job has been successfully submitted to the WMProxy
  Your job identifier is:

  https://tigerman.cnaf.infn.it:9000/stshjZvp-yPdn7jGVX64Ag

  The job identifier has been saved in the following file:
  /home/dorigoa/ids

  ==========================================================================
\end{verbatim}
\end{scriptsize}

- Monitor the status of the job (using the glite-wms-job-status command):
\begin{scriptsize}
\begin{verbatim}
  [trinity] /home/dorigoa > glite-wms-job-status -i ids 

  ------------------------------------------------------------------
  1 : https://tigerman.cnaf.infn.it:9000/NEs1GSI8uYDE33SaobBIyA
  2 : https://tigerman.cnaf.infn.it:9000/stshjZvp-yPdn7jGVX64Ag
  a : all
  q : quit
  ------------------------------------------------------------------

  Choose one or more jobId(s) in the list - [1-2]all:1


  *************************************************************
  BOOKKEEPING INFORMATION:

  Status info for the Job : https://tigerman.cnaf.infn.it:9000/NEs1GSI8uYDE33SaobBIyA
  Current Status:     Ready 
  Status Reason:      unavailable
  Destination:        lxde01.pd.infn.it:2119/blah-lsf-grid03
  Submitted:          Mon Jul 11 19:48:38 2005 CEST
  *************************************************************
\end{verbatim}
\end{scriptsize}

- Retrieve the output for the collection:
\begin{scriptsize}
\begin{verbatim}
  [trinity] /home/dorigoa > glite-wms-job-output --dir /home/dorigoa/coll_output -i ids 
  ------------------------------------------------------------------
  1 : https://tigerman.cnaf.infn.it:9000/NEs1GSI8uYDE33SaobBIyA
  2 : https://tigerman.cnaf.infn.it:9000/stshjZvp-yPdn7jGVX64Ag
  a : all
  q : quit
  ------------------------------------------------------------------

  Choose one or more jobId(s) in the list - [1-2]all (use , as separator or - for a range): 2

  ================================================================================

                        JOB GET OUTPUT OUTCOME
  Output files for the job:
  https://tigerman.cnaf.infn.it:9000/stshjZvp-yPdn7jGVX64Ag
  are stored in the directory: /home/dorigoa/coll_output

  ================================================================================

  [trinity] /home/dorigoa > ll /home/dorigoa/coll_output
  total 12
  drwxr-xr-x    2 dorigoa  grid         4096 Jul 11 20:03 dorigoa_FxBa61Ws4g4Z0wa3k28ckA
  drwxr-xr-x    2 dorigoa  grid         4096 Jul 11 20:02 dorigoa_vWDkzyzNOU6GG_ENPk-QEQ
  drwxr-xr-x    2 dorigoa  grid         4096 Jul 11 20:03 dorigoa_ZKpwNUXXyrbbUjDZd5B0dQ
\end{verbatim}
\end{scriptsize}

- Check the status of the Collection and its nodes:
\begin{scriptsize}
\begin{verbatim}
  [trinity] /home/dorigoa > glite-wms-job-status -v 0 -i ids

  ------------------------------------------------------------------
  1 : https://tigerman.cnaf.infn.it:9000/NEs1GSI8uYDE33SaobBIyA
  2 : https://tigerman.cnaf.infn.it:9000/stshjZvp-yPdn7jGVX64Ag
  a : all
  q : quit
  ------------------------------------------------------------------

  Choose one or more jobId(s) in the list - [1-4]all:2

  *************************************************************
  BOOKKEEPING INFORMATION:

  Status info for the Job : https://tigerman.cnaf.infn.it:9000/stshjZvp-yPdn7jGVX64Ag
  Current Status:     Done (Success)

  - Nodes information for: 
      Status info for the Job : https://tigerman.cnaf.infn.it:9000/vWDkzyzNOU6GG_ENPk-QEQ
      Current Status:     Done (Success)
      Status info for the Job : https://tigerman.cnaf.infn.it:9000/FxBa61Ws4g4Z0wa3k28ckA
      Current Status:     Done (Success)
      Status info for the Job : https://tigerman.cnaf.infn.it:9000/ZKpwNUXXyrbbUjDZd5B0dQ
      Current Status:     Done (Success)
  *************************************************************

\end{verbatim}
\end{scriptsize}

