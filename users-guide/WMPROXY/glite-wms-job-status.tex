
\subsection{glite-wms-job-status}
\label{glite-wms-job-status}

\medskip
\textbf{glite-wms-job-status}
\smallskip


\textbf{SYNOPSIS}
\smallskip

\textbf{glite-wms-job-status [options] <jobId 1> <jobId 2> ...}

\begin{verbatim}
options:
        --version
        --help
        --config, -c     <configfile>
        --vo <voname>
        --debug
        --logfile        <filepath>
        --noint
        --input, -i      <filepath>
        --output, -o     <filepath>
        --all
        --verbosity, -v  <level>
        --from           <[MM:DD:]hh:mm[:[CC]YY]>
        --to             <[MM:DD:]hh:mm[:[CC]YY]>
        --user-tag       <tag name>=<tag value>
        --status, -s     <status code>
        --exclude, -e    <status code>
        --nonodes
\end{verbatim}

\medskip
\textbf{DESCRIPTION}
\smallskip

This command prints the status of a job previously submitted using glite-wms-job-submit.
The job status request is sent to the LB that provides the requested information.
This can be done during the whole job life.
glite-wms-job-status can monitor one or more jobs: the jobs to be checked are identified by one or more job 
identifiers (jobIds returned by glite-wms-job-submit) provided as arguments to the command and 
separated by a blank space.


\medskip\textbf{OPTIONS}\smallskip



\textbf{-{}-version}

displays WMProxy client and servers version numbers.
If the -{}-endpoint option is specified only this service is contacted to retrieve its version. If it is not used, the endpoint contacted is that one specified with GLITE\_WMS\_WMPROXY\_ENDPOINT environment variable. If the option is not specified and the environment variable is not set, all servers listed in the WMProxyEndPoints attribute of the user configuration file are contacted. For each of these service, the returned result will be displayed on the standard output (the version numbers in case of success; otherwise the error message).




\textbf{-{}-help}

displays command usage




\textbf{-{}-config, -c <configfile>}

if the command is launched with this option, the configuration file pointed by <configfile> is used. This option is meaningless when used together with -{}-vo option




\textbf{-{}-vo <voname>}

this option allows the user to specify the Virtual Organization she/he is currently working for.
If the user proxy contains VOMS extensions then the VO specified through this option is overridden by the
default VO contained in the proxy (i.e. this option is only useful when working with non-VOMS proxies).
This option is meaningless when used together with -{}-config-vo option.




\textbf{-{}-debug}

when this option is specified, debugging information is displayed on the standard output and written also either into the default log file:


\begin{verbatim}
glite-wms-job-<command_name>_<uid>_<pid>_<time>.log
\end{verbatim}

located under the /var/tmp  directory or in the log file specified with -{}-logfile option.




\textbf{-{}-logfile <filepath>}

When this option is specified, a command log file (whose pathname is <filepath> ) is created.



\textbf{-{}-noint}

if this option is specified, every interactive question to the user is skipped and the operation 
is continued (when possible)





\textbf{-{}-input, -i <filepath>}

Allow the user to select the JobId(s) from an input file located in <filepath>.
The list of jobIds contained in the file is displayed and the user is prompted for a choice. Single 
jobs can be selected specifying the numbers associated to the job identifiers separated by commas. 
E.g. selects the first, the third and the fifth jobId in the list.
Ranges can also be selected specifying ends separated by a dash. E.g. selects jobIds in the list 
from third position (included) to sixth position (included). It is worth mentioning that it is 
possible to select at the same time ranges and single jobs. E.g. selects the first job id in the list, 
the ids from the third to the fifth (ends included) and finally the eighth one.
When specified toghether with -{}-noint, all available JobId are selected.
 This option cannot be used when one or more jobIds have been specified as extra command argument





\textbf{-{}-output, -o <filepath>}

writes the results of the operation in the file specified by <filepath> instead of the standard output. 
<filepath> can be either a simple name or an absolute path (on the submitting machine). In the former case 
the file <filepath> is created in the current working directory.





\textbf{-{}-all}

displays status information about all job owned by the user submitting the command. This option can't 
be used
either if one or more jobIds have been specified or if the -{}-input option has been specified. All LBs
listed in the vo-specific UI configuration file

\$GLITE\_WMS\_LOCATION/etc/<vo\_name>/glite\_wmsui.conf 

 are contacted to fulfill this request.





\textbf{-{}-verbosity, -v <level>}

sets the detail level of information about the job displayed to the user. Possible values for <level> are 
0 (only JobId and status/event displayed), 1 (timestamp and source information added), 
 2 (all information but JDLs displayed), 3 (complete information containing all JDL strings)





\textbf{-{}-from <[MM:DD:]hh:mm[:[CC]YY]>}

makes the command query LB for jobs that have been submitted (more precisely entered the "Submitted" status) 
after the specified date/time.
If only hours and minutes are specified then the current day is taken into account. 
If the year is not specified then the current year is taken into account.





\textbf{-{}-to <[MM:DD:]hh:mm[:[CC]YY]>}

makes the command query LB for jobs that have been submitted (more precisely entered the "Submitted" status) 
before the specified date/time.
If only hours and minutes are specified then the current day is taken into account.
 If the year is not specified then the current year is taken into account.





\textbf{-{}-user-tag <tag name>=<tag value>}

makes the command include only jobs that have defined specified usertag name and value





\textbf{-{}-status, -s <status code>}

makes the command query LB for jobs that are in the specified status.
The status value can be either an integer or a (case insensitive) string; the following possible values 
are allowed:
UNDEF (0), SUBMITTED(1), WAITING(2), READY(3), SCHEDULED(4), RUNNING(5), DONE(6), CLEARED(7), ABORTED(8), 
CANCELLED(9),
UNKNOWN(10), PURGED(11).
This option can be repeated several times, all status conditions will be considered as in a logical OR 
operation

(i.e.  -s SUBMITTED -{}-status 3  will query all jobs that are either in SUBMITTED or in READY status)





\textbf{-{}-exclude, -e <status code>}

makes the command query LB for jobs that are NOT in the specified status.
The status value can be either an integer or a (case insensitive) string; the following possible values 
are allowed:
UNDEF (0), SUBMITTED(1), WAITING(2), READY(3), SCHEDULED(4), RUNNING(5), DONE(6), CLEARED(7), ABORTED(8), 
CANCELLED(9),
UNKNOWN(10), PURGED(11).
This option can be repeated several times, all status conditions will be considered as in a logical AND 
operation

(i.e.  -e SUBMITTED -{}-exclude 3  will query all jobs that are neither in SUBMITTED nor in READY status)





\textbf{-{}-nonodes}

This option will not display any information of (if present) sub jobs of any dag, only requested JobId(s) 
info will be taken into account





\medskip
\textbf{FILES}
\smallskip


The path \verb /etc must contain the following UI configuration files:

\begin{itemize}


\item  \verb glite_wmsui_cmd_var.conf {}

\item  \verb glite_wmsui_cmd_err.conf {}

\item  \verb glite_wmsui_cmd_help.conf {}

\end{itemize}


\verb glite_wmsui_cmd_var.conf {} will contain custom configuration default values.
A different configuration file may be specified either by using the -{}-config option or by setting 
the GLITE\_WMSUI\_COMMANDS\_CONFIG environment variable. 
Here follows a possible example:
\begin{verbatim}
  [
    RetryCount = 3 ;
    ErrorStorage= "/tmp" ;
    OutputStorage="/tmp";
    ListenerStorage = "/tmp" ;
    LoggingTimeout = 30 ;
    LoggingSyncTimeout = 30 ;
    NSLoggerLevel = 0;
    DefaultStatusLevel = 1 ;
    DefaultLogInfoLevel = 1;
  ]
\end{verbatim}

\verb glite_wmsui_cmd_err.conf {} will contain UI exception mapping 
between error codes and error messages (no relocation possible)

\verb glite_wmsui_cmd_help.conf {} will contain UI long-help information (no relocation possible)


\verb <voName>/glite_wmsui.conf {}: The user interface configuration file; 
the standard path location is \verb /etc/glite-wms : 

here follows a possible example:
\begin{verbatim}
  [
    JdlDefaultAttributes = [
      virtualorganisation="infngrid";
      requirements = other.GlueCEStateStatus == "Production";
      retryCount = 3;
      rank = -other.GlueCEStateEstimatedResponseTime;
      MyProxyServer="myproxy.cern.ch";
    ];
    
    DelegationId = "luca";
    ErrorStorage="\${GLITE\_LOCATION\_LOG}";
    OutputStorage="/tmp";
    ListenerStorage="\${GLITE\_LOCATION\_TMP}";
    WMProxyEndPoints = {"https://ghemon.cnaf.infn.it:7443/glite\_wms\_wmproxy\_server"};
    LBAddress = "ghemon.cnaf.infn.it:9000";
    LBServiceDiscoveryType ="org.glite.lb.server";
    WMProxyServiceDiscoveryType="org.glite.wms.wmproxy";
  ]
\end{verbatim}

\verb /tmp/x509up_u<uid> {}: A valid X509 user proxy; 
use the X509\_USER\_PROXY environment variable to override the default location



\medskip
\textbf{ENVIRONMENT}
\smallskip


\begin{itemize}



\item GLITE\_WMS\_WMPROXY\_ENDPOINT: This variable may be set to specify the endpoint URL


\item GLOBUS\_LOCATION: This variable must be set when the Globus installation is not located in the default path \verb /opt/globus {}.


\item GLOBUS\_TCP\_PORT\_RANGE="<val min> <val max>": This variable must be set to define a range of ports to be used for inbound connections in the interactivity context


\item X509\_CERT\_DIR: This variable may be set to override the default location of the trusted certificates directory, which is normally \verb /etc/grid-security/certificates {}.


\item X509\_USER\_PROXY: This variable may be set to override the default location of the user proxy credentials, which is normally \verb /tmp/x509up_u<uid> {}.


\item GLITE\_SD\_PLUGIN: If Service Discovery querying is needed, this variable can be used in order to set a specific (or more) plugin, normally bdii, rgma (or both, separated by comma)


\item LCG\_GFAL\_INFOSYS: If Service Discovery querying is needed, this variable can be used in order to set a specific Server where to perform the queries: for instance LCG\_GFAL\_INFOSYS='gridit-bdii-01.cnaf.infn.it:2170'


\end{itemize}

