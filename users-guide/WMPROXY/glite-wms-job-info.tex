
\subsection{glite-wms-job-info}
\label{glite-wms-job-info}

\medskip
\textbf{glite-wms-job-info}
\smallskip


\textbf{SYNOPSIS}
\smallskip

\textbf{glite-wms-job-info [options] <operation options> <job ID>}

\begin{verbatim}
operation options (mandatory):
        --jdl
        --jdl-original, -j
        --proxy, -p
        --delegationid, -d <id_string> (*)

options:
        --help
        --endpoint, -e  <service_URL>
        --config, -c    <file_path>
        --vo            <vo_name>
        --input, -i     <file_path> (*)
        --output, -o    <file_path>
        --noint
        --debug
        --logfile       <file_path>

(*) argument <job Id> is not required
\end{verbatim}

\medskip
\textbf{DESCRIPTION}
\smallskip

This command is only available for glite version >= 3.1.

Allow retrieving useful information about the user delegated proxy, the delegated identification details of the user, 
the JDL of a previously submitted job, etc.
Each piece of information has to be queried through the proper option.
E.g.:

\begin{enumerate}


\item  \begin{verbatim}
glite-wms-job-info --delegationid myDelegationID
\end{verbatim}


\item  \begin{verbatim}glite-wms-job-info --jdl mySubmittedJobID
\end{verbatim}


\item  \begin{verbatim}
glite-wms-job-info --proxy mySubmittedJobID
\end{verbatim}

\end{enumerate}



\medskip\textbf{OPTIONS}\smallskip



\textbf{-{}-version}

displays WMProxy client and servers version numbers.
If the -{}-endpoint option is specified only this service is contacted to retrieve its version. If it is not used, the endpoint contacted is that one specified with GLITE\_WMS\_WMPROXY\_ENDPOINT environment variable. If the option is not specified and the environment variable is not set, all servers listed in the WMProxyEndPoints attribute of the user configuration file are contacted. For each of these service, the returned result will be displayed on the standard output (the version numbers in case of success; otherwise the error message).




\textbf{-{}-help}

displays command usage




\textbf{-{}-config, -c <configfile>}

if the command is launched with this option, the configuration file pointed by <configfile> is used. This option is meaningless when used together with -{}-vo option




\textbf{-{}-vo <voname>}

this option allows the user to specify the Virtual Organization she/he is currently working for.
If the user proxy contains VOMS extensions then the VO specified through this option is overridden by the
default VO contained in the proxy (i.e. this option is only useful when working with non-VOMS proxies).
This option is meaningless when used together with -{}-config-vo option.




\textbf{-{}-debug}

when this option is specified, debugging information is displayed on the standard output and written also either into the default log file:


\begin{verbatim}
glite-wms-job-<command_name>_<uid>_<pid>_<time>.log
\end{verbatim}

located under the /var/tmp  directory or in the log file specified with -{}-logfile option.




\textbf{-{}-logfile <filepath>}

When this option is specified, a command log file (whose pathname is <filepath> ) is created.



\textbf{-{}-noint}

if this option is specified, every interactive question to the user is skipped and the operation is continued 
(when possible)





\textbf{-{}-input, -i <filepath>}

Allow the user to select the desired job among the ones contained in the file located in <filepath>. 
This option cannot be used if the jobId has been specified. A job has to be selected specifying 
the numbers associated to the job identifier





\textbf{-{}-delegationid, -d <idstring>}

if this option is specified, the user proxy associated to the id delegated previously will be displayed.





\textbf{-{}-proxy, -p <JobID>}

Retrieves and displays information for the proxy delegated with the specified JobId: 
the WMProxy service where the job was submitted is contacted and asked for needed information.





\textbf{-{}--jdl <jobid>}

Retrieves and displays the JDL registered to the LB service for the specified JobId: 
the WMProxy service where the job was submitted is contacted and asked for needed information.
This option is only available from glite version >= 3.1.





\textbf{-{}-jdl-original, -j <jobid>}

Retrieves and displays the original JDL submitted by the user to the WMProxy service. 
This information might be quite different from the registered JDL, mostly for parametric and collection of jobs.
This option is only available from glite version >= 3.1.





\textbf{-{}-endpoint, -e <serviceURL>}

when this option is specified, the operations are performed contacting the WMProxy 
service represented by the given <serviceURL>. If it is not used, the endpoint contacted is that one 
specified by the GLITE\_WMS\_WMPROXY\_ENDPOINT environment variable. 
If both this option is not specified 
and the environment variable is not set, the service request will be sent to one of the endpoints listed 
in the WMProxyEndPoints attribute in the user configuration file (randomly chosen among the URLs of the 
attribute list). If the chosen service is not available, a succession of attempts are performed on the 
other specified services until the connection with one of these endpoints can be established or all 
services have been contacted without success. In this latter case the operation can not be obviously 
performed and the execution of the command is stopped with an error message.





\textbf{-{}-output, -o <filename>}

Stores the retrieved information into the specified file instead of the standard output. 
The specified file path can be either a simple name or an absolute path (on the submitting machine). 
In the former case the file <filepath> is created in the current working directory.
This option is only available from glite version >= 3.1.





\medskip
\textbf{EXAMPLES}
\smallskip


\begin{enumerate}


\item  Display information for the proxy delegated to the WMProxy service with the specified identifier:
\begin{verbatim}
glite-wms-job-info -d <delegationid>
\end{verbatim}


\item  Display information for the proxy delegated with a previously submitted Job:
\begin{verbatim}
glite-wms-job-info -p <JobId>
\end{verbatim}


\item  Display the submission string registered to LB server for a previously submitted Job:
\begin{verbatim}
glite-wms-job-info --jdl <JobId> -o <OutputFile>
\end{verbatim}


\item  Display the original submission string sent to the WMProxy service for a previously submitted Job
\begin{verbatim}
glite-wms-job-info -j <JobId>
\end{verbatim}


\item  Send the request to the WMProxy service whose URL is specified with the -e option
\begin{verbatim}
glite-wms-job-info -d <delegationid> -e \
                   https://wmproxy.glite.it:7443/glite_wms_wmproxy_server
\end{verbatim}


\item  Store into a file the submission string registered to LB server for a previously submitted Job:
\begin{verbatim}
glite-wms-job-info --jdl <JobId> -o <OutputFile>
\end{verbatim}

\end{enumerate}


When -{}-endpoint (-e) is not specified, the search of an available WMProxy service is performed according to the modality
reported in the description of the -{}-endpoint option.


\medskip
\textbf{FILES}
\smallskip

\verb voName/glite_wms.conf {}: The user configuration file. The standard 
path location is \verb /etc/glite-wms . 

\verb /tmp/x509up_u<uid> {}: A valid X509 user proxy; 
use the X509\_USER\_PROXY environment variable to override the default location

JDL: The file containing the description of the job in the JDL language located in the
 path specified by JDL\_file (the last argument of this command); multiple jdl files can be 
used with the -{}-collection option


\medskip
\textbf{ENVIRONMENT}
\smallskip


\begin{itemize}



\item GLITE\_WMS\_WMPROXY\_ENDPOINT: This variable may be set to specify the endpoint URL


\item GLOBUS\_LOCATION: This variable must be set when the Globus installation is not located in the default path \verb /opt/globus {}.


\item GLOBUS\_TCP\_PORT\_RANGE="<val min> <val max>": This variable must be set to define a range of ports to be used for inbound connections in the interactivity context


\item X509\_CERT\_DIR: This variable may be set to override the default location of the trusted certificates directory, which is normally \verb /etc/grid-security/certificates {}.


\item X509\_USER\_PROXY: This variable may be set to override the default location of the user proxy credentials, which is normally \verb /tmp/x509up_u<uid> {}.


\item GLITE\_SD\_PLUGIN: If Service Discovery querying is needed, this variable can be used in order to set a specific (or more) plugin, normally bdii, rgma (or both, separated by comma)


\item LCG\_GFAL\_INFOSYS: If Service Discovery querying is needed, this variable can be used in order to set a specific Server where to perform the queries: for instance LCG\_GFAL\_INFOSYS='gridit-bdii-01.cnaf.infn.it:2170'


\end{itemize}

