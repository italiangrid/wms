%\section{Security}

For the Grid to be an effective framework for largely distributed computation, users, user processes
and Grid services must work in a secure environment.
 
Due to this, the communication between the client and the WMProxy service is authenticated.
Depending on the specific interaction, an entity authenticates itself to the 
other peer using either its own credential or a delegated user credential or both. For example when the 
WMProxy user interface passes a job to the WMProxy service, the client authenticates using a 
proxy user credential 
(a proxy certificate) whereas WMProxy uses its own service credential. 

The WMProxy user interface uses a proxy user credential to limit the risk 
of compromising the original credential in the hands of the user. 
 
The user or service identity and their public key are included in a X.509 certificate signed by a trusted 
Certification Authority (CA), whose purpose is to guarantee the association between that public key and its 
owner.

According to what just premised, the user has to possess a valid X.509 
certificate on the submitting machine, consisting of two files: the certificate file and the private 
key file. The location of the two mentioned files is assumed to be either pointed to respectively 
\textit{"\$X509\_USER\_CERT"} and \textit{"\$X509\_USER\_KEY"} or 
by \textit{"\$HOME\-/.globus\-/usercert.pem"} and \textit{"\$HOME\-/.globus/\-userkey.pem"} if the 
X509 environment variables are not set. 

\medskip


\framebox{
\parbox[c]{15cm}{
 
\textbf{How to extract userkey.pem and usercert.pem from your certificate}\medskip

Usually X.509 Certificates are downloaded using a browser and managed by the browser itself. 
Anyway it is possible to export your certificate in a file PKCS12 (which will probably have the 
extension .p12 or .pfx). 
 
The procedure to export the certificate vary from browser to browser, for example Internet Explorer 
starts with "Tools $->$ Internet Options $->$ Content"; Netscape Communicator has a "Security" button 
on the top menu bar; Mozilla starts with "Edit $->$ Preferences $->$ Privacy and Security $->$ Certificates" 
and Firefox has "Edit $->$ Preferences $->$ Advanced $->$ Certificates $->$ manage certificates $->$ backup".
  
PKCS12 format is not accepted by the Grid Security Infrastructure, but you can easily convert 
it into the supported standard (PEM). This operation will split your *.p12 file in two files: the certificate 
(usercert.pm) and the private key (userkey.pm). The conversion can be performed with openssl tool:

\medskip

{\scriptsize{ 
\$ openssl pkcs12 -nocerts -in mycert.p12 -out userkey.pem\\
\$ openssl pkcs12 -clcerts -nokeys -in mycert.p12 -out usercert.pem\\
\$ chmod 0400 userkey.pem\\
\$ chmod 0600 usercert.pem\\
 }}
\medskip

Permission must be set as shown not only for security reasons: \emph{voms-proxy-init} and \emph{grid-proxy-init}
commands will fail if your private key is not protected as listed above.
}} 
%end of framebox and parbox
\medskip

Actually the user certificate and private key files are not mandatory on the WMProxy client machine; indeed they are 
only needed for the creation of the proxy user credentials through the \textit{grid-proxy-init} or 
\textit{voms-proxy-init} commands. Please refer to the VOMS User's Guide ~\cite{voms-core} for detailed 
information about the VOMS client commands.  

Alternatively you can download the proxy credentials from a trusted site and work with it without having 
the cert and key available locally.

All WMProxy client commands, when started, check for the existence and expiration date of a user proxy credentials in 
the location pointed to by \textit{"\$X509\_USER\_PROXY"} or in \textit{"/tmp/\-x509up\_u$<$UID$>$"} 
(where $<$UID$>$ is the user identifier in the submitting machine OS) if the X509 environment variable is 
not set. If the proxy certificate does not exist or has expired, the issued WMProxy client command returns an 
error message to the user and exits.

It is important to note that besides authentication, proxy credential issued by VOMS, i.e. containing 
FQANs (Fully Qualified Attribute Names ~\cite{voms-core}), are used by WMProxy to get the VO the user is currently working for. 
If a given proxy credential contains attributes for more then one VO, than the default one (i.e. the one first 
position) is considered.   


\subsection{Authorization}
The client must be properly authorized when interacting with the WMProxy service.
As the AuthZ is implemented in WMProxy via an allow-deny gacl file, 
this means that either the FQAN or the DN (in case of globus-style proxies) of the
client must be properly listed and authorized in the \emph{glite\_wms\_wmproxy.gacl} file 
on the WMProxy service machine. 


\subsubsection{Authorization on job management operations}
\label{jobauthz}

As a general rule, a user can manage only her own jobs (i.e. the ones
she submitted). So, for example, a user can't cancel or retrieve the output
of jobs submitted by other users.

\subsection{Proxy credential delegation}

Each job submitted to the WMProxy service has to be associated with a credential previously delegated 
by the user who submitted it. This credential is then used by the service to perform job related 
operations requiring interaction with other services (e.g. submission to the CE, a GridFTP 
file transfer etc.).  


As described in Sects.~\ref{glite-wms-job-delegate-proxy}, ~\ref{glite-wms-job-submit} and 
~\ref{glite-wms-job-list-match}, it is possible to ask either the ``automatic'' delegation 
of the credentials during the submission (or list-match) operation, or to explicitly 
delegate credentials, and then ask to rely on these previously delegated credentials.
It is recommended to rely on this latter mechanism, using the same delegated proxy for multiple 
job submissions, instead of delegating each time a proxy.
Delegating a proxy, in fact, is an operation that can require a non negligible
time. Be aware that using ``automatic'' delegation heavily reduces the overall submission rate of the WMS to the CREAM-based CEs.
