
\subsection{glite-wms-job-delegate-proxy}
\label{glite-wms-job-delegate-proxy}

\medskip
\textbf{glite-wms-job-delegate-proxy}
\smallskip


\textbf{SYNOPSIS}
\smallskip

\textbf{glite-wms-job-delegate-proxy <delegation-opts> [options]}

\begin{verbatim}
delegation-opts:
        --delegationid, -d <idstring>
        --autm-delegation, -a

options:
        --version
        --help
        --config, -c    <filepath>
        --vo            <voname>
        --debug
        --logfile       <filepath>
        --noint
        --endpoint, -e  <serviceURL>
        --output, -o    <filepath>
        --all
\end{verbatim}

\medskip
\textbf{DESCRIPTION}
\smallskip

glite-wms-job-delegate-proxy allows delegating a user proxy to the WMProxy service. 
The user can either specify the delegationid to be associated with the delegated proxy by
using the -{}-delegationid option:
\begin{verbatim}
glite-wms-job-delegate-proxy -d mydelegationiD
\end{verbatim}
or make the command automatically generate an ID by using the -{}-autm-delegation option.

One of these two options is mandatory and they're mutually exclusive.
If a delegation with the same id and belonging to the same user already exists in the WMProxy service, 
the new proxy will overwrite the one associated to that delegation id.
Credential delegation with the same id used for a proxy belonging to another user is possible.


\medskip\textbf{OPTIONS}\smallskip



\textbf{-{}-version}

displays WMProxy client and servers version numbers.
If the -{}-endpoint option is specified only this service is contacted to retrieve its version. If it is not used, the endpoint contacted is that one specified with GLITE\_WMS\_WMPROXY\_ENDPOINT environment variable. If the option is not specified and the environment variable is not set, all servers listed in the WMProxyEndPoints attribute of the user configuration file are contacted. For each of these service, the returned result will be displayed on the standard output (the version numbers in case of success; otherwise the error message).




\textbf{-{}-help}

displays command usage




\textbf{-{}-config, -c <configfile>}

if the command is launched with this option, the configuration file pointed by <configfile> is used. This option is meaningless when used together with -{}-vo option




\textbf{-{}-vo <voname>}

this option allows the user to specify the Virtual Organization she/he is currently working for.
If the user proxy contains VOMS extensions then the VO specified through this option is overridden by the
default VO contained in the proxy (i.e. this option is only useful when working with non-VOMS proxies).
This option is meaningless when used together with -{}-config-vo option.




\textbf{-{}-debug}

when this option is specified, debugging information is displayed on the standard output and written also either into the default log file:


\begin{verbatim}
glite-wms-job-<command_name>_<uid>_<pid>_<time>.log
\end{verbatim}

located under the /var/tmp  directory or in the log file specified with -{}-logfile option.




\textbf{-{}-logfile <filepath>}

When this option is specified, a command log file (whose pathname is <filepath> ) is created.



\textbf{-{}-noint}

if this option is specified, every interactive question to the user is skipped ('yes' is assumed as user's answer) and the operation is continued (when possible)





\textbf{-{}-delegationid, -d <idstring>}

if this option is specified, the proxy that will be delegated is identified by <idstring>. This proxy can be therefore used for operations like
job registration, job submission and job list matching until its expiration specifying the <idstring>. It must be used in place of -{}-autm-delegation option.





\textbf{-{}-autm-delegation, -a}

this option is specified to make automatic generation of the identifier string (delegationid) that will be associated to the delegated proxy. It
must be used in place of the -{}-delegationid (-d) option.





\textbf{-{}-endpoint, -e <serviceURL>}

when this option is specified, the operations are performed contacting the WMProxy service represented by the given serviceURL. If it is not used, the endpoint contacted is that one specified by the GLITE\_WMS\_WMPROXY\_ENDPOINT environment variable. If none of these options is specified and the environment variable is not set, the service request will be sent to one of the endpoints listed in the WMProxyEndPoints attribute in the user configuration file (randomly chosen among the URLs of the attribute list). If the chosen service is not available, a succession of attempts are performed on the other specified services until the connection with one of these endpoints can be established or all services have been contacted without success. In this latter case the operation can not be obviously performed and the execution of the command is stopped with an error message.




\textbf{-{}-all}

when this option is specified, delegation is performed on all the endpoints




\textbf{-{}-output, -o <filepath>}

if this option is specified, the following information is saved in the file specified by <filepath>: date and time the delegation operation was performed; WMProxy service URL;  idstring associated to the delegated proxy. filepath can be either a simple name or an absolute path (on the submitting machine). In the former case the file is created in the current working directory.





\medskip
\textbf{EXAMPLES}
\smallskip


\begin{enumerate}


\item  delegates the user credential with "exID" identifier :
\begin{verbatim}
glite-wms-job-delegate -d exID
\end{verbatim}


\item  delegates the user  credential with "exID" identifier  to the WMProxy service specified with the -e option:
\begin{verbatim}
glite-wms-job-delegate -d exID \
                       -e https://wmproxy.glite.it:7443/glite_wms_wmproxy_server
\end{verbatim}


\item  delegates the user credential automatically generating the id string :
\begin{verbatim}
glite-wms-job-delegate -a
\end{verbatim}


\item  delegates the user credential to the WMProxy service specified with the -e option automatically generating the id string  :
\begin{verbatim}
glite-wms-job-delegate -a \
                       -e https://wmproxy.glite.it:7443/glite_wms_wmproxy_server
\end{verbatim}


\end{enumerate}


When -{}-endpoint (-e) is not specified, the search of an available WMProxy service is performed according to the modality reported in the
description of the -{}-endpoint option.


\medskip
\textbf{FILES}
\smallskip

\verb voName/glite_wms.conf {}: The user configuration file. The standard 
path location is \verb /etc/glite-wms . 

\verb /tmp/x509up_u<uid> {}: A valid X509 user proxy; 
use the X509\_USER\_PROXY environment variable to override the default location


\medskip
\textbf{ENVIRONMENT}
\smallskip


\begin{itemize}



\item GLITE\_WMS\_WMPROXY\_ENDPOINT: This variable may be set to specify the endpoint URL


\item GLOBUS\_LOCATION: This variable must be set when the Globus installation is not located in the default path \verb /opt/globus {}.


\item GLOBUS\_TCP\_PORT\_RANGE="<val min> <val max>": This variable must be set to define a range of ports to be used for inbound connections in the interactivity context


\item X509\_CERT\_DIR: This variable may be set to override the default location of the trusted certificates directory, which is normally \verb /etc/grid-security/certificates {}.


\item X509\_USER\_PROXY: This variable may be set to override the default location of the user proxy credentials, which is normally \verb /tmp/x509up_u<uid> {}.


\item GLITE\_SD\_PLUGIN: If Service Discovery querying is needed, this variable can be used in order to set a specific (or more) plugin, normally bdii, rgma (or both, separated by comma)


\item LCG\_GFAL\_INFOSYS: If Service Discovery querying is needed, this variable can be used in order to set a specific Server where to perform the queries: for instance LCG\_GFAL\_INFOSYS='gridit-bdii-01.cnaf.infn.it:2170'


\end{itemize}

