%\subsection{Commandline Configuration}

This section provides a detailed description of the WMProxy client commands configuration files, describing
where they are supposed to be stored, which attributes they may contain, and which is the meaning of each attribute.
It is basically the same configuration file (i.e. same structure and same set of admitted attributes) that can be 
located in multiple, different locations thus assuming slightly different meanings.  
Eventually is reported an example of a configuration file.


\subsection{Configuration Files}
\label{wmpconf}

Configuration of the WMProxy client commands is accomplished via two possible
configuration files:


\begin{enumerate}

\item
A VO-specific configuration file named \verb glite\_wmsui.conf , that is looked for in the directory {\verb /etc/glite-wms/<VO\_name\_lowercase>/ }

\item
A user-specific configuration file that is looked for in the following order:

\begin{itemize}


\item
The file specified with the -{}-config  option made available by each WMProxy client command.

\item
\$HOME/.glite/<VO name-lowercase>/glite\_wms.conf otherwise.
VO name must be specified by either using a proxy credential issued by VOMS ~\cite{voms-core} or specifying the -{}-vo  option of the WMProxy client commands.
For those commands taking as input a JDL description (i.e. glite-wms-job-submit and glite-wms-job-list-match), the VO name can be read inside the JDL (VirtualOrganisation attribute) if not specified elsewhere.

\end{itemize}


\end{enumerate}


Each of the above mentioned files is a classad containing definitions for the set of supported 
attributes (see Sect.~\ref{confattr}).


Please note that:
\begin{itemize}

\item If the same attribute is defined in more than a configuration file, the
definition in the user-specific configuration file (if any) is considered.
%Likewise the definitions in the VO-specific configuration file
%have higher priority than the ones specified in the general
%configuration file.

\item At least one configuration file must be present on the client machine, in the above listed locations: 
this is enough for the commands to work.
\item All non-mandatory attributes for which no default value is mentioned below are simply ignored by the commands.
\item All string attributes set to empty string will be ignored by the commands
(therefore possible previously found values will not be taken into account): this is the way a user has for 'disabling' 
features 'enabled' in the general or VO-specific configuration files (e.g. MyProxyServer, HLRLocation, JobProvenance etc.)   

\end{itemize}



\subsection{Configuration Attributes}
\label{confattr}


Here below are listed together, with a brief description, the attributes that can be specified in the 
configuration files; note that some of them must be inserted inside the JdlDefaultAttributes section (indicated by an asterisk before the
attribute name in the following list) as shown in the example at page (\pageref{cliconfex}):


\begin{itemize}

\item
\textbf{ErrorStorage}: the path where error and debug log files are stored.
The default is \verb /var/tmp .

\item
\textbf{OutputStorage}: the path where to copy retrieved job output and perusal files.
This attribute is only read by glite-wms-job-output and glite-wms-job-perusal.
The default is \verb /tmp .

\item
\textbf{ListenerStorage}: the path where to create listener input/output pipes when submitting an interactive job.
The default is \verb /tmp .

\item
$^*$\textbf{VirtualOrganisation}: the value of the Virtual Organization the file is related to.
This attribute is mandatory.


\item
$^*$\textbf{Rank}: the default value for Rank attribute, if not specified in the JDL being submitted.
This attribute is only read by glite-wms-job-submit and glite-wms-job-list-match.


\item
$^*$\textbf{Requirements}: the default value for Requirements attribute, if not specified in the JDL being submitted.
Notice that when original job's JDL already contains this attribute, the final requirements will be a logical AND 
between the two expressions. This attribute is only read by glite-wms-job-submit and glite-wms-job-list-match.


\item
$^*$\textbf{RetryCount}: the default value of RetryCount attribute, if not specified in the JDL being submitted.
This attribute is only read by glite-wms-job-submit and glite-wms-job-list-match.
The default value is 3.

\item
$^*$\textbf{ShallowRetryCount}:the default value of ShallowRetryCount attribute, if not specified in the JDL being submitted.
This attribute is only read by glite-wms-job-submit and glite-wms-job-list-match.
The default value is 3.

\item
\textbf{WMProxyEndPoints}: the WMProxy Service endpoint where to send requests. 
Multiple service endpoints can be specified through this attribute by means of a list of strings. In this case
the glite-wms-job-submit and glite-wms-job-list-match commands choose the endpoint randomly in the list.  
This attribute is mandatory.


\item
$^*$\textbf{MyProxyServer}: the default value of MyProxyServer attribute (i.e. the address of the MyProxy server), if 
not specified in the JDL being submitted. The presence of this attribute activates the proxy renewal support 
in the WMS: if needed, the job user proxy will be renewed by WMS via the MyProxy server during job execution.


\item
\textbf{HLRLocation}: the default value of HLRLocation attribute (i.e. the address of the HLR for the given VO in the format 
<hostname >:<port>:<X509contact string>), if not specified in the JDL being submitted.
HLR is the service responsible for managing the economic transactions and the accounts of user and resources.
The presence of this attribute enables accounting for the submitted job.


\item
$^*$\textbf{JobProvenance}: the default value of  JobProvenance attribute (i.e. the address of the JobProvenance server), 
if not specified in the JDL being submitted. The presence of this attribute activates Job Provenance support: 
WMS will upload job sandbox files on the specified server.

\item
$^*$\textbf{PerusalFileEnable}: Determines, for the job being submitted, whether the job file perusal support has to be 
enabled or not. This attribute is only read by glite-wms-job-submit.
The default is \emph{false}.

\item
$^*$\textbf{AllowZippedISB}: When set to true, makes the glite-wms-job-submit command upload the job's input sandbox files to 
the WMProxy service node through a single tar-zipped file. This attribute is only read by glite-wms-job-submit.
The default is \emph{false}.



\end{itemize}
%\newpage



\subsection{Configuration File Example}
\label{confex}

This is an example of a WMProxy client configuration file:
\label{cliconfex}
\begin{verbatim}

[
  JdlDefaultAttributes = [
	
    virtualorganisation="dteam";
    RetryCount = 3;
    ShallowRetryCount = 5;
    rank =-other.GlueCEStateEstimatedResponseTime;
    requirements = other.GlueCEStateStatus == "Production";
    MyProxyServer="myproxy.cern.ch";
    JobProvenance = "skurut.ics.muni.cz:8901";
    PerusalFileEnable = true ;
    AllowZippedISB    = true ;
  ];
        
    ErrorStorage="/var/log";
    OutputStorage="/tmp";
    ListenerStorage="/tmp";
    WMProxyEndPoints = {
        "https://cream-44.pd.infn.it:7443/glite_wms_wmproxy_server",
        "https://devel09.cnaf.infn.it:7443/glite_wms_wmproxy_server"
    };

    HLRLocation = "skurut.ics.muni.cz:8901";
]


\end{verbatim}
