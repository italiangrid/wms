
\subsection{glite-wms-job-output}
\label{glite-wms-job-output}

\medskip
\textbf{glite-wms-job-output}
\smallskip


\textbf{SYNOPSIS}
\smallskip

\textbf{glite-wms-job-output [options]  <job ID 1> <job ID 2> ...}

\begin{verbatim}
options:
        --version
        --help
        --config, -c  <configfile>
        --vo <voname>
        --debug
        --logfile     <filepath>
        --noint
        --input, -i   <filepath>
        --dir         <directorypath>
        --nosubdir
        --nopurge
        --proto       <protocol>
        --list-only
        --json
        --pretty-print
\end{verbatim}

\medskip
\textbf{DESCRIPTION}
\smallskip

glite-wms-job-output retrieves the output files of a job that has been submitted through the glite-wms-job-submit 
command with a job description file including the OutputSandbox attribute.
After the submission, when the job has terminated its execution, the user can download the files generated 
by the job and temporarily stored on the WMS machine as specified by the OutputSandbox attribute, 
issuing the glite-wms-job-output with as input the jobId returned by the glite-wms-job-submit.


\medskip\textbf{OPTIONS}\smallskip



\textbf{-{}-version}

displays WMProxy client and servers version numbers.
If the -{}-endpoint option is specified only this service is contacted to retrieve its version. If it is not used, the endpoint contacted is that one specified with GLITE\_WMS\_WMPROXY\_ENDPOINT environment variable. If the option is not specified and the environment variable is not set, all servers listed in the WMProxyEndPoints attribute of the user configuration file are contacted. For each of these service, the returned result will be displayed on the standard output (the version numbers in case of success; otherwise the error message).




\textbf{-{}-help}

displays command usage




\textbf{-{}-config, -c <configfile>}

if the command is launched with this option, the configuration file pointed by <configfile> is used. This option is meaningless when used together with -{}-vo option




\textbf{-{}-vo <voname>}

this option allows the user to specify the Virtual Organization she/he is currently working for.
If the user proxy contains VOMS extensions then the VO specified through this option is overridden by the
default VO contained in the proxy (i.e. this option is only useful when working with non-VOMS proxies).
This option is meaningless when used together with -{}-config-vo option.




\textbf{-{}-debug}

when this option is specified, debugging information is displayed on the standard output and written also either into the default log file:


\begin{verbatim}
glite-wms-job-<command_name>_<uid>_<pid>_<time>.log
\end{verbatim}

located under the /var/tmp  directory or in the log file specified with -{}-logfile option.




\textbf{-{}-logfile <filepath>}

When this option is specified, a command log file (whose pathname is <filepath> ) is created.



\textbf{-{}-noint}

if this option is specified, every interactive question to the user is skipped and the operation is continued (when possible)





\textbf{-{}-input, -i <filepath>}

this option makes the command return the OutputSandbox files for each jobId contained in the <filepath>. 
This option can't be used if one (or more) jobIds have been already specified. 
The format of the input file must be as follows: one jobId for each line and comment lines must begin with a 
'\#' or a '\*' character. When this option is used, the list of jobIds contained in the file is displayed and 
the user is prompted for a choice. Single jobs can be selected specifying the numbers associated to the job 
identifiers separated by commas.
E.g. selects the first,the third and the fifth jobId in the list. Ranges can also be selected specifying 
ends separated by a dash. E.g. "3-6" selects jobIds in the list from third position (included) to sixth position 
(included). It is worth mentioning that it is possible to select at the same time ranges and single jobs.
E.g. selects the first job id in the list, the ids from the third to the fifth (ends included) and finally 
the eighth one.





\textbf{-{}-dir <directory\_path>}

if this option is specified, the retrieved files (previously listed by the user through the OutputSandbox 
attribute of the job description file) are stored in a directory whose name is 
\verb username_<JOB_Identifier> {}; this directory, in turn, is located in the 
location indicated by directory\_path. If -{}-dir is not specified, the directory containing the job's output 
files will be put in the destination folder specified as value of the JDL attribute OutputStorage 
(see section \ref{confattr} at page \pageref{confattr}).





\textbf{-{}-nosubdir }

job's output files are located in the destination folder (specified with -{}-dir option or through the 
OutputStorage JDL parameter) without creating any subdirectory.






\textbf{-{}-nopurge}

if this option is specified, the output sandbox on WMS node is not purged, and the command can be repeated 
(the job's output can be retrieved again).





\textbf{-{}-proto <protocol>}

this option specifies the protocol to be used for file transferring. It will be ignored when the specified 
protocol is not found among WMProxy service available protocols: in this case the default one (generally gsiftp ) 
will be used instead.
This option is only available from glite version >= 3.1.





\textbf{-{}-list-only}

when this option is specified, the output files (previously listed by the user through the OutputSandbox 
attribute of the job description file) are not retrieved. The command only displays the list of URIs indicating
 where they are currently located.





\textbf{-{}-json}

This option makes the command produce its output in JSON-compliant format, that can be parsed by proper json 
libraries for python/perl and other script languages.





\textbf{-{}-pretty-print}

This option should be used with -{}-json. Without it the JSON format is machine-oriented 
(no carriage returns, no indentations). 
-{}-pretty-print makes the JSON output easily readable by a human. Using this option without -{}-json has no effect.





\medskip
\textbf{EXAMPLES}
\smallskip


\begin{enumerate}


\item  
\begin{verbatim}glite-wms-job-output https://wmproxy.glite.it:9000/7O0j4Fequpg7M6SRJ-NvLg
\end{verbatim} 

if the operation succeeds, the \verb /tmp/<jobId-UniqueString> {}  directory contains the retrieved files


\item 
\begin{verbatim}
glite-wms-job-output --dir $HOME/my_dir \
                     https://wmproxy.glite.it:9000/7O0j4Fequpg7M6SRJ-NvLg
\end{verbatim}

if the operation succeeds, the \$HOME/my\_dir directory contains the retrieved files


\item  request for output of multiple jobs:
\begin{verbatim}
glite-wms-job-output https://wmproxy.glite.it:9000/7O0j4Fequpg7M6SRJ-NvLg \
                     https://wmproxy.glite.it:9000/wqikja_-de83jdqd \
                     https://wmproxy.glite.it:9000/jdh_wpwkd134ywhq6p
\end{verbatim}

if the operation succeeds, each \verb /tmp/<jobId-UniqueString> {} directory contains the retrieved files for the corresponding job


\item  the myids.in input file contains the jobid(s)
\begin{verbatim}
glite-wms-job-output --input myids.in
\end{verbatim}
if the operation succeeds, each \verb /tmp/<jobId-UniqueString> {} directory contains the retrieved files for the corresponding job

\end{enumerate}



\medskip
\textbf{FILES}
\smallskip

\verb voName/glite_wms.conf {}: The user configuration file. The standard 
path location is \verb /etc/glite-wms . 

\verb /tmp/x509up_u<uid> {}: A valid X509 user proxy; 
use the X509\_USER\_PROXY environment variable to override the default location



\medskip
\textbf{ENVIRONMENT}
\smallskip


\begin{itemize}



\item GLITE\_WMS\_WMPROXY\_ENDPOINT: This variable may be set to specify the endpoint URL


\item GLOBUS\_LOCATION: This variable must be set when the Globus installation is not located in the default path \verb /opt/globus {}.


\item GLOBUS\_TCP\_PORT\_RANGE="<val min> <val max>": This variable must be set to define a range of ports to be used for inbound connections in the interactivity context


\item X509\_CERT\_DIR: This variable may be set to override the default location of the trusted certificates directory, which is normally \verb /etc/grid-security/certificates {}.


\item X509\_USER\_PROXY: This variable may be set to override the default location of the user proxy credentials, which is normally \verb /tmp/x509up_u<uid> {}.


\item GLITE\_SD\_PLUGIN: If Service Discovery querying is needed, this variable can be used in order to set a specific (or more) plugin, normally bdii, rgma (or both, separated by comma)


\item LCG\_GFAL\_INFOSYS: If Service Discovery querying is needed, this variable can be used in order to set a specific Server where to perform the queries: for instance LCG\_GFAL\_INFOSYS='gridit-bdii-01.cnaf.infn.it:2170'


\end{itemize}

